%!TEX root = JLAMP-Main.tex




\subsection{Gateways versus direct communications}

In Remark \ref{rem:dircomm}, we pointed out how difficult it could be
 to try and use direct communications to connect systems
instead of using gateways. We mentioned that using direct communications
to connect systems $S$ and $S'$ of our working example, 
the automaton for role $\MM$ should be completely redesigned.
We provide now a further example supporting the
gateway approach, in which, to preserve the communication properties
the whole system needs to be redesigned. 

Let $S_1$ and $S_2$ be the systems of Figure \ref{fig:dircomm}.
It is not difficult to check that both $S_1$ and $S_2$ are ``well-behaved'' for what concerns communication properties.
 Moreover, they could be connected through $\HH$ and $\KK$ because these two roles are compatible.\\

\begin{figure}

{\footnotesize
$$
\begin{array}{@{\hspace{-6mm}}c@{\hspace{0cm}}c}
      
 
   \begin{array}{cc}
                     \begin{tikzpicture}[->,>=stealth',shorten >=1pt,auto,node distance=1.5cm,semithick]
  
  \node[state]   (one)                        {$1$};
  \node[draw=none,fill=none] (start) [above left = 0.3cm  of one]{{\tt A}};
  \node[state]            (two) [below of=one] {$2$};
  \node[state]            (three) [below of=two] {$3$};

   \path  (start) edge node {} (one) 
      %      (two)  edge                                   node {\edgelabel{JM}!{text}} (one)
           (one) edge       [bend left=100]       node [ left] {\edgelabel{AH}{!}{n}} (two)
                   edge        [bend right=100]      node [ left] {\edgelabel{AH}{!}{m}} (two)
           (two) edge   node{\edgelabel{AB}{!}{f}} (three);
           
       \end{tikzpicture}
      &
                    \begin{tikzpicture}[->,>=stealth',shorten >=1pt,auto,node distance=1.5cm,semithick]
  
  \node[state]   (one)                        {$1$};
  \node[draw=none,fill=none] (start) [above left = 0.3cm  of one]{{\tt H}};
  \node[state]            (two) [below of=one] {$2$};
  \node[state]            (three) [below of=two] {$3$};

   \path  (start) edge node {} (one) 
      %      (two)  edge                                   node {\edgelabel{JM}!{text}} (one)
           (one) edge       [bend left=100]       node [ left] {\edgelabel{AH}{?}{n}} (two)
                   edge        [bend right=100]      node [ left] {\edgelabel{AH}{?}{m}} (two)
           (two) edge   node{\edgelabel{BH}{?}{r}} (three);
           
       \end{tikzpicture}\\[8mm]
  \multicolumn{2}{c}{
                   \begin{tikzpicture}[->,>=stealth',shorten >=1pt,auto,node distance=1.5cm,semithick]
  
  \node[state]   (one)                        {$1$};
  \node[draw=none,fill=none] (start) [above left = 0.3cm  of one]{{\tt B}};
  \node[state]            (two) [below of=one] {$2$};
  \node[state]            (three) [below of=two] {$3$};

   \path  (start) edge node {} (one) 
      %      (two)  edge                                   node {\edgelabel{JM}!{text}} (one)
           (one) edge        node [ left] {\edgelabel{AB}{?}{f}}  (two)
           (two) edge   node{\edgelabel{BH}{!}{r}} (three);
           
       \end{tikzpicture}
                              }
   \end{array}                 
                              
       &
            
  
    
\end{array}
\hspace{8mm}
\begin{array}{c@{}c}
            \begin{array}{c}
       \begin{tikzpicture}[->,>=stealth',shorten >=1pt,auto,node distance=1.5cm,semithick]
  
  \node[state]   (one)                        {$1$};
  \node[draw=none,fill=none] (start) [above left = 0.3cm  of one]{{\tt K}};
  \node[state]            (two) [below  left of=one] {$2$};
  \node[state]            (three) [below right of=one] {$3$};
  \node[state]            (four) [below left of=three] {$4$};


   \path  (start) edge node {} (one) 
           (one) edge       [bend left]       node [ right] {\edgelabel{KD}{!}{n}} (three)
                   edge        [bend right]      node [ left] {\edgelabel{KC}{!}{m}} (two)
           (two) edge [bend right]   node[ left]{\edgelabel{KC}{!}{r}} (four)
           (three) edge [bend left]  node[ right]{\edgelabel{KD}{!}{r}} (four);
           
       \end{tikzpicture}
       \\
       \begin{array}{c}
                     \begin{tikzpicture}[->,>=stealth',shorten >=1pt,auto,node distance=1.5cm,semithick]
                     \node[state]           (one)                        {$1$};
                     \node[draw=none,fill=none] (start) [above left = 0.3cm  of one]{$\mathtt{C}$};
                     \node[state]            (two) [below left  of=one] {$2$};
                      \node[state]            (three) [below of=two] {$3$};
                      \node[state]            (four) [below of=three] {$4$};
                     \node[state]            (five) [below right of=one] {$5$};
                     
             \path  (start) edge node {} (one) 
                      (one)  edge           node [pos=0.2, left] {\edgelabel{KC}{?}{m}} (two)
                                edge           node [pos=0.2, right] {\edgelabel{DC}{?}{g}} (five)
                      (two)  edge           node [left] {\edgelabel{KC}{?}{r}} (three)        
                      (three)  edge         node [left] {\edgelabel{CD}{!}{g}} (four);
                     \end{tikzpicture}
      \end{array}
       \end{array}
 &
     \begin{array}{c}
                     \begin{tikzpicture}[->,>=stealth',shorten >=1pt,auto,node distance=1.5cm,semithick]
                     \node[state]           (one)                        {$1$};
                     \node[draw=none,fill=none] (start) [above left = 0.3cm  of one]{$\mathtt{D}$};
                     \node[state]            (two) [below left  of=one] {$2$};
                      \node[state]            (three) [below of=two] {$3$};
                      \node[state]            (four) [below of=three] {$4$};
                     \node[state]            (five) [below right of=one] {$5$};
                     
             \path  (start) edge node {} (one) 
                      (one)  edge           node [pos=0.2, left] {\edgelabel{KD}{?}{n}} (two)
                                edge           node [pos=0.2, right] {\edgelabel{CD}{?}{g}} (five)
                      (two)  edge           node [left] {\edgelabel{KD}{?}{r}} (three)        
                      (three)  edge         node [left] {\edgelabel{DC}{!}{g}} (four);
                     \end{tikzpicture}
      \end{array}
\end{array}
$$
}

\caption{$S_1$ and $S_2$ for the direct-communications example}\label{fig:dircomm}
\end{figure}

 Now let us try using direct communications to connect the two systems, getting rid of the gateways $\HH$ and $\KK$.
 For what concerns role $\AA$, one could think of simply modifying the target roles of communications in the following way:
 
 {\footnotesize
$$
\begin{tikzpicture}[->,>=stealth',shorten >=1pt,auto,node distance=1.5cm,semithick]
  
  \node[state]   (one)                        {$1$};
  \node[draw=none,fill=none] (start) [above left = 0.3cm  of one]{{\tt A}};
  \node[state]            (two) [below of=one] {$2$};
  \node[state]            (three) [below of=two] {$3$};

   \path  (start) edge node {} (one) 
      %      (two)  edge                                   node {\edgelabel{JM}!{text}} (one)
           (one) edge       [bend left=100]       node [ left] {\edgelabel{AD}{!}{n}} (two)
                   edge        [bend right=100]      node [ left] {\edgelabel{AC}{!}{m}} (two)
           (two) edge   node{\edgelabel{AB}{!}{f}} (three);
           
       \end{tikzpicture}
$$
}

For what concerns roles $\CC$ and $\DD$, the source roles of some of their communication
could be simply renamed as follows:

{\footnotesize
$$
\begin{array}{c@{}c}
            \begin{array}{c}
       
       \begin{array}{c}
                     \begin{tikzpicture}[->,>=stealth',shorten >=1pt,auto,node distance=1.5cm,semithick]
                     \node[state]           (one)                        {$1$};
                     \node[draw=none,fill=none] (start) [above left = 0.3cm  of one]{$\mathtt{C}$};
                     \node[state]            (two) [below left  of=one] {$2$};
                      \node[state]            (three) [below of=two] {$3$};
                      \node[state]            (four) [below of=three] {$4$};
                     \node[state]            (five) [below right of=one] {$5$};
                     
             \path  (start) edge node {} (one) 
                      (one)  edge           node [pos=0.2, left] {\edgelabel{AC}{?}{m}} (two)
                                edge           node [pos=0.2, right] {\edgelabel{DC}{?}{g}} (five)
                      (two)  edge           node [left] {\edgelabel{BC}{?}{r}} (three)        
                      (three)  edge         node [left] {\edgelabel{CD}{!}{g}} (four);
                     \end{tikzpicture}
      \end{array}
       \end{array}
 &
     \begin{array}{c}
                     \begin{tikzpicture}[->,>=stealth',shorten >=1pt,auto,node distance=1.5cm,semithick]
                     \node[state]           (one)                        {$1$};
                     \node[draw=none,fill=none] (start) [above left = 0.3cm  of one]{$\mathtt{D}$};
                     \node[state]            (two) [below left  of=one] {$2$};
                      \node[state]            (three) [below of=two] {$3$};
                      \node[state]            (four) [below of=three] {$4$};
                     \node[state]            (five) [below right of=one] {$5$};
                     
             \path  (start) edge node {} (one) 
                      (one)  edge           node [pos=0.2, left] {\edgelabel{AD}{?}{n}} (two)
                                edge           node [pos=0.2, right] {\edgelabel{CD}{?}{g}} (five)
                      (two)  edge           node [left] {\edgelabel{BD}{?}{r}} (three)        
                      (three)  edge         node [left] {\edgelabel{DC}{!}{g}} (four);
                     \end{tikzpicture}
      \end{array}
\end{array}
$$
}

It remains to modify $\BB$ for getting direct communications.
The transition from state 2 to 3 could be relabelled either to $\BB\CC\texttt{!r}$ or to
$\BB\DD\texttt{!r}$. 
It is, however, not difficult to check that none of the two modification of $\BB$ would lead to a system enjoying the progress property.
In fact, $\BB$ has no way to know whether $\AA$ has chosen to send $\texttt{m}$ to $\CC$ or
$\texttt{n}$ to $\DD$. Without such an information, the system could reach a weak-deadlock configuration.

The only possibility to recover progress would be to redesign the system as shown in Figure \ref{fig:recprogr}.
In particular, we must redesign the machines for $\AA$ and for $\BB$, by adding several new states and communications.






\begin{figure}
{\footnotesize
$$
\begin{array}{@{\hspace{-6mm}}c@{\hspace{0cm}}c}
      
 
   \begin{array}{cc}
                     \begin{tikzpicture}[->,>=stealth',shorten >=1pt,auto,node distance=1.5cm,semithick]
  
  \node[state]   (one)                        {$1$};
  \node[draw=none,fill=none] (start) [above left = 0.3cm  of one]{{\tt A}};
  \node[state]            (two) [below left of=one] {$2$};
  \node[state]            (three) [below of=two] {$3$};
  \node[state]            (four) [below right of=three] {$4$};
  \node[state]            (five) [below right of=one] {$5$};
  \node[state]            (six) [below of=five] {$6$};

   \path  (start) edge node {} (one) 
            (one) edge       [bend right]       node [ left] {\edgelabel{AC}{!}{m}} (two)
                   edge        [bend left]      node [ right] {\edgelabel{AD}{!}{n}} (five)
           (two) edge   node{\edgelabel{AB}{!}{f}} (three)
           (five) edge   node{\edgelabel{AB}{!}{f}} (six)
           (three) edge  [bend right] node  [ left]{\edgelabel{AB}{!}{m}} (four)
           (six) edge  [bend left] node{\edgelabel{AB}{!}{n}} (four);
           
       \end{tikzpicture}
      &
                   \\[8mm]
  \multicolumn{2}{c}{
                  \begin{tikzpicture}[->,>=stealth',shorten >=1pt,auto,node distance=1.5cm,semithick]
  
  \node[state]   (one)                        {$1$};
  \node[draw=none,fill=none] (start) [above left = 0.3cm  of one]{{\tt B}};
  \node[state]            (two) [below of=one] {$2$};
  \node[state]            (three) [below left  of=two] {$3$};
  \node[state]            (four) [below right of=three] {$4$};
  \node[state]            (five) [below right of=two] {$5$};


   \path  (start) edge node {} (one) 
           (one) edge   node{\edgelabel{AB}{?}{f}} (two)
            (two) edge       [bend right]       node [ left] {\edgelabel{AB}{?}{m}} (three)
                   edge        [bend left]      node [ right] {\edgelabel{AB}{?}{n}} (five)
           (three) edge  [bend right] node  [ left]{\edgelabel{BC}{!}{m}} (four)
           (six) edge  [bend left] node{\edgelabel{BD}{!}{n}} (four);
           
       \end{tikzpicture}
                              }
   \end{array}                 
                              
       &
            
  
    
\end{array}
\hspace{8mm}
\begin{array}{c@{}c}
            \begin{array}{c}
       
       \begin{array}{c}
                     \begin{tikzpicture}[->,>=stealth',shorten >=1pt,auto,node distance=1.5cm,semithick]
                     \node[state]           (one)                        {$1$};
                     \node[draw=none,fill=none] (start) [above left = 0.3cm  of one]{$\mathtt{C}$};
                     \node[state]            (two) [below left  of=one] {$2$};
                      \node[state]            (three) [below of=two] {$3$};
                      \node[state]            (four) [below of=three] {$4$};
                     \node[state]            (five) [below right of=one] {$5$};
                     
             \path  (start) edge node {} (one) 
                      (one)  edge           node [pos=0.2, left] {\edgelabel{AC}{?}{m}} (two)
                                edge           node [pos=0.2, right] {\edgelabel{DC}{?}{g}} (five)
                      (two)  edge           node [left] {\edgelabel{BC}{?}{r}} (three)        
                      (three)  edge         node [left] {\edgelabel{CD}{!}{g}} (four);
                     \end{tikzpicture}
      \end{array}
       \end{array}
 &
     \begin{array}{c}
                     \begin{tikzpicture}[->,>=stealth',shorten >=1pt,auto,node distance=1.5cm,semithick]
                     \node[state]           (one)                        {$1$};
                     \node[draw=none,fill=none] (start) [above left = 0.3cm  of one]{$\mathtt{D}$};
                     \node[state]            (two) [below left  of=one] {$2$};
                      \node[state]            (three) [below of=two] {$3$};
                      \node[state]            (four) [below of=three] {$4$};
                     \node[state]            (five) [below right of=one] {$5$};
                     
             \path  (start) edge node {} (one) 
                      (one)  edge           node [pos=0.2, left] {\edgelabel{AD}{?}{n}} (two)
                                edge           node [pos=0.2, right] {\edgelabel{CD}{?}{g}} (five)
                      (two)  edge           node [left] {\edgelabel{BD}{?}{r}} (three)        
                      (three)  edge         node [left] {\edgelabel{DC}{!}{g}} (four);
                     \end{tikzpicture}
      \end{array}
\end{array}
$$
}
\caption{Recovering progress in the direct-communications example}\label{fig:recprogr}
\end{figure}



 
 
 


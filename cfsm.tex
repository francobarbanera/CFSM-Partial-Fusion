%!TEX root = Main-CFSM-partial-fusion.tex
\section{Systems of Communicating Finite State Machines}
\label{sect:cfsm}

Communicating Finite State Machines (CFSM)   are 
 a widely investigated
formalism for the description and analysis of distributed systems, originally proposed in \cite{BZ83}.
CFSM are a variant of finite state I/O-automata that represent processes which communicate by asynchronous exchanges of messages via FIFO channels. 
We now recall (partly following \cite{CF05,DY12,TY15,BdLH19}) the definitions of CFSM and systems of CFSMs.

We assume %given 
a countably infinite set  
$\roles_\mathfrak{U}$ of participant names (ranged over by $\ttp,\ttq,\ttr,\HH,\KK,\ttv,\ttw\ldots$) and a countably infinite alphabet $\mathbb{A}_\mathfrak{U}$ 
of messages (ranged over by $\msg[a]$, $\msg[b]$, $\msg[c]$, $\msg[m],\ldots$).\\

\begin{definition}[FSA and $\varepsilon$-FSA with no final states]
 
\end{definition}
** Explain that we start from the very general definition of FSA since we shall use results on FSA like
the elimination of $\varepsilon$-transitions-

\begin{definition}[CFSM**TO BE DEFINED SIMPLY AS FSA on $\textit{Act}_{\roles,\mathbb{A}}$ with NO FINAL STATES**]\label{def:cfsm}%\hfill\\
Let $\roles$  and $\mathbb{A}$ be finite subsets of $\roles_\mathfrak{U}$ and $\mathbb{A}_\mathfrak{U}$, respectively.
\begin{enumerate}[i)] 
\item
The set $C_\roles$ of {\em channels} over $\roles$ is defined by\ \
$C_\roles=\Set{\ttp\ttq \mid \ttp,\ttq\in \roles, \ttp\neq\ttq}$
\item
The set $\mathit{Act}_{\roles,\mathbb{A}}$ of {\em actions}  over $\roles$ and $\mathbb{A}$ is defined by\ 
$\textit{Act}_{\roles,\mathbb{A}} = C_\roles\times\Set{!,?}\times\mathbb{A}$

The {\em subject} of an output action $\ttp\ttq!\msg[m]$ and of an input action $\ttq\ttp?\msg[m]$ is 
$\ttp$.
\item
\label{def:cfsm-iii}
A {\em communicating finite state machine over} $\roles$ \emph{and} $\mathbb{A}$
is a finite transition system given by a tuple\\
\centerline{ $M=(Q,q_0,\mathbb{A},\delta)$ }
where $Q$ is a finite set of states, $q_0\in Q$ is the initial state, and
$\delta\subseteq Q\times\textit{Act}_{\roles,\mathbb{A}}\times Q$ is a set of transitions
such that all the actions have the same subject, to which we refer as the {\em name} of $M$.
\end{enumerate}
\end{definition}
\noindent
We shall write $M_{\ttp}$ to denote a CFSM with name $\ttp$. 
Where no ambiguity arises we shall refer to a CFSM by its name.


Notice that the above definition of CFSM is generic with respect to the underlying sets
$\roles$ and $\mathbb{A}$.
This is necessary,  since we shall not deal with a single system of CFSMs but with an arbitrary number of  systems of CFSMs that can be {\em composed}.
We shall write $C$ and $\mathit{Act}$ instead of $C_\roles$ and $\mathit{Act}_{\roles,\mathbb{A}}$ when no ambiguity can arise.
%\brc Let us check whether we ever use $C$ and $\mathit{Act}$.
%But even then, I would prefer to see $C_\roles$ and $\mathit{Act}_{\roles,\mathbb{A}}$ which is minimal longer.\erc
 We assume $\elle,\elle',\ldots$ to range over $\textit{Act}$
%$\varphi,\varphi',\ldots$ to range over $\textit{Act}^*$ (the set of finite words over $\textit{Act}$), 
and $w,w',\ldots$ to range over $\mathbb{A}^*$ (the set of finite words over $\mathbb{A}$).
The symbol $\varepsilon\,(\notin \mathbb{A}\cup\textit{Act})$ denotes the empty word and 
$\mid w\mid$ the length of a word $w\in \mathbb{A}^*$.
%$\mid v\mid$ the lenght of a word $v\in \textit{Act}^*\cup\mathbb{A}^*$.

%%>>>>>>>> DEFINITIONS NOT USED in the present paper
%Given a word $v$ with prefix $v'$, i.e. such that $v=v'\cdot v''$ for a certain $v''$, we define $v\setminus v' =v''$.
%Moreover, given a  word $v$ with $\msg[a]$ as last  element, i.e. $v=v'\cdot \msg[a]$ for a certain (possibly empty) $v'$, we define
%$\mathsf{init}(v) = v'$ and  $\mathsf{last}(v) = \msg[a]$. 
%Moreover, we shall denote by $\widetilde{v}$ the reverse of the word $v$. \\


The transitions of a CFSM are labelled by actions; a label $\tts\ttr!\msg[a]$ represents
the asynchronous sending of message $\msg[a]$ from machine $\tts$ to $\ttr$ through channel $\tts\ttr$ and, dually,
$\tts\ttr?\msg[a]$ represents the reception (consumption) of $\msg[a]$ by $\ttr$ from channel
$\tts\ttr$. 

 Given a CFSM $M=(Q,q_0,\mathbb{A},\delta)$,
we also define \\
\centerline{$\inn{M}=\Set{\msg[a] \mid (\_,\_\,\_?\msg[a],\_)\in \delta }$
\quad \text{ and }\quad $\outt{M}=\Set{\msg[a] \mid (\_,\_\,\_!\msg[a],\_)\in \delta }$.}
 If $M$ is a CFSM with name $\ttp$, we also write $\inn{\ttp}$ for $\inn{M}$ and
 $\outt{\ttp}$ for $\outt{M}$.
  Note that, in concrete examples, the name of a CFSM together with its input and output messages can be graphically depicted as in~\cref{fig:four-ips}. 




%We write $\lang{M}\subseteq\textit{Act}^*$ for
%the language over $\textit{Act}$ accepted by the automaton corresponding
%to machine $M$, where each state of $M$ is an accepting state. 
A state
$q\in Q$ with no outgoing transition is {\em final}; 
$q$ is a {\em sending} (resp. {\em receiving}) state if it is not final and
all outgoing transitions are labelled with sending (resp. receiving) actions;
$q$  is a {\em mixed} state if there are at least two outgoing transitions such that one is labelled with a sending action and the other one is labelled with a receiving action.



%
%  ?!-DETERMINISM DEFINITION <<<<<<<<<<<<<<<<<<<<<<<<
% 
%\vspace{2mm}
%A CFSM $M = (Q,q_0,\mathbb{A},\delta)$ is:
%\begin{enumerate}[a)]
%\item
% {\em deterministic} if for all transitions:\quad % states $q\in Q$ and all actions $\elle$: 
%$(q,\elle, q'), (q,\elle,q'')\in \delta$ imply $q'=q''$;
%\item
%{\em ?-deterministic} (resp. {\em !-deterministic}) if for all transitions:\\ % all states  $q\in Q$ and all actions:\\
%$\qquad$ $(q,\ttr\tts?\msg[a], q'), (q,\ttp\ttq?\msg[a],q'')\in \delta$ (resp. $(q,\ttr\tts!\msg[a], q'), (q,\ttp\ttq!\msg[a],q'')\in \delta$) imply $q'=q''$;\footnote{Note that, by Definition \ref{def:cfsm}(\ref{def:cfsm-iii}), we have
%necessarily that $\tts=\ttq$ in the clause for ?-determinism and $\ttr=\ttp$ in the one for
%!-determinism.}
%\item
%{\em ?!-deterministic} if it is both ?-deterministic and !-deterministic.
%\end{enumerate}
%
%The notion of ?!-deterministic machine is more demanding than in usual CFSM settings. It will be needed in order to guarantee preservation of communication properties when systems are connected. 
%Note that a ?!-deterministic CFSM is also deterministic, but the converse does not hold
%(since the channel names are abstracted away in the definition of ?!-determinism). \\

A {\em communicating system}, called ``protocol'' in \cite{BZ83}, is a finite set of CFSMs.
% over some vocabulary of messages such that senders and receivers are identified by the 
%names of CFSMs. 
 In~\cite{CF05,DY12,TY15} the names of the CFSMs in a system are called {\em roles}. In the present paper we call them {\em participants}.
 

The dynamics of a system are
formalised as a transition relation on configurations, where a configuration is a
pair of tuples: a tuple of states of the machines in the system and a tuple of buffers representing the content of the channels. A buffer is described as an element of $\mathbb{A^*}$. 

\begin{definition}[Communicating system and configuration]%\hfill\\
Let $\roles$  and $\mathbb{A}$ be as in Def.~\ref{def:cfsm}.
\begin{enumerate}[i)]
\item
A {\em communicating system (CS)
over} $\roles$ \emph{and} $\mathbb{A}$ is a  set  %tuple 
$S= (M_\ttp)_{\ttp\in\roles}$
%\centerline{$S= (M_\ttp)_{\ttp\in\roles}$}
where\\
%\\
%-  $\roles\subseteq_{\text{fin}}\roles_\mathfrak{U}$   is the set of {\em roles} (participants) of $S$, and\\
for each $\ttp\in \roles$,
$M_\ttp=(Q_\ttp,q_{0\ttp},\mathbb{A},\delta_\ttp)$ is a CFSM  over $\roles$ and $\mathbb{A}$.
\item
A {\em configuration} of a system $S$ is a pair $s = (\vec{q},\vec{w})$
%\centerline{$s = (\vec{q},\vec{w})$}
where\\
\centerline{$\vec{q}= (q_\ttp)_{\ttp\in\roles}$ with $q_\ttp \in Q_\ttp$,
\qquad and \qquad  $\vec{w}  = (w_{\ttp\ttq})_{\ttp\ttq\in C}$ with $w_{\ttp\ttq}\in\mathbb{A^*}$.}
%\begin{itemize}
%\item[-]  $\vec{q}= (q_\ttp)_{\ttp\in\roles}$ with $q_\ttp \in Q_\ttp$,
%\item[-]  $\vec{w}  = (w_{\ttp\ttq})_{\ttp\ttq\in C}$ with $w_{\ttp\ttq}\in\mathbb{A^*}$.
%\end{itemize}

The component $\vec{q}$ is the {\em control state\/} of the system and $q_\ttp \in Q_\ttp$ is the 
{\em local state\/} of machine $M_\ttp$. 
The component $\vec{w}$ represents the state of the channels of the system and $w_{\ttp\ttq} \in \mathbb{A}^*$ is the state of the channel $\ttp\ttq$, i.e. the messages sent from $\ttp$ to $\ttq$. The initial configuration of $S$ is $s_0=  (\vec{q_0},\vec{\varepsilon})$
with $\vec{{q_0}} = (q_{0_\ttp})_{\ttp\in\roles}$.
\end{enumerate}
\end{definition}

\noindent
In the following we will often denote a communicating system $(M_{\ttp})_{\ttp\in \Set{\ttr_i}_{i\in I}}$ by $(M_{\ttr_i})_{i\in I}$.



\begin{definition}[Transitions and reachable configurations]
Let $S$ be a communicating system over $\roles$ and $\mathbb{A}$, and let $s= (\vec{q},\vec{w})$ and $s'= (\vec{q'},\vec{w'})$ 
be two configurations of $S$. %\\
Configuration $s'$ {\em is reachable from} $s$
{\em by firing  a transition} with action $\elle$, written $s\lts{\elle}s'$, if there is $\msg[a]\in\mathbb{A}$
such that one of the following conditions holds:
%\begin{center}
%\begin{tabular}{ r l r l}
%$1.$ & $\elle = \tts\ttr!\msg[a]$ and $(q_\tts,\elle,q'_\tts)\in\delta_\tts$ and &  
%$2.$ & $\elle = \tts\ttr?\msg[a]$ and $(q_\ttr,\elle,q'_\ttr)\in\delta_\ttr$ and\\
%        & $a)$ for all $\ttp\neq\tts: ~ q'_\ttp =  q_\ttp$  and &
%        & $a)$ for all $\ttp\neq\ttr: ~ q'_\ttp =  q_\ttp$  and \\
%        & $b)$ $w'_{\tts\ttr} =  w_{\tts\ttr}\cdot \msg[a]$ and for all $\ttp\ttq\neq\tts\ttr: ~ w'_{\ttp\ttq} =  w_{\ttp\ttq}$; &
%        & $b)$  $w_{\tts\ttr} =  \msg[a]\cdot w'_{\tts\ttr}$ and for all $\ttp\ttq\neq\tts\ttr: ~w'_{\ttp\ttq} =  w_{\ttp\ttq}$.
%\end{tabular}
%\end{center}
\begin{enumerate}
\item
$\elle = \tts\ttr!\msg[a]$ and $(q_\tts,\elle,q'_\tts)\in\delta_\tts$ and
\qquad \begin{enumerate}[a)]
\item
for all $\ttp\neq\tts: ~ q'_\ttp =  q_\ttp$  and
\item
$w'_{\tts\ttr} =  w_{\tts\ttr}\cdot \msg[a]$ and for all $\ttp\ttq\neq\tts\ttr: ~ w'_{\ttp\ttq} =  w_{\ttp\ttq}$;
\end{enumerate}
\item 
$\elle = \tts\ttr?\msg[a]$ and $(q_\ttr,\elle,q'_\ttr)\in\delta_\ttr$ and
\begin{enumerate}[a)]
\item
for all $\ttp\neq\ttr: ~ q'_\ttp =  q_\ttp$  and
\item
$w_{\tts\ttr} =  \msg[a]\cdot w'_{\tts\ttr}$ and for all $\ttp\ttq\neq\tts\ttr: ~w'_{\ttp\ttq} =  w_{\ttp\ttq}$.
\end{enumerate}
\end{enumerate}
We write $s\lts{}s'$ if there exists $\elle$ such that  $s\lts{\elle}s'$
 and we write $s\notlts{}\hspace{2mm}$ if no $s'$ and no $\elle$ exist with
$s\lts{\elle}s'$.
As usual, we denote the reflexive and transitive 
closure of $\lts{}$ by $\to^*$.
The set of {\em reachable configurations} of S is $\RS(S) = \Set{s \mid s_0 \to^* s}.$
\end{definition}
\noindent
According to the above definition, communication happens asynchronously via buffered channels following the FIFO principle.\\


\begin{example}{\em
As sketched in the Introduction, CFSMs can be graphically represented as in the following
simple example of a communicating system describing an asynchronous protocol of mutually exclusive use of
a resource by two participants $\msg[p]$ and $\msg[q]$ controlled  by a participant $\msg[c]$.
$$
\begin{tikzpicture}[mycfsm]
  \node[state]           (idle)                        {\small $\mathit{Idle}$};
  \node[draw=none,fill=none] (wpc) [right of = idle, xshift=10mm,yshift=-4mm]{\small$\mathit{w}_{\ptp[p]\ptp[c]}=\varepsilon$};
   \node[draw=none,fill=none] (start) [above left = 0.3cm  of idle]{$\ptp[p]$};
  \node[state]           (wait) [below right of=idle] {\small $\mathit{Wait}$};
  \node[state]            (busy) [below left of=wait] {\small $\mathit{Busy}$};
    \node[draw=none,fill=none] (wcp) [right of = busy, xshift=10mm,yshift=4mm]{\small$\mathit{w}_{\ptp[c]\ptp[p]}=\varepsilon$};

   \path  (start) edge node {} (idle) 
             (idle)        edge   [bend left]      node [above]  {${\ttp\ttc}!{\msg[request]}$} (wait)
             (busy)        edge  [bend left]         node [above] {${\ttp\ttc}!{\msg[release]}$} (idle)
             (wait)  edge  [bend left]      node [below] {${\ttc\ttp}?{\msg[granted]}$} (busy);       
             \end{tikzpicture}
\qquad
\begin{tikzpicture}[mycfsm]
  \node[state]           (noreq)                        {\small $\mathit{no\text{-}reqs}$};
   \node[draw=none,fill=none] (start) [above left = 0.3cm  of idle]{$\ptp[c]$};
  \node[state]           (qreq) [below right of=idle,xshift=8mm] {\small $\mathit{\mathtt{q}\text{-}req}$};
  \node[state]           (preq) [below left of=idle,xshift=-8mm] {\small $\mathit{\mathtt{p}\text{-}req}$};
  \node[state]            (qgrant) [below left of=wait,xshift=8mm,yshift=-4mm] {\small $\mathit{\mathit{\mathtt{q}\text{-}grant}}$};
  \node[state]            (pgrant) [below left of=wait,xshift=-8mm,yshift=-4mm] {\small $\mathit{\mathit{\mathtt{p}\text{-}grant}}$};
%
   \path  (start) edge node {} (noreq) 
             (noreq)        edge   [bend left]      node [above]  {${\ttq\ttc}?{\msg[request]}$} (qreq)
             (qgrant)        edge          node [above] {${\ttq\ttc}?{\msg[release]}$} (noreq)
             (qreq)  edge  [bend left]      node [below] {${\ttc\ttq}!{\msg[granted]}$} (qgrant)
             (noreq)        edge   [bend right]      node [above]  {${\ttp\ttc}?{\msg[request]}$} (preq)
             (pgrant)        edge          node [above] {${\ttp\ttc}?{\msg[release]}$} (noreq)
             (preq)  edge  [bend right]      node [below] {${\ttc\ttp}!{\msg[granted]}$} (pgrant);       
             \end{tikzpicture}
\qquad
\begin{tikzpicture}[mycfsm]
  \node[state]           (idle)                        {\small $\mathit{Idle}$};
  \node[draw=none,fill=none] (wpc) [left of = idle, xshift=-10mm,yshift=-4mm]{\small$\mathit{w}_{\ptp[q]\ptp[c]}=\varepsilon$};
   \node[draw=none,fill=none] (start) [above left = 0.3cm  of idle]{$\ptp[q]$};
  \node[state]           (wait) [below right of=idle] {\small $\mathit{Wait}$};
  \node[state]            (busy) [below left of=wait] {\small $\mathit{Busy}$};
    \node[draw=none,fill=none] (wcp) [left of = busy, xshift=-10mm,yshift=4mm]{\small$\mathit{w}_{\ptp[c]\ptp[q]}=\varepsilon$};

   \path  (start) edge node {} (idle) 
             (idle)        edge   [bend left]      node [above]  {${\ttq\ttc}!{\msg[request]}$} (wait)
             (busy)        edge  [bend left]         node [above] {${\ttq\ttc}!{\msg[release]}$} (idle)
             (wait)  edge  [bend left]      node [below] {${\ttc\ttq}?{\msg[granted]}$} (busy);       
             \end{tikzpicture}
 $$
 The following configuration
 $$
 ((\mathit{Idle}_{\ttp},\mathit{\mathtt{q}\text{-}grant}_{\ttc},\mathit{Wait}_{\ttq}),
  (\mathit{w}_{\ptp[p]\ptp[c]}=\langle\msg[release]\rangle,
   \mathit{w}_{\ptp[c]\ptp[p]}=\varepsilon,
   \mathit{w}_{\ptp[q]\ptp[c]}=\langle\msg[request]\rangle,
    \mathit{w}_{\ptp[c]\ptp[q]}=\varepsilon)
)
$$
is reachable from the initial configuration after that both $\ptp[p]$ and $\ptp[q]$ have requested the resource,
$\ptp[c]$ has granted it to $\ptp[p]$ and, after acquiring it, $\ptp[p]$ has sent to $\ptp[c]$ the information
about its release. In the rest of the paper we do not show channels in drawings 
%draw the channel 
when their representation is not strictly necessary.
\finex
}\end{example}


%DEFINITIONS NOT USED in the present paper 
%%>>>>>>>>>>>
%We shall use $\xi, \xi', \ldots$ to range over sequences of transitions of the form 
%$s_1\lts{\elle_2}s_2\lts{\elle_3} \ldots \lts{\elle_{n-1}}s_{n-1}\lts{\elle_{n}}s_{n}$.
%If $\xi = s_1\lts{\elle_2} \ldots \lts{\elle_{n}}s_{n}$ we denote by $|\xi|$ its length, defined as $|\xi| = n-1$.
%In case $n=1$ we have a degenerate transition sequence of lenght $0$ made of a single configuration.\\
%We shall denote by
%$\upto{\xi}{i}$ the subsequence of the first $i$ transitions of a sequence $\xi$.\\
%Let $\xi$ be a sequence of the form $s_1\lts{\elle_2} \ldots \lts{\elle_{n-1}}s_{n-1}$, and let $s_{n-1}\lts{\elle_n}s_n$.
%We shall denote by $\xi\lts{\elle_n}s_n$  the transition sequence
%$s_1\lts{\elle_2} \ldots \lts{\elle_{n-1}}s_{n-1}\lts{\elle_n}s_n$.





%%%%%%%%%%%%%%%%%%%%%
%\textbf{Rolf: I have changed reference \cite{BZ83} in the title of the next definition to  \cite{DY12} because the latter uses exactly the definition that you give in i) below
%and which is different from our submitted paper.\\
%I have adjusted the proofs later to the new deadlock definition which is,
%if there are final states, in general not equivalent to the old one given in the submission.\\
%\cite{DY12}
%and~\cite{CF05} use the same definition of deadlock as in i) below. 
%The definition in~\cite{TY15} is different (it is the one used in our submitted version). Also the definition in~\cite{TG18} is different, see page 23 in~\cite{TG18}.
%There they say ``This defintion is adpated from~\cite{CF05}.''  
%By ``adapted'' they mean obviously that the definition has been slightly changed.
%The deadlock definition in~\cite{BZ83} is still different, because there a deadlock would already occur if all CFSMs are in a final state. This would also be a deadlock 
%in~\cite{TG18} if not all buffers are empty. 
%\\
%In summay:\\
%Deadlock in~\cite{CF05} = deadlock in~\cite{DY12} = deadlock above in i).\\
%Deadlock in~\cite{DY12} implies deadlock in~\cite{TY15} implies deadlock in~\cite{TG18} implies deadlock in~\cite{BZ83}.} \\
%%%%%%%%%%%%%%%%%%%%%%%






%\begin{definition}[Interfacing Policy]\label{def:intpol}
%An {\em interfacing policy} $\inp$ for a multiparty session $\Pi_{i\in I}{\pP{\hh_i}{\PH_i}}$ is a multiparty session $\Pi_{i\in I}{\pP{\hh_i}{\PK_i}}$ such that
%$\PK_i\in\GS{\PH_i}{\SP\setminus\set{\hh_i}}$ for all $i\in I$,  where $\SP=\set{\hh_i\mid i\in I}$.
%% $\SP=\set{\hh_i\mid i\in I}$ and $\PK_i\in\GS{\PH_i}{\SP\setminus\set{\hh_i}}$ for all $i\in I$.
%An interfacing policy is {\em valid} if $\inp$ is typable.
%\end{definition}

%\begin{example}[Interfacing policies]\label{simplewe3}
% Let us consider the four sessions of Example~\ref{simplewe4}. 
%Then an interfacing policy for the multiparty session $\Pi_{i=1}^{4}\pP{\hh_i}{\PH_i}$ is the multiparty session $\Pi_{i=1}^{4}\pP{\hh_i}{\PK_i}$
% where
% \Cline{
% \PK_1=
%\hh_3 ?\msg{start}.\,\hh_4!\msg{react}.\hh_2 !
%\left\{ \begin{array}{l}                                                                                                                                           \msg{rc}.\,\hh_2 ?\msg{img}.\,\PK_1                                                                                                                                \\
%\msg{nc}.\,\hh_2 ?\msg{img}.\,\PK_1                                                                          \end{array}\right.
%}
% \Cline{\PK_2=
%\hh_3 !\msg{react}.\,\hh_4 ?\msg{pars}.\,\hh_1 ?
%\left\{ \begin{array}{l}                                                                                                                                           \msg{rc}.\,\hh_1!\msg{img}.\,\PK_2                                                                                                                                \\
%\msg{nc}.\,\hh_1!\msg{img}.\,\PK_2
%       \end{array}\right.
%}
% \Cline{\PK_3=
%\hh_1 !\msg{start}.\,\hh_2 ?\msg{react}.\,\PK_3
%\qquad
% \PK_4=
%\hh_1 ?\msg{react}.\,\hh_2 ! \msg{pars}.\,\PK_4
%}
%
%\noindent
% This policy is valid, since the multiparty session $\Pi_{i=1}^{4}\pP{\hh_i}{\PK_i}$ can be typed by the following global type
% \Cline{
% \G=
%\hh_3\to\hh_1{:}\msg{start}.\,\hh_2\to\hh_3{:}\msg{react}.\,\hh_1\to\hh_4{:}\msg{react}.\, \hat\G}
%where
%\Cline{ \hat\G=\hh_4\to\hh_2{:}\msg{pars}.\,
%\hh_1\to\hh_2{:}
%\left\{ \begin{array}{l}                                                                                                                                           \msg{rc}.\,\hh_2\to\hh_1{:}\msg{img}.\,\G \\
%\msg{nc}.\,\hh_2\to\hh_1{:}\msg{img}.\,\G                                                                          \end{array}\right.}
%Note that, according to the above interfacing policy, the greeting depends on the reactions sent by the sensor driven by $\pq$.
% It is not difficult to check that there exists another valid interfacing policy for $\Pi_{i=1}^4{\pP{\hh_i}{\PH_i}}$, namely the one according to which the greeting depends on the reactions sent by the sensor driven by $\pp$.  I.e.  also  $\Pi_{i=1}^{4}\pP{\hh_i}{\PK_i'}$  is  an interfacing policy for the multiparty session $\Pi_{i=1}^{4}\pP{\hh_i}{\PH_i}$ 
% where
% \Cline{
% \PK'_1=
%\hh_3 ?\msg{start}.\,  \hh_3 !\msg{react}.\,  %\hh_4!\msg{react}.
%\hh_2 !
%\left\{ \begin{array}{l}                                                                                                                                           \msg{rc}.\,\hh_2 ?\msg{img}.\,\PK'_1                                                                                                                                \\
%\msg{nc}.\,\hh_2 ?\msg{img}.\,  \PK'_1  %\PK_1                                                                         
%\end{array}\right.
%}
% \Cline{\PK'_2=  \hh_4 !\msg{react}.\, 
%\hh_4 ?\msg{pars}.\,\hh_1 ?
%\left\{ \begin{array}{l}                                                                                                                                           \msg{rc}.\,\hh_1!\msg{img}.\,  \PK'_2  %\PK_2                                                                                                                               
% \\
%\msg{nc}.\,\hh_1!\msg{img}.\,\PK'_2
%       \end{array}\right.
%}
% \Cline{\PK'_3=
%\hh_1 !\msg{start}.\,\hh_1 ?\msg{react}.\,\PK'_3
%\qquad
% \PK'_4=
% \hh_2  %\hh_1 
%?\msg{react}.\,\hh_2 ! \msg{pars}.\,\PK'_4
%}
%
%\noindent
% This policy is valid, since the multiparty session $\Pi_{i=1}^{4}\pP{\hh_i}{\PK'_i}$ can be typed by the following global type
% \Cline{
% \G'=
%\hh_3\to\hh_1{:}\msg{start}.\,\hh_1\to\hh_3{:}\msg{react}.\,  \hh_2  %\hh_1
%\to\hh_4{:}\msg{react}.\,\hat{\G'}}
%where
%%\Clinefe{
%\Cline{
%\hat{\G'}=\hh_4\to\hh_2{:}\msg{pars}.\,
%\hh_1\to\hh_2{:}
%\left\{ \begin{array}{l}                                                                                                                                           \msg{rc}.\,\hh_2\to\hh_1{:}\msg{img}.\,\G' \\
%\msg{nc}.\,\hh_2\to\hh_1{:}\msg{img}.\,\G'                                                                            \end{array}\right.}  
% \\[-5.5mm] \finex
%%\finex
%\end{example}
%\begin{definition}[Safety properties \cite{BZ83,CF05}]\hfill\\

The overall behaviour of a system can be described (at least) by the traces of configurations that are reachable from a distinguished initial one. Configurations may exhibit some pathological properties, like various forms of {\em deadlock} or {\em progress violation}, channels containing messages that will never be consumed ({\em orphan messages}) or 
participants expecting messages which are different from those 
present in their input channels
({\em unspecified receptions}). 

The goal of the analysis of communicating systems is to check whether certain kinds of configurations
are not reachable, like, e.g., deadlock configurations or configurations with reception error. 
Although the desirable system properties are undecidable in general~\cite{BZ83}, sufficient conditions are known that are effectively checkable
relying, for instance, on half-duplex communication~\cite{CF05}, on the form of network topologies~\cite{DBLP:conf/concur/ClementeHS14}, or on synchronous compatibility checking~\cite{HB18}.

We formalise now a number of relevant communication properties for systems of CFSMs
that we deal with in the present paper.  

\begin{definition}[Communication properties]%\hfill\\
\label{def:safeness}
Let $S$ be a communicating system, and let $s= (\vec{q},\vec{w})$ be a configuration of $S$.
\begin{enumerate}[i)]
\item
\label{def:safeness-i}
$s$ is a {\em deadlock configuration} of $S$ if \hspace{2mm}
$\vec{w}=\vec{\varepsilon}\quad\text{and}\quad \forall \ttp\in\roles.~q_\ttp \text{ is a receiving state}$.\\
I.e. all buffers are empty, but all machines are waiting for a message.\\
We say that $S$ is {\em deadlock-free} whenever, for any $s\in \RS(S)$, $s$ is not a  deadlock configuration.

%\item
%\label{def:safeness-i}
%$s$ is a {\em deadlock configuration} if $s\, \not\!\!\lts{}$ and either
%\begin{enumerate}[a)]
%\item $\exists \ttr\in\roles$ such that $q_\ttr \lts{\ttr\tts?a} q'_\ttr$ , or
%\item
%\label{def:safeness-wnotem}
%$\vec{w}\neq\vec{\varepsilon}$ 
%\end{enumerate}
%i.e. $s$ is stuck  because all machines which are not in a final state are in a receiving state waiting for messages that cannot be read from the buffer; moreover if  $\vec{q}$ is final all buffers are empty.\\
%We say that $S$ is {\em deadlock-free} whenever, for any $s\in \RS(S)$, $s$ is not a  deadlock configuration.

\item
$s$ is an {\em  orphan-message  configuration} of $S$ if \hspace{2mm}
$\forall \ttp\in\roles. ~ q_\ttp \text{ is final} \quad\text{and}\quad  \vec{w}\neq \vec{\varepsilon}$.\\
I.e. each machine is in a final state, but there is still  at least one non-empty buffer.
We say that $S$ is {\em orphan-message free} whenever, for any $s\in \RS(S)$, $s$ is not an orphan-message configuration.

\item
\label{def:safeness-ur}
$s$ is an {\em unspecified reception configuration} of $S$  if ~$\exists \ttr \in\roles$ such that  
\begin{enumerate}[a)]
\item
%$\exists \ttr \in\roles. ~ 
$q_\ttr \text{ is a receiving state}$; and
\item
$\forall\tts\in\roles.[~(q_\ttr,\tts\ttr?\msg[a],q'_\ttr)\in\delta_\ttr  \implies
(|w_{\tts\ttr}| > 0~~\wedge~~ w_{\tts\ttr}\not\in  \msg[a]\cdot\mathbb{A}^*)~ ]$.
\end{enumerate}
I.e. there is a receiving  state $q_\ttr$ 
which is prevented from
receiving any message from any of its buffers.
(In other words, in each channel $\tts\ttr$ from which participant $\ttr$ could consume, there
is a message which cannot be received by $\ttr$ in state $q_\ttr$.)
We say that $S$ is {\em reception-error free} whenever, for any $s\in \RS(S)$, $s$ is not an unspecified reception configuration.
\item
\label{def:progress-i}
$S$ satisfies the {\em progress property} if for all $s= (\vec{q},\vec{w}) \in \RS(S)$, either there exists $s'$ such that $s\lts{} s'$
or $~\forall \ttp\in\roles. ~ q_\ttp \text{ is final}$. 
\item
\label{def:lock-freedom}
$s$ is a $\ttp$-{\em lock configuration} of $S$ if $\ttp\in\roles$ and such that
\begin{enumerate}[a)]
\item
$q_{\ttp}$ is a receiving state; and
\item 
 $\ttp$ does not appear as subject in any label of any transition sequence from $s$.
\end{enumerate}
I.e. $\ttp$ remains stuck in all possible transition sequences from $s$.
We say that $S$ is {\em lock-free} whenever, for each $\ttp\in\roles$ and each $s\in \RS(S)$, $s$ is not a $\ttp$-lock configuration.
\end{enumerate}
\end{definition}

Note that progress property (\ref{def:progress-i}) implies deadlock-freedom, whereas the inverse implication does not hold: just consider a stuck configuration with at least a non final state and a non empty buffer.
Moreover, an unspecified reception configuration is trivially a $\ttp$-lock for some 
$\ttp$. This immediately implies that lock-freedom implies
reception-error-freedom.
It is also straightforward to check that lock-freedom does imply  both  deadlock-freedom
 and progress. 
The other properties are mutually independent. 
%\brc I believe that lock-freedom also implies progress.
%Then we have that lock-freedom implies everything but orphan-message freedom
%and the converse does also not hold. Also we know from our previous paper that
%all other properties are mutually independent which then should also hold here.
%But before we must say that  lock-freedom also implies progress if you agree.
%\erc
 Communication properties above are essentially as presented in~\cite{DY12,TY15,CF05,BZ83,DY12}. 

%The above definitions of communication properties (\ref{def:safeness-i})--(\ref{def:progress-i}) are the same as the properties considered in~\cite{DY12},
%though the above formulation of progress is slightly simpler but equivalent to the one in~\cite{DY12}.
%The notions of orphan message and unspecified reception are also the same as in~\cite{TY15}.
%The same notions of deadlock and unspecified reception are given in~\cite{CF05} and inspired by~\cite{BZ83}. The deadlock notions in~\cite{BZ83} and~\cite{TY15} coincide with~\cite{CF05} and~\cite{DY12} if the local CFSMs have no final states. Otherwise, deadlock in~\cite{TY15} is weaker than deadlock above.
%A still weaker notion of deadlock configuration, and hence a stronger notion of deadlock-freedom, has been suggested in~\cite{TG18}. 
%This deadlock notion has been formally related to the above 
%communication properties in~\cite{BdLH19}.

%\brc
%A further comment: If we can save enough space, I would so much
%prefer to move the definition of projection and~\cref{lem:nohatrestrict}
%to the main part of the paper as a hint for the most important proposition in our proofs.
%\erc

%To distinguish it from the notion above, we call it \emph{strong deadlock-freedom}, as done in \cite{BdLH19}.

\begin{definition}[CFSM with interface edges]\label{def:cfsmie}%\hfill\\
A {\em CFSM with interface edges} is a tuple $M=(Q,q_0,\mathbb{A},\bm{\delta})$ 
where $Q$, $q_0$ and $\mathbb{A}$ are as in the definition of CFSM, whereas
$$\bm{\delta} \subseteq Q\times\textit{Act}_{\roles,\mathbb{A}}\times Q \times \Set{\intf,\nintf}$$
An element of $\bm{\delta}$ with the form $(\_,\_,\_,\intf)$ is called {\em interface edge}.
\end{definition}

A CFSM can be looked at as a CFSM with interface edges where all the edges are non interface ones.

We use the notation $q\lts{l}q'$ for $(q,l,q',\nintf)$ and
 $q
 \raisebox{2.7mm}
{\begin{tikzpicture}[mycfsm]
      % 
      \node[state, draw=none] (zero) [yshift=-4mm, xshift=5mm] {$~$};
      \node[state, draw=none] (one) [right of=zero, xshift=-10mm]   {$~$};
      % 
      \draw (zero) edge[-to,line width=0.5mm] node[above]{$l$} (one)
      ;
 \end{tikzpicture}
 } 
\!\! q'$ for $(q,l,q',\intf)$.

\begin{definition}[$\I(M)$]\label{def:IM}%\hfill\\
Let $M=(Q,q_0,\textit{Act},\bm{\delta})$ be a CFSM with interface edges
and let $M'=(Q',q'_0,\textit{Act},\delta')$ be a standard CFSM. We say that
$M'$ is an {\em interface for $M$ via $(f,g)$}, $\I^M_{\!\!(f,g)}(M')$, whenever
\begin{itemize}
\item[-]
$f:Q\to Q'$  is onto and such that, for all $q\in Q$, $\langin(q)=\langin(f(q))$;
\item[-]
$g:\delta'\to\Set{e\in\delta\mid e \text{ is an interface edge}}$ is such that 
$\I(M)$ as the CFSM obtained out of $\bm{\varepsilon}(M)$ using the standard procedure
to get a FSA without $\varepsilon$-transitions out of a $\varepsilon$-FSA \cite[someStandardReference]. 
Roughly:\\ 
- one first calculate the $\varepsilon$-closure for each state, which is the set of all states reachable from a given state using only $\varepsilon$-transitions;\\ 
- then, for each element of $\textit{Act}$, define new transitions for each state by considering the $\varepsilon$-closures of the states reachable via the original transition function.
\end{itemize}
\end{definition}

\begin{example}[$\I(M)$]\label{def:IM}%\hfill\\
Let $M=(Q,q_0,\textit{Act},\bm{\delta})$ be a CFSM with interface edges. We define
\begin{enumerate}[i)]
\item
$\bm{\varepsilon}(M)$ as  the $\varepsilon$-FSA $(Q,q_0,\textit{Act}\cup\Set{\varepsilon},\delta')$   where\\
\centerline{
$\delta' = \Set{(q,l,q') \mid (q,l,q',\nintf) \in \bm{\delta}}\cup \Set{(q,\varepsilon,q') \mid \exists l.(q,l,q',\intf) \in \bm{\delta}}$  }
\item
$\I(M)$ as the CFSM obtained out of $\bm{\varepsilon}(M)$ using the standard procedure
to get a FSA without $\varepsilon$-transitions out of a $\varepsilon$-FSA \cite[someStandardReference]. 
Roughly:\\ 
- one first calculate the $\varepsilon$-closure for each state, which is the set of all states reachable from a given state using only $\varepsilon$-transitions;\\ 
- then, for each element of $\textit{Act}$, define new transitions for each state by considering the $\varepsilon$-closures of the states reachable via the original transition function.
\end{enumerate}
\end{example}



\begin{definition}[Interface decorations]\label{def:IDM}
Let $M=(Q,q_0,\textit{Act},\delta)$ be a CFSM. We define the {\em interface decorations set} of $M$
as the following set of CFSMs with interface edges:
$$\IDS(M) = \Set{(Q,q_0,\textit{Act},\bm{\delta'}) \mid \proj{\bm{\delta'}}{Q\times\textit{Act}\times Q} =\delta}$$
\end{definition}


\section{Partial-fusion Composition}



Definition: Structurally identical.

\begin{definition}[$\bm{\delta}$-complementarity]
Let $M^1_\hh = (Q^1, q^1_0, \textit{Act}, \delta^1)$ and $M^2_\hh = (Q^2, q^2_0, \textit{Act}, \delta^2)$  be two CFSMs with the same name $\hh$.
We say that
\begin{enumerate}[i)]
\item
$M^1_\hh$ is {\em $\bm{\delta^2}$-complementary} with $M^2_\hh$, written $\emb{\bm{\delta^2}}{M^1_\hh}{M^2_\hh}$, whenever 
\begin{itemize}
\item[-] 
there exixts $M'_\hh = (Q', q'_0, \textit{Act}, \bm{\delta^2})\in\IDS(M^2_\hh)$;
\item[-]
$M^1_\hh$ is structurally identical to $\I(M'_\hh)$;
\item[-]
$q_1 \lts{\hh\ttp!\msg[m]} q_2 \in \delta^2$ for some $\ttp$ \quad iff \quad 
$\eps(q_1)\lts{\ttp'\hh?\msg[m]}\eps(q_2)\in \delta^1$ for some $\ttp'$;
\item[-]
$q_1 \lts{\ttp\hh?\msg[m]} q_2 \in \delta^2$ for some $\ttp$ \quad iff \quad 
$\eps(q_1)\lts{\hh\ttp'!\msg[m]}\eps(q_2)\in \delta^1$  for some $\ttp'$.
\end{itemize}
\end{enumerate}
\end{definition}

\begin{definition}[Partial Fusion]
Let $M^1_\hh = (Q^1, q^1_0, \textit{Act}, \delta^1)$ and $M^2_\hh = (Q^2, q^2_0, \textit{Act}, \delta^2)$  be two CFSMs with the same name $\hh$ such that $\emb{\bm{\delta^2}}{M^1_{\hh}}{M^2_\hh}$.
We define the {\em partial fusion of $M^1_{\hh}$ and $M^2_{\hh}$ via $\bm{\delta^2}$} as
$$\fusion_{\!\bm{\delta^2}}(M^1_{\hh},M^2_{\hh}) = (Q^2\cup\widehat{Q},q_0,\textit{Act},\widehat{\delta})$$
\begin{tabular}{lc@{\hspace{2mm}}l}
where &  $\bullet$  & $\widehat{Q} =\Set{q^{(q, l,q')} \mid (q, l,q',\intf)\in\bm{\delta^2}}$; \\[1mm]
          &  $\bullet$  & $\widehat\delta = \Set{(q, l,q') \mid (q, l,q',\nintf)\in\bm{\delta^2}}\,\cup$\\ 
           &    & ${\hspace{20pt}}\Set{(q,\ttr\HH?\msg[a],\widehat q), (\widehat q,\HH\tts!\msg[a],q') \mid  (q,\HH\tts!\msg[a],q',\intf)\in\bm{\delta^2}, (\eps(q),\ttr\hh?\msg[a],\ \eps(q'))\in\delta^1,\ \widehat q=q^{(q,\HH\tts!\msg[a],q')}}\, \cup$ \\
                &    & ${\hspace{20pt}}\Set{(q,\tts\HH?\msg[a],\widehat q), (\widehat q,\HH'\ttr!\msg[a],q') \mid  (q,\tts\HH?\msg[a],q',\intf)\in\bm{\delta^2},\ (\eps(q),\hh'\ttr!\msg[a],\eps(q'))\in\delta^1,\ \widehat q=q^{(q,\tts\HH?\msg[a],q')}}.$
 \end{tabular} 
\end{definition}



\begin{definition}[Composition by Partial Fusion]
\label{def:cpf}
Let $S_1=(M^1_\ttx)_{\ttx\in\roles_1}$ and $S_2=(M^2_\ttx)_{\ttx\in\roles_2}$ be two communicating systems such that $\roles_1\cap\roles_2=\Set{\hh}$
and $\emb{\bm{\delta}}{M^1_{\hh}}{M^2_{\hh}}$.
We define the {\em composition of $S_1$ and $S_2$ via partial fusion of $\hh$} by
$$\fusioncomp_{\!\hh}(S_1,S_2) = (\widetilde{M}_\ttx)_{\ttx\in\roles_1\cup\roles_2}$$ 
\begin{tabular}{lc@{\hspace{2mm}}l@{\hspace{4mm}}l}
where &  $\bullet$  & $\widetilde M_\ttx = M^1_\ttx$  & $\text{if}\quad \ttx\in\roles_1\setminus\Set{\hh} $; \\[1mm]
          &   $\bullet$  & $\widetilde M_\ttx = M^2_\ttx$ &  $\text{if}\quad \ttx\in\roles_2\setminus\Set{\hh} $; \\[1mm]
                    &   $\bullet$  & $\widetilde M_{\hh} =\fusion_{\!\bm{\delta}}(M^1_{\hh},M^2_{\hh})$.
 \end{tabular} 

\end{definition}















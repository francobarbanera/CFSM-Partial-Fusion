%!TEX root = JLAMP-Main.tex


\section{Discussion  of the Assumptions and Usefulness of Gateways}
\label{sec:nomix}

In this section we discuss {\em no mixed state} and {\em ?!-determinsm} assumptions and provide for each case counterexamples showing that, by omitting one of the two assumptions preservation results fail. The literature gives already a hint that both assumptions are important.
In~\cite{GMY80} it has been shown that two compatible finite state machines which are deterministic and have no mixed states are free from deadlocks and unspecified receptions. 
C{\'{e}}c{\'{e} and Finkel show in~\cite{CF05} that two compatible finite state machines
yield a half-duplex system under the restriction that both don't have mixed states and are deterministic.
Moreover, they show that for half-duplex systems of two CFSMs several properties, among them deadlock-freeness and the absence of unspecified receptions, are decidable. It is still worth to discuss the relevance
of our two restrictions, since we are dealing with a different problem, namely preservation of communication properties under composition of open systems, and we use distinguished gateways for their connection.


\subsection{The no mixed state assumption}
\label{sec:nomix-assumption}

The impact of the no mixed state assumption for interface compatibility is interesting,
since for interfaces whose machines have mixed states, their gateways do, by definition, not have mixed states.
Nevertheless the no mixed state assumption is important since a mixed state $q$ of an interface machine
gives rise to two types of receiving actions in the state $q$ of its gateway, one coming from inside and one coming from outside. %; see the example below.  


%\paragraph{Mixed states and orphan messages}
%
%Let us consider 
%a system $S_1$ with roles {\tt A} and $\HH$,  where $\HH$ is the interface,
%and let $S_2$ be a system with roles {\tt B} and $\KK$, where $\KK$ is the interface.
%Let us assume $M_\HH$ and $M_\KK$ to be the following CFSMs:
%
%{\footnotesize
%$$
%\begin{array}{c@{\hspace{2cm}}c}
%      \begin{tikzpicture}[->,>=stealth',shorten >=1pt,auto,node distance=1.5cm,semithick]
% 
%  \node[state]           (one)                        {$1$};
%   \node[draw=none,fill=none] (start) [above left = 0.3cm  of one]{$\HH$};
%  \node[state]            (two) [below left of=one] {$2$};
%  \node[state]           (three) [below right of=one] {$3$};
%
%
%   \path  (start) edge node {} (one) 
%            (one)  edge                                   node [pos=0.1,left] {\edgelabel{AH}?{a}} (two)
%                      edge                                   node  {\edgelabel{HA}!{b}} (three);
%
%       \end{tikzpicture}
%       &
%       \begin{tikzpicture}[->,>=stealth',shorten >=1pt,auto,node distance=1.5cm,semithick]
%  
%
%  \node[state]           (one)                        {$1$};
%   \node[draw=none,fill=none] (start) [above left = 0.3cm  of one]{$\KK$};
%  \node[state]            (two) [below left of=one] {$2$};
%  \node[state]           (three) [below right of=one] {$3$};
%
%
%   \path  (start) edge node {} (one) 
%            (one)  edge                                   node [pos=0.1,left] {\edgelabel{KB}!{a}} (two)
%                      edge                                   node  {\edgelabel{BK}?{b}} (three);
%
%       \end{tikzpicture}
%\end{array}
%$$
%}
%
%The two machines have dual languages and are ?!-deterministic, 
%but both have a mixed state.
%Let us now consider the following CFSMs representing
%$\gateway{M_\HH,\KK}$ and $\gateway{M_\KK,\HH}$:
%
%
%{\footnotesize
%$$
%\begin{array}{c@{\hspace{2cm}}c}
%      \begin{tikzpicture}[->,>=stealth',shorten >=1pt,auto,node distance=1.5cm,semithick]
% 
%  \node[state]           (one)                        {$1$};
%  \node[state]           (hatonetwo)          [below left of=one]              {$\widehat{1}'$};
%   \node[draw=none,fill=none] (start) [above left = 0.3cm  of one]{$\HH$};
%  \node[state]            (two) [below left of=hatonetwo] {$2$};
%  \node[state]           (hatonethree) [below right of=one] {$\widehat{1}''$};
%  \node[state]           (three) [below right of=hatonethree] {$3$};
%
%
%   \path  (start) edge node {} (one) 
%            (one)  edge                                   node [pos=0.1,left] {\edgelabel{AH}?{a}} (hatonetwo)
%                  edge                                   node  {\edgelabel{KH}?{b}} (hatonethree)
%             (hatonetwo)  edge                                   node [pos=0.1,left] {\edgelabel{HK}!{a}} (two)
%             (hatonethree)  edge                                   node {\edgelabel{HA}!{b}} (three);
%
%       \end{tikzpicture}
%       &
%       \begin{tikzpicture}[->,>=stealth',shorten >=1pt,auto,node distance=1.5cm,semithick]
%  
%
%   \node[state]           (one)                        {$1$};
%  \node[state]           (hatonetwo)          [below left of=one]              {$\widehat{1}'$};
%   \node[draw=none,fill=none] (start) [above left = 0.3cm  of one]{$\KK$};
%  \node[state]            (two) [below left of=hatonetwo] {$2$};
%  \node[state]           (hatonethree) [below right of=one] {$\widehat{1}''$};
%  \node[state]           (three) [below right of=hatonethree] {$3$};
%
%
%   \path  (start) edge node {} (one) 
%            (one)  edge                                   node [pos=0.1,left] {\edgelabel{HK}?{a}} (hatonetwo)
%                  edge                                   node  {\edgelabel{BK}?{b}} (hatonethree)
%             (hatonetwo)  edge                                   node [pos=0.1,left] {\edgelabel{KB}!{a}} (two)
%             (hatonethree)  edge                                   node {\edgelabel{KH}!{b}} (three);
%
%       \end{tikzpicture}
%\end{array}
%$$
%}
%
%Assume that in $S_1$ the machine of {\tt A} sends {\sf a} to {\tt H} and then terminates,
%and in $S_2$ the machine of {\tt B} sends {\sf b} to {\tt K} and then terminates.
%Obviously, both $S_1$ and $S_2$ are free from orphan messages.
%But if $S_1$ and $S_2$ were connected by using the above gateway machines
%then all machines can get into final states where the buffers between $\HH$ and $\KK$ are not empty;
%more precisely,  $\gateway{M_\HH,\KK}$ has reached state {\tt 2}, $w_{\HH\KK}=\mess{a}$, 
%$\gateway{M_\KK,\HH}$ has reached state {\tt 3}, and $w_{\KK\HH}=\mess{b}$.
%Thus $S={S_{1}}\connect{\HH}{\KK} {S_{2}}$ has an orphan-message configuration.


%\paragraph{Mixed states, deadlock, progress and unspecified reception}
\paragraph{Mixed states, deadlock and progress}
Let us consider 
a system $S_1$ with roles {\tt A} and $\HH$,  where $\HH$ is the interface,
and let $S_2$ be a system with roles {\tt B} and $\KK$, where $\KK$ is the interface.
Let us assume $M_\HH$ and $M_\KK$ to be the following CFSMs:

{\footnotesize
$$
\begin{array}{c@{\hspace{2cm}}c}
      \begin{tikzpicture}[->,>=stealth',shorten >=1pt,auto,node distance=1.5cm,semithick]
 
  \node[state]           (one)                        {$1$};
   \node[draw=none,fill=none] (start) [above left = 0.3cm  of one]{$\HH$};
  \node[state]            (two) [below left of=one] {$2$};
  \node[state]           (three) [below right of=one] {$3$};


   \path  (start) edge node {} (one) 
            (one)  edge                                   node [pos=0.1,left] {\edgelabel{AH}?{a}} (two)
                      edge                                   node  {\edgelabel{HA}!{b}} (three);

       \end{tikzpicture}
       &
       \begin{tikzpicture}[->,>=stealth',shorten >=1pt,auto,node distance=1.5cm,semithick]
  

  \node[state]           (one)                        {$1$};
   \node[draw=none,fill=none] (start) [above left = 0.3cm  of one]{$\KK$};
  \node[state]            (two) [below left of=one] {$2$};
  \node[state]           (three) [below right of=one] {$3$};


   \path  (start) edge node {} (one) 
            (one)  edge                                   node [pos=0.1,left] {\edgelabel{KB}!{a}} (two)
                      edge                                   node  {\edgelabel{BK}?{b}} (three);

       \end{tikzpicture}
\end{array}
$$
}

The two machines have dual languages and are ?!-deterministic, 
but both have a mixed state.
Let us now consider the following CFSMs representing
$\gateway{M_\HH,\KK}$ and $\gateway{M_\KK,\HH}$:


{\footnotesize
$$
\begin{array}{c@{\hspace{2cm}}c}
      \begin{tikzpicture}[->,>=stealth',shorten >=1pt,auto,node distance=1.5cm,semithick]
 
  \node[state]           (one)                        {$1$};
  \node[state]           (hatonetwo)          [below left of=one]              {$\widehat{1}'$};
   \node[draw=none,fill=none] (start) [above left = 0.3cm  of one]{$\HH$};
  \node[state]            (two) [below left of=hatonetwo] {$2$};
  \node[state]           (hatonethree) [below right of=one] {$\widehat{1}''$};
  \node[state]           (three) [below right of=hatonethree] {$3$};


   \path  (start) edge node {} (one) 
            (one)  edge                                   node [pos=0.1,left] {\edgelabel{AH}?{a}} (hatonetwo)
                  edge                                   node  {\edgelabel{KH}?{b}} (hatonethree)
             (hatonetwo)  edge                                   node [pos=0.1,left] {\edgelabel{HK}!{a}} (two)
             (hatonethree)  edge                                   node {\edgelabel{HA}!{b}} (three);

       \end{tikzpicture}
       &
       \begin{tikzpicture}[->,>=stealth',shorten >=1pt,auto,node distance=1.5cm,semithick]
  

   \node[state]           (one)                        {$1$};
  \node[state]           (hatonetwo)          [below left of=one]              {$\widehat{1}'$};
   \node[draw=none,fill=none] (start) [above left = 0.3cm  of one]{$\KK$};
  \node[state]            (two) [below left of=hatonetwo] {$2$};
  \node[state]           (hatonethree) [below right of=one] {$\widehat{1}''$};
  \node[state]           (three) [below right of=hatonethree] {$3$};


   \path  (start) edge node {} (one) 
            (one)  edge                                   node [pos=0.1,left] {\edgelabel{HK}?{a}} (hatonetwo)
                  edge                                   node  {\edgelabel{BK}?{b}} (hatonethree)
             (hatonetwo)  edge                                   node [pos=0.1,left] {\edgelabel{KB}!{a}} (two)
             (hatonethree)  edge                                   node {\edgelabel{KH}!{b}} (three);

       \end{tikzpicture}
\end{array}
$$
}

%Now suppose 
Suppose that in $S_1$ above the machine of {\tt A} has a single transition waiting to receive {\sf b} from $\HH$ and terminates when it has received {\sf b}.
Similarly, in $S_2$ the machine of {\tt B} waits to receive {\sf a} from $\KK$ and terminates when it has received {\sf a}.
Obviously, both $S_1$ and $S_2$ satisfy the progress property and therefore are also deadlock-free.
(Moreover, $S_1$ and $S_2$ are orphan-message free and reception-error free.)
But if $S_1$ and $S_2$ were connected by the gateways, the system $S={S_{1}}\connect{\HH}{\KK} {S_{2}}$
is immediately in a deadlock configuration where all machines wait for the reception of a message
and all buffers are empty. Moreover, this shows that $S$ does also not satisfy the progress property.

\paragraph{Mixed states and unspecified reception}

To get a counterexample in the case of unspecified receptions, we extend the two machines
$M_\HH$ and $M_\KK$ as follows:

{\footnotesize
$$
\begin{array}{c@{\hspace{2cm}}c}
      \begin{tikzpicture}[->,>=stealth',shorten >=1pt,auto,node distance=1.5cm,semithick]
 
  \node[state]           (one)                        {$1$};
   \node[draw=none,fill=none] (start) [above left = 0.3cm  of one]{$\HH$};
  \node[state]            (two) [below left of=one] {$2$};
  \node[state]           (three) [below right of=one] {$3$};
   \node[state]           (four) [below left of=two] {$4$};
    \node[state]           (five) [below right of=two] {$5$};



   \path  (start) edge node {} (one) 
            (one)  edge                                   node [pos=0.1,left] {\edgelabel{AH}?{a}} (two)
                      edge                                   node  {\edgelabel{HA}!{b}} (three)
            (two)  edge                                   node [pos=0.1,left] {\edgelabel{HA}!{c}} (four)
 			edge                                   node {\edgelabel{AH}?{d}} (five);

       \end{tikzpicture}
       &
       \begin{tikzpicture}[->,>=stealth',shorten >=1pt,auto,node distance=1.5cm,semithick]
  

  \node[state]           (one)                        {$1$};
   \node[draw=none,fill=none] (start) [above left = 0.3cm  of one]{$\KK$};
  \node[state]            (two) [below left of=one] {$2$};
  \node[state]           (three) [below right of=one] {$3$};
    \node[state]           (four) [below left of=two] {$4$};
        \node[state]           (five) [below right of=two] {$5$};



   \path  (start) edge node {} (one) 
            (one)  edge                                   node [pos=0.1,left] {\edgelabel{KB}!{a}} (two)
                      edge                                   node  {\edgelabel{BK}?{b}} (three)
	     (two)  edge                                   node [pos=0.1,left] {\edgelabel{BK}?{c}} (four)
	              edge                                   node {\edgelabel{KB}!{d}} (five);

       \end{tikzpicture}
\end{array}
$$
}

The two machines have dual languages and are ?!-deterministic, 
%satisfy conditions 1) and 3) of our compatibility notion in Definition~\ref{interface-comp}, part i),
but both have mixed states.
The gateway machines $\gateway{M_\HH,\KK}$ and $\gateway{M_\KK,\HH}$
are the following CFSMs:

{\footnotesize
$$
\begin{array}{c@{\hspace{2cm}}c}
      \begin{tikzpicture}[->,>=stealth',shorten >=1pt,auto,node distance=1.5cm,semithick]
 
  \node[state]           (one)                        {$1$};
  \node[state]           (hatonetwo)          [below left of=one]              {$\widehat{1}'$};
   \node[draw=none,fill=none] (start) [above left = 0.3cm  of one]{$\HH$};
  \node[state]            (two) [below left of=hatonetwo] {$2$};
  \node[state]           (hatonethree) [below right of=one] {$\widehat{1}''$};
  \node[state]           (three) [below right of=hatonethree] {$3$};
      \node[state]           (hattwofour) [below left of=two] {$\widehat{2}'$};
    \node[state]           (four) [below left of=hattwofour] {$4$};
        \node[state]           (hattwofive) [below right of=two] {$\widehat{2}''$};
    \node[state]           (five) [below right of=hattwofive] {$5$};



   \path  (start) edge node {} (one) 
            (one)  edge                                   node [pos=0.1,left] {\edgelabel{AH}?{a}} (hatonetwo)
                  edge                                   node  {\edgelabel{KH}?{b}} (hatonethree)
             (hatonetwo)  edge                                   node [pos=0.1,left] {\edgelabel{HK}!{a}} (two)
             (hatonethree)  edge                                   node {\edgelabel{HA}!{b}} (three)
              (two)  edge                                   node [pos=0.1,left] {\edgelabel{KH}?{c}} (hattwofour)
                  edge                                   node  {\edgelabel{AH}?{d}} (hattwofive)
            (hattwofour)  edge                                   node [pos=0.1,left] {\edgelabel{HA}!{c}} (four)
             (hattwofive)  edge                                   node {\edgelabel{HK}!{d}} (five);
       \end{tikzpicture}
\end{array}
$$
}

{\footnotesize
$$
\begin{array}{c@{\hspace{2cm}}c}
       \begin{tikzpicture}[->,>=stealth',shorten >=1pt,auto,node distance=1.5cm,semithick]
  

   \node[state]           (one)                        {$1$};
  \node[state]           (hatonetwo)          [below left of=one]              {$\widehat{1}'$};
   \node[draw=none,fill=none] (start) [above left = 0.3cm  of one]{$\KK$};
  \node[state]            (two) [below left of=hatonetwo] {$2$};
  \node[state]           (hatonethree) [below right of=one] {$\widehat{1}''$};
  \node[state]           (three) [below right of=hatonethree] {$3$};
    \node[state]           (hattwofour) [below left of=two] {$\widehat{2}'$};
    \node[state]           (four) [below left of=hattwofour] {$4$};
        \node[state]           (hattwofive) [below right of=two] {$\widehat{2}''$};
    \node[state]           (five) [below right of=hattwofive] {$5$};
 
    



   \path  (start) edge node {} (one) 
            (one)  edge                                   node [pos=0.1,left] {\edgelabel{HK}?{a}} (hatonetwo)
                  edge                                   node  {\edgelabel{BK}?{b}} (hatonethree)
             (hatonetwo)  edge                                   node [pos=0.1,left] {\edgelabel{KB}!{a}} (two)
             (hatonethree)  edge                                   node {\edgelabel{KH}!{b}} (three)
            (two)  edge                                   node [pos=0.1,left] {\edgelabel{BK}?{c}} (hattwofour)
                  edge                                   node  {\edgelabel{HK}?{d}} (hattwofive)
            (hattwofour)  edge                                   node [pos=0.1,left] {\edgelabel{KH}!{c}} (four)
             (hattwofive)  edge                                   node {\edgelabel{KB}!{d}} (five);


       \end{tikzpicture}
\end{array}
$$
}

Assume that in $S_1$ the machine of {\tt A} sends % two times {\sf a} and then terminates.
repeatedly (in a loop) message {\sf a} and then terminates,
and in $S_2$ the machine of {\tt B} sends %{\sf b} and then terminates.
repeatedly (in a loop) message {\sf b} and then terminates.
Both $S_1$ and $S_2$ are free from unspecified reception errors since none of their machines has a receiving state.
(Moreover, $S_1$ and $S_2$ are deadlock-free, orphan-message free and satisfy the progress property.)
To see that $S={S_{1}}\connect{\HH}{\KK} {S_{2}}$ has an unspecified reception configuration we
perform the following execution in $S$:

\vspace{2mm}
\noindent{\tt B} puts {\sf b} into the buffer $w_{\BB\KK}$;\\
{\tt K} consumes {\sf b} from the buffer $w_{\BB\KK}$;\\
{\tt K} puts {\sf b} into the buffer $w_{\KK\HH}$;\\
{\tt A} puts {\sf a} into the buffer $w_{\AA\HH}$;\\
{\tt H} consumes {\sf a} from the buffer $w_{\AA\HH}$;\\
{\tt H} puts {\sf a} into the buffer $w_{\HH\KK}$;\\
{\tt A} puts {\sf a} into the buffer $w_{\AA\HH}$;

\vspace{2mm}
Now $w_{\KK\HH} = {\sf b}, w_{\AA\HH} = {\sf a}$
and the machine $\gateway{M_\HH,\KK}$ is in receiving state {\tt 2}
expecting either the element {\sf c} in the buffer $w_{\KK\HH}$
or the element {\sf d} in the buffer $w_{\AA\HH}$.
Thus we have reached an unspecified reception configuration.\footnote{ 
Let us remark here that it is an open question whether 
the no mixed state assumption is really needed for preservation of orphan-message freeness.} 

\subsection{The ?!-determinism assumption}
\label{sec:det-assumption}

The notion of  ?!-determinsm requires that interfaces should have a deterministic
behaviour when focusing just on the names of the messages and on their type (input or output).
To abstract away the roles involved in sending and receiving messages to and from an interface
looks quite natural.
While in CFSM systems the presentation of a message 
depends also on the involved channel, an open system
receives or sends a message via an interface playing the role of an environment whose constitution is unknown.
Thus an open system simply knows that the message is coming from or going to another open system; the information about the specific role which sent or received it is necessarily to be abstracted away.
%For what concerns the second conditions, it intuitively states that
%the meaning of messages passed between connected systems must not {\em depend} on
%the participant that actually sent it (i.e. the gateways must be sort of {\em  deterministic with respect to the messages}).
%This does not come as an actual surprise.
%In CFSMs systems the meaning of a message 
%depends also on the role which sent it. However, if a system receives it through a gateway,
%it simply knows that the message is coming from another system; the information about the specific role which sent it is necessarily abstracted from.
%So, any message exchanged between connected systems must have an unambiguos meaning.\\
For a more precise understanding, we illustrate the above intuitive argument by means of examples.

\paragraph{Non $?!$-determinism and orphan-messages}


Let $S_1 = (M_\ttp)_{\ttp\in\Set{\HH,\AA}}$ and $S_2 = (M_\ttp)_{\ttp\in\Set{\KK,\BB,\CC}}$ be the systems
with the following components:

{\footnotesize
$$
\begin{array}{@{\hspace{0mm}}c@{\hspace{0cm}}c}
   \begin{array}{cc}
                     \begin{tikzpicture}[->,>=stealth',shorten >=1pt,auto,node distance=1.5cm,semithick]
                     \node[state]           (one)                        {$1$};
                     \node[draw=none,fill=none] (start) [above left = 0.3cm  of one]{$\AA$};
                     \node[state]            (two) [below of=one] {$2$};
                     \node[state]           (three)          [below of=two]              {$3$};
                                  \path  (start) edge node {} (one) 
                                             (one)  edge           node [left] {\edgelabel{HA}?{a}} (two)
                                              (two)  edge           node [left] {\edgelabel{AH}!{b}} (three);
                     \end{tikzpicture}
      &
                     \begin{tikzpicture}[->,>=stealth',shorten >=1pt,auto,node distance=1.5cm,semithick]
                     \node[state]           (one)                        {$1$};
                     \node[draw=none,fill=none] (start) [above left = 0.3cm  of one]{$\HH$};
                     \node[state]            (two) [below of=one] {$2$};
                     \node[state]           (three)          [below of=two]              {$3$};
                                  \path  (start) edge node {} (one) 
                                             (one)  edge           node [left] {\edgelabel{HA}!{a}} (two)
                                              (two)  edge           node [left] {\edgelabel{AH}?{b}} (three);
                     \end{tikzpicture}
   \end{array}                   
\hspace{22mm}
 \begin{array}{cc}
                     \begin{tikzpicture}[->,>=stealth',shorten >=1pt,auto,node distance=1.5cm,semithick]
                     \node[state]           (one)                        {$1$};
                     \node[draw=none,fill=none] (start) [above left = 0.3cm  of one]{$\KK$};
                     \node[state]            (two) [below of=one] {$2$};
                     \node[state]           (three)          [below of=two]              {$3$};
                     \node[state]           (four)          [below right of=one]              {$4$};
                                  \path  (start) edge node {} (one) 
                                             (one)  edge           node [left] {\edgelabel{CK}?{a}} (two)
                                              (two)  edge           node [left] {\edgelabel{KC}!{b}} (three)
                                              (one)  edge           node [pos = 0.1, right] {\edgelabel{BK}?{a}} (four);
                     \end{tikzpicture}
      &
                 \begin{array}{c} 
                    \begin{tikzpicture}[->,>=stealth',shorten >=1pt,auto,node distance=1.5cm,semithick]
                     \node[state]           (one)                        {$1$};
                     \node[draw=none,fill=none] (start) [above left = 0.3cm  of one]{$\BB$};
                     \node[state]            (two) [below of=one] {$2$};
                                  \path  (start) edge node {} (one) 
                                             (one)  edge                                   node [left] {\edgelabel{BK}!{a}} (two);
                     \end{tikzpicture}\\
                     \begin{tikzpicture}[->,>=stealth',shorten >=1pt,auto,node distance=1.5cm,semithick]
                     \node[state]           (one)                        {$1$};
                     \node[draw=none,fill=none] (start) [above left = 0.3cm  of one]{$\CC$};
                     
                                  \path  (start) edge node {} (one);
                     \end{tikzpicture}
                  \end{array} 
   \end{array}    
\end{array}     
$$
}

%S1 consists of the following H and A:
%
%H:
%0 --HA!a--> 1 --AH?b--> 2
%A:
%0 --HA?a--> 1 --AH!b--> 2
%
%S2 consists of the following K, C, D:
%
%K:
%0 --CK?a--> 1 --KC!b--> 2
%0 --DK?a--> 3
%C:
%0 (no transition)
%D:
%0 --DK!a--> 1
Obviously, both $M_\HH$ and $M_\KK$
 have no mixed state and are compatible, but %are not $?!$-deterministic. In particular, 
 $M_\KK$ is not $?$-deterministic.
 Moreover, both systems $S_1$ and $S_2$ are orphan-message free (and deadlock-free, reception-error free and progressing). In fact:\\
 In $S_1$, $\AA$ and $\HH$ exchange messages $\mess{a}$ and $\mess{b}$ and the system so terminates in a final configuration with
 all the buffers empty.
 In $S_2$, $\BB$ can send $\mess{a}$ to $\KK$ which consumes the message from the buffer and then 
 the system terminates with all buffers empty.
 %the only first possible two transitions consists in $\BB$ sending the message $\mess{b}$ to $\KK$, which receives it and the systems so terminates in a final configuration with
%all the buffers empty. 
 The left branch of $\KK$ can never be followed,
  %it starts with an input from $\CC$, which can never be fired,
since $\CC$ is unable to send any message. %, being it made of a single state.\\
However, the composed system $S_1\connect{\HH}{\KK}S_2$, described in  Figure~\ref{fig:ompgw},  is not
 orphan-message free. In fact, the following can happen:
 
\begin{figure}[ht]
%\label{fig:ompgw} moved to caption
{\footnotesize
$$
\begin{array}{@{\hspace{0mm}}c@{\hspace{0cm}}c}
   \begin{array}{cc}
                     \begin{tikzpicture}[->,>=stealth',shorten >=1pt,auto,node distance=1.5cm,semithick]
                     \node[state]           (one)                        {$1$};
                     \node[draw=none,fill=none] (start) [above left = 0.3cm  of one]{$\AA$};
                     \node[state]            (two) [below of=one] {$2$};
                     \node[state]           (three)          [below of=two]              {$3$};
                                  \path  (start) edge node {} (one) 
                                             (one)  edge           node [left] {\edgelabel{HA}?{a}} (two)
                                              (two)  edge           node [left] {\edgelabel{AH}!{b}} (three);
                     \end{tikzpicture}
      &
                     \begin{tikzpicture}[->,>=stealth',shorten >=1pt,auto,node distance=1.5cm,semithick]
                     \node[state]           (one)                        {$1$};
                     \node[state]           (onew)          [below of=one]              {$\widehat{1}$};
                     \node[draw=none,fill=none] (start) [above left = 0.3cm  of one]{$\HH$};
                     \node[state]            (two) [below of=onew] {$2$};
                     \node[state]            (twow) [below of=two] {$\widehat{2}$};
                     \node[state]           (three)          [below of=twow]              {$3$};
                                  \path  (start) edge node {} (one) 
                                             (one)  edge           node [left] {\edgelabel{KH}?{a}} (onew)
                                             (onew)  edge           node [left] {\edgelabel{HA}!{a}} (two)
                                             (two)  edge           node [left] {\edgelabel{AH}?{b}} (twow)
                                              (twow)  edge           node [left] {\edgelabel{HK}!{b}} (three);
                     \end{tikzpicture}
   \end{array}                   
\hspace{6mm}
 \begin{array}{cc}
                     \begin{tikzpicture}[->,>=stealth',shorten >=1pt,auto,node distance=1.5cm,semithick]
                     \node[state]           (one)                        {$1$};
                     \node[state]           (onew)       [below of=one]        {$\widehat{1}$};
                     \node[draw=none,fill=none] (start) [above left = 0.3cm  of one]{$\KK$};
                     \node[state]            (two) [below of=onew] {$2$};
                     \node[state]            (twow) [below of=two] {$\widehat{2}$};
                     \node[state]           (three)          [below of=twow]              {$3$};
                     \node[state]           (fourw)          [below right of=one]              {$\widehat{1'}$};
                     \node[state]           (four)          [below  of=four]              {$4$};
                                  \path  (start) edge node {} (one) 
                                              (one)  edge           node [left] {\edgelabel{CK}?{a}} (onew)
                                             (onew)  edge           node [left] {\edgelabel{KH}!{a}} (two)
                                             (two)  edge           node [left] {\edgelabel{HK}?{b}} (twow)
                                              (twow)  edge           node [left] {\edgelabel{KC}!{b}} (three)
                                              (one)  edge           node [pos = 0.1, right] {\edgelabel{BK}?{a}} (fourw)
                                              (fourw)  edge           node [left] {\edgelabel{KH}!{a}} (four);
                     \end{tikzpicture}
      &
                 \begin{array}{c} 
                    \begin{tikzpicture}[->,>=stealth',shorten >=1pt,auto,node distance=1.5cm,semithick]
                     \node[state]           (one)                        {$1$};
                     \node[draw=none,fill=none] (start) [above left = 0.3cm  of one]{$\BB$};
                     \node[state]            (two) [below of=one] {$2$};
                                  \path  (start) edge node {} (one) 
                                             (one)  edge                                   node [left] {\edgelabel{BK}!{a}} (two);
                     \end{tikzpicture}\\
                     \begin{tikzpicture}[->,>=stealth',shorten >=1pt,auto,node distance=1.5cm,semithick]
                     \node[state]           (one)                        {$1$};
                     \node[draw=none,fill=none] (start) [above left = 0.3cm  of one]{$\CC$};
                     
                                  \path  (start) edge node {} (one);
                     \end{tikzpicture}
                  \end{array} 
   \end{array}    
\end{array}     
$$
}
\caption{$S_1\connect{\HH}{\KK}S_2$ of the counterexample for orphan-message freeness preservation in absence of $?!$-determinism}\label{fig:ompgw}
\end{figure}


\vspace{2mm}
\noindent{\tt B} puts {\sf a} into the buffer $w_{\BB\KK}$ and reaches its final state;\\
{\tt K} consumes {\sf a} from the buffer $w_{\BB\KK}$;\\
{\tt K} puts {\sf a} into the buffer $w_{\KK\HH}$ and reaches a final state;\\
{\tt H} consumes {\sf a} from the buffer $w_{\KK\HH}$;\\
{\tt H} puts {\sf a} into the buffer $w_{\HH\AA}$;\\
{\tt A} consumes {\sf a} from the buffer $w_{\HH\AA}$;\\
{\tt A} puts {\sf b} into the buffer $w_{\AA\HH}$ and reaches its final state;\\
{\tt H} consumes {\sf b} from the buffer $w_{\AA\HH}$;\\
{\tt H} puts {\sf b} into the buffer $w_{\HH\KK}$ and reaches its final state.
\vspace{2mm}

Notice that $\CC$ is already in its final state (which is also the initial one).
So, by the above transition sequence,  system $S_1\connect{\HH}{\KK}S_2$ reaches
a  configuration where all the machines are in a final state, whereas not all the buffers are empty,
since  $w_{\HH\KK}=\mess{b}$. This means that an orphan-message configuration  belongs to 
the set of reachable configurations of $S_1\connect{\HH}{\KK}S_2$.
The crucial point of the counterexample is that compatibility of interface roles
based on comparing languages cannot discover incompatible behaviours
if nondeterminism is involved.


%=======================
%
%\noindent
%Let us consider the following systems $S_1$ and $S_2$:
%In $S_1$ the interface  role $\II$ receives twice a message {\sf a} from the role {\tt A} and terminates, but
%can also receive once the message {\sf a} from the role {\tt B} and terminate.
%In $S_2$ the interface role $\JJ$ sends twice a message {\sf a} to  role {\tt C} and terminates,
%but can also send  once a message {\sf a} to  role {\tt D} and terminate.
%The CFSMs corresponding to $\II$ and $\JJ$ look as follows:
%
%
%{\footnotesize
%$$
%\begin{array}{c@{\hspace{2cm}}c}
%      \begin{tikzpicture}[->,>=stealth',shorten >=1pt,auto,node distance=1.5cm,semithick]
% 
%  \node[state]           (one)                        {$1$};
%   \node[draw=none,fill=none] (start) [above left = 0.3cm  of one]{$\II$};
%  \node[state]            (two) [left of=one] {$2$};
%  \node[state]           (three) [left of=two] {$3$};
%  \node[state]           (four)  [below of=two] {$4$};
%
%
%   \path  (start) edge node {} (one) 
%            (one)  edge                                   node [above] {\edgelabel{AI}?{a}} (two)
%                      edge                                   node  {\edgelabel{BI}?{a}} (four)
%           (two) edge                                     node  [above] {\edgelabel{AI}{?}{a}} (three);
%
%       \end{tikzpicture}
%       &
%       \begin{tikzpicture}[->,>=stealth',shorten >=1pt,auto,node distance=1.5cm,semithick]
%  
%   \node[state]           (one)                        {$1$};
%   \node[draw=none,fill=none] (start) [above left = 0.3cm  of one]{$\JJ$};
%  \node[state]            (two) [left of=one] {$2$};
%  \node[state]           (three) [left of=two] {$3$};
%  \node[state]           (four)  [below of=two] {$4$};
%
%
%   \path  (start) edge node {} (one) 
%            (one)  edge                                   node [above] {\edgelabel{JC}!{a}} (two)
%                      edge                                   node  {\edgelabel{JD}!{a}} (four)
%           (two) edge                                     node  [above] {\edgelabel{JC}{!}{a}} (three);
%           
%       \end{tikzpicture}
%\end{array}
%$$
%}
%
%
%Obviously, the above CFSMs have dual languages and no mixed states.
%But they are not ?!-deterministic.
%Let us now take into account the CFSMs
%$\gateway{M_\II,\JJ}$ and $\gateway{M_\JJ,\II}$ depicted below.
%
%
%{\footnotesize
%$$
%\begin{array}{c@{\hspace{2cm}}c}
%      \begin{tikzpicture}[->,>=stealth',shorten >=1pt,auto,node distance=1.5cm,semithick]
% 
%  \node[state]           (one)                        {$1$};
%  \node[state]           (hatone)              [left of=one]          {$\widehat {1}'$};
%   \node[draw=none,fill=none] (start) [above left = 0.3cm  of one]{$\II$};
%  \node[state]            (two) [left of=hatone] {$2$};
%  \node[state]            (hattwo) [left of=two] {$\widehat 2$};
%  \node[state]           (three) [left of=hattwo] {$3$};
%  \node[state]           (hatfour)  [below of=hatone] {$\widehat {1}''$};
%  \node[state]           (four)  [below of=two] {$4$};
%
%
%   \path  (start) edge node {} (one) 
%            (one)  edge                                   node [above] {\edgelabel{AI}?{a}} (hatone)
%                      edge                                   node  {\edgelabel{BI}?{a}} (hatfour)
%            (hatone) edge                                     node  [above] {\edgelabel{IJ}{!}{a}} (two)
%           (two) edge                                     node  [above] {\edgelabel{AI}{?}{a}} (hattwo)
%           (hatfour) edge                                     node  [above] {\edgelabel{IJ}{!}{a}} (four)
%             (hattwo) edge                                     node  [above] {\edgelabel{IJ}{!}{a}} (three);
%       
%
%       \end{tikzpicture}
%\end{array}
%$$
%}
%
%{\footnotesize
%$$
%\begin{array}{c@{\hspace{2cm}}c}
%       \begin{tikzpicture}[->,>=stealth',shorten >=1pt,auto,node distance=1.5cm,semithick]
%  
%   \node[state]           (one)                        {$1$};
%  \node[state]           (hatone)              [left of=one]          {$\widehat {1}'$};
%   \node[draw=none,fill=none] (start) [above left = 0.3cm  of one]{$\JJ$};
%  \node[state]            (two) [left of=hatone] {$2$};
%  \node[state]            (hattwo) [left of=two] {$\widehat 2$};
%  \node[state]           (three) [left of=hattwo] {$3$};
%  \node[state]           (hatfour)  [below of=hatone] {$\widehat {1}''$};
%  \node[state]           (four)  [below of=two] {$4$};
%
%
%   \path  (start) edge node {} (one) 
%            (one)  edge                                   node [above] {\edgelabel{IJ}?{a}} (hatone)
%                      edge                                   node  {\edgelabel{IJ}?{a}} (hatfour)
%            (hatone) edge                                     node  [above] {\edgelabel{JC}{!}{a}} (two)
%           (two) edge                                     node  [above] {\edgelabel{IJ}{?}{a}} (hattwo)
%           (hatfour) edge                                     node  [above] {\edgelabel{JD}{!}{a}} (four)
%             (hattwo) edge                                     node  [above] {\edgelabel{JC}{!}{a}} (three);
%       
%           
%       \end{tikzpicture}
%\end{array}
%$$
%}
%
%
%
%\noindent
%
%Assume that in $S_1$ the machine of {\tt A} sends two times {\sf a} and then terminates
%and {\tt B} does nothing.
%For system $S_2$ we assume that the machine of {\tt C} receives two times {\sf a} and then terminates
%and {\tt D} receives one {\sf a} and then terminates.
%%Obviously, both $S_1$ and $S_2$ are orphan message-free.
%
%\vspace{2mm}
%\textbf{!!Rolf:} I have a problem with this example. To get a counterexample,
%we must be able to choose {\tt A, B, C, D} such that $S_1$ and $S_2$ are orphan message-free
%but the composed system is not. What I wrote above is not good, since $S_2$ is not orphan message-free
%in this case. The reason is that either {\tt C} or {\tt D} would not reach a final state.
%Also your argument below
%``We would end up with an orphan message configuration, where all the states are final''
%would not be true for my choice of {\tt A, B, C, D}. But I don't see an easy solution
%how to choose {\tt A, B, C, D}?}
%
%Moreoever, do we have counterexamples as well for the other properties?
%\vspace{2mm}
%
% Let us see what could happen  if $S_1$ and $S_2$ were connected by using the above machines as gateways:\\
%{\tt A} sends a message {\sf a} to $\gateway{M_\II,\JJ}$, which is 
%then passed to $\gateway{M_\JJ,\II}$. Notice that $\gateway{M_\JJ,\II}$ has two different transitions labelled
%by $\II\JJ?a$. Assume that the transition leading to state $\widehat{1}''$ of $\gateway{M_\JJ,\II}$ is fired, and then the message {\sf a} is
%delivered to  {\tt D} which consumes the message.
%Now {\tt A} sends a second message {\sf a} to $\gateway{M_\II,\JJ}$
%which is put in the buffer to $\gateway{M_\JJ,\II}$.
%We would end up with an orphan message configuration, where all the states are final, but a message {\sf a} lies unconsumed
%in the channel from $\gateway{M_\II,\JJ}$ to $\gateway{M_\JJ,\II}$. 
%Such a  behaviour does not come as a surprise, actually. In system $S'$, the meaning of the message {\sf a}
%depends also on whom sends it. In case $\JJ$ is an interface role, this role represents the environment.
%The environment necessarily looks at the message as coming from the system, not as coming from some specific role of the system.
%So, any message must be unambiguous.
%=============================================



\paragraph{Non $?!$-determinism, unspecified reception and progress}

Let us consider the systems $S_1 = (M_\ttp)_{\ttp\in\Set{\HH,\AA,\BB,\CC}}$ and $S_2 = (M_\ttp)_{\ttp\in\Set{\KK,\DD,\EE}}$ as described in Figure~\ref{fig:urfpp}. It is easy to check that both $M_\HH$ and $M_\KK$
 have no mixed state and are compatible, but are not $?!$-deterministic. In particular, $M_\HH$ is not $!$-deterministic
 and $M_\KK$ is not $?$-deterministic.
  Moreover, both systems $S_1$ and $S_2$ are free from unspecified receptions (as well as deadlock-free, orphan-message free and progressing). In particular:\\
 $S_1$ is free from unspecified receptions since from the initial state machine $\HH$  sends in sequence either the messages 
 $\mess{a}$, $\mess{c}$ and $\mess{a}$ to, respectively, $\AA$, $\CC$ and $\BB$, or the messages 
 $\mess{a}$, $\mess{b}$ and $\mess{a}$ to, respectively $\BB$, $\CC$ and $\AA$. In both cases the messages are
 received and the system arrives at a final configuration with all buffers empty.\\
 $S_2$ is  free from unspecified receptions since from the initial configuration the only machine that can send messages is
 $\DD$ and it sends in  sequence the messages 
 $\mess{a}$, $\mess{c}$ and $\mess{a}$ to $\KK$, which is able to receive all of them by following its left branch.
 The system hence arrives at a final configuration  with all buffers empty. Notice that $\KK$ cannot ever take its right branch since
 it starts with an input from $\EE$, which is made of by a single state and no transitions.\\
 The composed system $S_1\connect{\HH}{\KK}S_2$ is described in  Figure \ref{fig:urfppgw}.
 It is not free from unspecified receptions since the following can happen:
  
  \vspace{2mm}
\noindent$\DD$ puts $\mess{a}$ into the buffer $w_{\DD\KK}$;\\
$\KK$ consumes $\mess{a}$ from the buffer $w_{\DD\KK}$;\\
$\KK$ puts $\mess{a}$ into the buffer $w_{\KK\HH}$;\\
$\HH$ consumes $\mess{a}$ from the buffer $w_{\KK\HH}$ by going from state $1$ to state $\widehat{1'}$. i.e. by taking its right branch;\\
$\HH$ puts $\mess{a}$ into the buffer $w_{\HH\BB}$, going from state $\widehat{1'}$ to state $5$;\\
$\BB$ consumes $\mess{a}$ from the buffer $w_{\HH\BB}$;\\
$\DD$ puts $\mess{c}$ into the buffer $w_{\DD\KK}$;\\
$\KK$ consumes $\mess{c}$ from the buffer $w_{\DD\KK}$;\\
$\KK$ puts $\mess{c}$ into the buffer $w_{\KK\HH}$;\\
\vspace{2mm}

By the above transition sequence, $S_1\connect{\HH}{\KK}S_2$ reaches a configuration where 
$\HH$ is in a receiving state, namely $5$, waiting for a message $\mess{b}$, whereas from the nonempty buffer $w_{\KK\HH}$ 
only the message $\mess{c}$ could be consumed. Moreover $w_{\KK\HH}$ is the only buffer $\HH$ can consume messages from. This means that $S_1\connect{\HH}{\KK}S_2$ can reach an unspecified reception configuration.\\

\begin{figure}
%\label{fig:urfpp}
{\footnotesize
$$
\begin{array}{@{\hspace{-6mm}}c@{\hspace{0cm}}c}
      
 
   \begin{array}{cc}
                     \begin{tikzpicture}[->,>=stealth',shorten >=1pt,auto,node distance=1.5cm,semithick]
                     \node[state]           (one)                        {$1$};
                     \node[draw=none,fill=none] (start) [above left = 0.3cm  of one]{$\mathtt{A}$};
                     \node[state]            (two) [below of=one] {$2$};
                                  \path  (start) edge node {} (one) 
                                             (one)  edge                                   node [left] {\edgelabel{HA}?{a}} (two);
                     \end{tikzpicture}
      &
                    \begin{tikzpicture}[->,>=stealth',shorten >=1pt,auto,node distance=1.5cm,semithick]
                     \node[state]           (one)                        {$1$};
                     \node[draw=none,fill=none] (start) [above left = 0.3cm  of one]{$\mathtt{B}$};
                     \node[state]            (two) [below of=one] {$2$};
                                  \path  (start) edge node {} (one) 
                                             (one)  edge                                   node [left] {\edgelabel{HB}?{a}} (two);
                     \end{tikzpicture}\\[8mm]
  \multicolumn{2}{c}{
                   \begin{tikzpicture}[->,>=stealth',shorten >=1pt,auto,node distance=1.5cm,semithick]
                   \node[state]           (one)                        {$1$};
                   \node[draw=none,fill=none] (start) [above left = 0.3cm  of one]{$\mathtt{C}$};
                   \node[state]            (two) [below left of=one] {$2$};
                   \node[state]           (three) [below right of=one] {$3$};

                           \path  (start) edge node {} (one) 
                                    (one)  edge                                   node [pos=0.1,left] {\edgelabel{HC}?{c}} (two)
                                              edge                                   node  {\edgelabel{HC}?{b}} (three);
                   \end{tikzpicture}
                              }
   \end{array}                 
                              
       &
            
       \begin{tikzpicture}[->,>=stealth',shorten >=1pt,auto,node distance=1.5cm,semithick]
  

       \node[state]           (one)                        {$1$};
       \node[draw=none,fill=none] (start) [above left = 0.3cm  of one]{$\HH$};
       \node[state]            (two) [below left of=one] {$2$};
       \node[state]           (five) [below right of=one] {$5$};
       \node[state]           (three) [below  of=two] {$3$};
       \node[state]            (four) [below right of=three] {$4$};
       \node[state]           (six) [below  of=five] {$6$};
    %   \node[state]            (seven) [below  of=six] {$7$};

       



   \path  (start) edge node {} (one) 
            (one)  edge                                   node [pos=0.1,left] {\edgelabel{HA}!{a}} (two)
                      edge                                   node  {\edgelabel{HB}!{a}} (five)
             (two)  edge                                   node [left] {\edgelabel{HC}!{c}} (three)        
             (three)  edge                                  node [pos=0.7,left] {\edgelabel{HB}!{a}} (four)
              (five)  edge                                   node [left] {\edgelabel{HC}!{b}} (six)        
             (six)  edge                                  node [pos=0.7, right] {\edgelabel{HA}!{a}} (four);

       \end{tikzpicture}
    
\end{array}
\hspace{12mm}
\begin{array}{c@{}c}
            \begin{array}{c}
       \begin{tikzpicture}[->,>=stealth',shorten >=1pt,auto,node distance=1.5cm,semithick]
  

       \node[state]           (one)                        {$1$};
       \node[draw=none,fill=none] (start) [above left = 0.3cm  of one]{$\KK$};
       \node[state]            (two) [below left of=one] {$2$};
       \node[state]           (five) [below right of=one] {$5$};
       \node[state]           (three) [below  of=two] {$3$};
       \node[state]            (four) [below right of=three] {$4$};
       \node[state]           (six) [below  of=five] {$6$};
  %     \node[state]            (seven) [below  of=six] {$7$};

       



   \path  (start) edge node {} (one) 
            (one)  edge                                   node [pos=0.1,left] {\edgelabel{DK}?{a}} (two)
                      edge                                   node  {\edgelabel{EK}?{a}} (five)
             (two)  edge                                   node [left] {\edgelabel{DK}?{c}} (three)        
             (three)  edge                                  node [pos=0.7, left] {\edgelabel{DK}?{a}} (four)
              (five)  edge                                   node [left] {\edgelabel{EK}?{b}} (six)        
             (six)  edge                                  node [pos=0.7, right] {\edgelabel{EK}?{a}} (four);

       \end{tikzpicture}\\
       \begin{tikzpicture}[->,>=stealth',shorten >=1pt,auto,node distance=1.5cm,semithick]
                     \node[state]           (one)                        {$1$};
                     \node[draw=none,fill=none] (start) [above left = 0.3cm  of one]{$\mathtt{E}$};
                              \path  (start) edge node {} (one) ;
                     \end{tikzpicture}
       \end{array}
 &
   
                     \begin{tikzpicture}[->,>=stealth',shorten >=1pt,auto,node distance=1.5cm,semithick]
                     \node[state]           (one)                        {$1$};
                     \node[draw=none,fill=none] (start) [above left = 0.3cm  of one]{$\mathtt{D}$};
                     \node[state]            (two) [below of=one] {$2$};
                      \node[state]            (three) [below of=two] {$3$};
                       \node[state]            (four) [below of=three] {$4$};
                                  \path  (start) edge node {} (one) 
                                             (one)  edge                                   node [left] {\edgelabel{DK}!{a}} (two)
                                            (two)  edge                                   node [left] {\edgelabel{DK}!{c}} (three)        
                                            (three)  edge                                  node [left] {\edgelabel{DK}!{a}} (four) ;
                     \end{tikzpicture}
\end{array}
$$
}
\caption{$S_1$ and $S_2$ of the counterexample for reception-error freeness and progress preservation in absence of $?!$-determinism}\label{fig:urfpp}
\end{figure}


\begin{figure}
%\label{fig:urfppgw}
{\footnotesize
$$
\begin{array}{@{\hspace{-6mm}}c@{\hspace{0cm}}c}
      
 
   \begin{array}{cc}
                     \begin{tikzpicture}[->,>=stealth',shorten >=1pt,auto,node distance=1.5cm,semithick]
                     \node[state]           (one)                        {$1$};
                     \node[draw=none,fill=none] (start) [above left = 0.3cm  of one]{$\mathtt{A}$};
                     \node[state]            (two) [below of=one] {$2$};
                                  \path  (start) edge node {} (one) 
                                             (one)  edge                                   node [left] {\edgelabel{HA}?{a}} (two);
                     \end{tikzpicture}
      &
                    \begin{tikzpicture}[->,>=stealth',shorten >=1pt,auto,node distance=1.5cm,semithick]
                     \node[state]           (one)                        {$1$};
                     \node[draw=none,fill=none] (start) [above left = 0.3cm  of one]{$\mathtt{B}$};
                     \node[state]            (two) [below of=one] {$2$};
                                  \path  (start) edge node {} (one) 
                                             (one)  edge                                   node [left] {\edgelabel{HB}?{a}} (two);
                     \end{tikzpicture}\\[8mm]
  \multicolumn{2}{c}{
                   \begin{tikzpicture}[->,>=stealth',shorten >=1pt,auto,node distance=1.5cm,semithick]
                   \node[state]           (one)                        {$1$};
                   \node[draw=none,fill=none] (start) [above left = 0.3cm  of one]{$\mathtt{C}$};
                   \node[state]            (two) [below left of=one] {$2$};
                   \node[state]           (three) [below right of=one] {$3$};

                           \path  (start) edge node {} (one) 
                                    (one)  edge                                   node [pos=0.1,left] {\edgelabel{HC}?{c}} (two)
                                              edge                                   node  {\edgelabel{HC}?{b}} (three);
                   \end{tikzpicture}
                              }
   \end{array}                 
                              
       &
            
       \begin{tikzpicture}[->,>=stealth',shorten >=1pt,auto,node distance=1.5cm,semithick]
  

       \node[state]           (one)                        {$1$};
       \node[state]           (onew)                [below left of=one]         {$\widehat{1}$};
       \node[state]           (oneww)                [below right of=one]         {$\widehat{1'}$};
       \node[draw=none,fill=none] (start) [above left = 0.3cm  of one]{$\HH$};
       \node[state]            (two) [below  of=onew] {$2$};
       \node[state]            (twow) [below  of=two] {$\widehat{2}$};
       \node[state]           (five) [below of=oneww] {$5$};
       \node[state]            (fivew) [below  of=five] {$\widehat{5}$};
       \node[state]           (three) [below  of=twow] {$3$};
        \node[state]            (threew) [below  of=three] {$\widehat{3}$};
       \node[state]            (four) [below right  of=threew] {$4$};
       \node[state]           (six) [below  of=fivew] {$6$};
       \node[state]            (sixw) [below  of=six] {$\widehat{6}$};
  %     \node[state]            (seven) [below  of=sixw] {$7$};

       
   \path  (start) edge node {} (one) 
            (one)  edge                                   node [pos=0.1,left] {\edgelabel{KH}?{a}} (onew)
                      edge                                   node  {\edgelabel{KH}?{a}} (oneww)
              (onew)  edge                                   node [left] {\edgelabel{HA}!{a}} (two)      
             (oneww)  edge                                   node  [left]   {\edgelabel{HB}!{a}} (five)  
             (five)  edge                                   node [left] {\edgelabel{KH}?{b}} (fivew) 
              (two)  edge                                   node [left] {\edgelabel{KH}?{c}} (twow) 
             (twow)  edge                                   node [left] {\edgelabel{HC}!{c}} (three)    
              (three)  edge                                   node  [left]  {\edgelabel{KH}?{a}} (threew)    
             (threew)  edge                                  node [pos=0.7,left] {\edgelabel{HB}!{a}} (four)
              (fivew)  edge                                   node [left] {\edgelabel{HC}!{b}} (six)    
              (six)  edge                                   node  [left]  {\edgelabel{KH}?{a}} (sixw)     
             (sixw)  edge                                  node [pos=0.7,right] {\edgelabel{HA}!{a}} (four);

       \end{tikzpicture}
    
\end{array}
\hspace{6mm}
\begin{array}{c@{}c}
            \begin{array}{c}
       \begin{tikzpicture}[->,>=stealth',shorten >=1pt,auto,node distance=1.5cm,semithick]
  

      \node[state]           (one)                        {$1$};
       \node[state]           (onew)                [below left of=one]         {$\widehat{1}$};
       \node[state]           (oneww)                [below right of=one]         {$\widehat{1'}$};
       \node[draw=none,fill=none] (start) [above left = 0.3cm  of one]{$\KK$};
       \node[state]            (two) [below  of=onew] {$2$};
       \node[state]            (twow) [below  of=two] {$\widehat{2}$};
       \node[state]           (five) [below of=oneww] {$5$};
       \node[state]            (fivew) [below  of=five] {$\widehat{5}$};
       \node[state]           (three) [below  of=twow] {$3$};
        \node[state]            (threew) [below  of=three] {$\widehat{3}$};
       \node[state]            (four) [below  right of=threew] {$4$};
       \node[state]           (six) [below  of=fivew] {$6$};
       \node[state]            (sixw) [below  of=six] {$\widehat{6}$};
   %    \node[state]            (seven) [below  of=sixw] {$7$};

       
   \path  (start) edge node {} (one) 
            (one)  edge                                   node [pos=0.1,left] {\edgelabel{DK}?{a}} (onew)
                      edge                                   node  {\edgelabel{EK}?{a}} (oneww)
              (onew)  edge                                   node [left] {\edgelabel{KH}!{a}} (two)      
             (oneww)  edge                                   node  [left]   {\edgelabel{KH}!{a}} (five)  
             (five)  edge                                   node [left] {\edgelabel{EK}?{b}} (fivew) 
              (two)  edge                                   node [left] {\edgelabel{DK}?{c}} (twow) 
             (twow)  edge                                   node [left] {\edgelabel{KH}!{c}} (three)    
              (three)  edge                                   node  [left]  {\edgelabel{DK}?{a}} (threew)    
             (threew)  edge                                  node [pos=0.7, left] {\edgelabel{KH}!{a}} (four)
              (fivew)  edge                                   node [left] {\edgelabel{KH}!{b}} (six)    
              (six)  edge                                   node  [left]  {\edgelabel{EK}?{a}} (sixw)     
             (sixw)  edge                                  node [pos=0.7, right] {\edgelabel{KH}!{a}} (four);
       \end{tikzpicture}
       \end{array}
 &
     \begin{array}{c}
                     \begin{tikzpicture}[->,>=stealth',shorten >=1pt,auto,node distance=1.5cm,semithick]
                     \node[state]           (one)                        {$1$};
                     \node[draw=none,fill=none] (start) [above left = 0.3cm  of one]{$\mathtt{D}$};
                     \node[state]            (two) [below of=one] {$2$};
                      \node[state]            (three) [below of=two] {$3$};
                       \node[state]            (four) [below of=three] {$4$};
                                  \path  (start) edge node {} (one) 
                                             (one)  edge                                   node [left] {\edgelabel{DK}!{a}} (two)
                                            (two)  edge                                   node [left] {\edgelabel{DK}!{c}} (three)        
                                            (three)  edge                                  node [left] {\edgelabel{DK}!{a}} (four) ;
                     \end{tikzpicture}\\
                    \begin{tikzpicture}[->,>=stealth',shorten >=1pt,auto,node distance=1.5cm,semithick]
                     \node[state]           (one)                        {$1$};
                     \node[draw=none,fill=none] (start) [above left = 0.3cm  of one]{$\mathtt{E}$};
                              \path  (start) edge node {} (one) ;
                     \end{tikzpicture}
      \end{array}
\end{array}
$$
}
\caption{$S_1\connect{\HH}{\KK}S_2$ of the counterexample for reception-error freeness and progress preservation in absence of $?!$-determinsm}\label{fig:urfppgw}
\end{figure}

Systems $S_1$ and $S_2$ are actually a counterexample also to the progress property. In fact, by the previous short descriptions
of the behaviours of $S_1$ and $S_2$, it can be checked that both of them enjoy the progress property.
Moreover, the  transition sequence described above, leads to a stuck configuration which is not made by final states only.

As before, the counterexample relies on the fact that due to nondeterminism
the compatibility of $M_\HH$ and $M_\KK$ (based on comparing languages)
cannot avoid incompatible behaviours.





 


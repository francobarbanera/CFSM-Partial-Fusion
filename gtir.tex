%!TEX root = JLAMP-Main.tex

\section{Global Types with Interface Roles}
\label{sec:gtir}

The previous sections were related to a semantic view of concurrent systems where the
semantic objects are CFSMs.
In this section we show how our results
can be directly applied to syntactic formalisms for the specification of interacting components
when the formalism allows for a semantic interpretation by CFSMs.

%\subsection{Global types}\label{subsec:globaltypes}

For systems of CFSMs, most of the relevant communications properties are, in general, undecidable
\cite{CF05} or computationally hard. 
Partly with the aim of overcoming such shortcomings, 
a number of formalisms have been recently proposed in the literature 
enabling
\begin{enumerate}[(1)]
\item
 to describe in a structured way the overall behavior of systems of CFSMs;
\item
 to steer the implementation of the system components, guaranteeing their compliance with the overall behaviour together with some relevant communication properties.
 \end{enumerate}

Among them are
the {\bf generalised global types} of~\cite{DY12}, the {\bf global graphs} of~\cite{TY15}
and the {\bf global choreographies} of~\cite{TG18}.
All of them provide syntactic means to describe the overall behaviour of a system
of interacting components (also called participants or roles) 
and they allow for a semantic interpretation in terms of CFSM systems. 

To consider an example for a global type (or graph) let us come back to the system,
called $S$ in Section~\ref{sec:opensys},
which tries to transmit texts to a social network. The example is inspired by
a similar one in~\cite{deAlfaro2005}.
The overall behaviour of the system is as follows:
Once a text is received from the outside, the system tries to transmit it at most $n$ times, where also the number $n$ of possible trials is provided from the outside when the system is initialised.
A successful transmission to the network is acknowledged by an {\sf ack} message; a {\sf nack} message
represents instead an unsuccessful transmission.
An {\sf ok} message is sent back in case of a successful transmission; a {\sf fail} message
in case of $n$ unsuccessful trials.
Before any transmission trial, a semantically-invariant transformation is applied to the message
in order to take into account requirements of the social network
concerning propriety of language.
If the message is not accepted by the
network, our system automatically transforms it maintaining its sense, and sending it again and again up to $n$ times, invariantly transforming it each time. 
A counter is used to keep track of the number of  trials and it is reset to $n$ each time
a message is successfully transmitted. It is instead automatically reset to $n$ each time $0$ is reached, before 
issuing a failure message and restarting the protocol with some new message.
The roles participating in the system are the following ones:\\
{\tt M}: the manager of the system;\\
{\tt T}: the process implementing the semantically-invariant message transformation;\\
{\tt C}: the trials counter;\\
{\tt I}, {\tt J} and {\tt H}: the roles representing those
parts of the environment which, respectively: initialises the system, sends the text message
and receives back the {\sf ok} or {\sf fail} message, receives the messages transmitted 
by the systems and acknowledges its propriety, if so.

The global type $G$ describing the behaviour of system $S$ is shown in Figure~\ref{fig:examplegg}.
In the figure a label $\tts\rightarrow \ttr : a$ represents an interaction where $\tts$ sends a message
$a$ to r. A vertex with label {\small $\bigcirc$} represents the starting point of the interaction
and {\huge $\diamond$}\hspace{-8.5pt}{$^+$} marks vertexes corresponding
to branch or merge points, or to entry points of loops.
%In our formalism, we can look at $G$ as a global type with interface roles by identifying the roles $\II,\JJ$ and $\HH$ as {\em interface roles}.
%We do that by writing, according to the syntax of Def.~\ref{def:pre-gtir}(i):
%                                                  $$\GTIR{G}{\Set{\II,\JJ,\HH}}$$
%Given a global type with interface roles, it is reasonable to expect all its roles to be implemented but the interface ones,
%since they are actually used to describe the behaviour of the ``environment'' of the system.
%The projections of global types with interface roles onto their interface roles yield CFSMs which are used instead to check whether
%two systems can be connected in a safe way. In particular, to check whether interface roles are ``compatible''.
According to the projection algorithm for generalised global types (see \S 3.1 and Def. 3.4 in~\cite{DY12}), the projection $\projecton{G}{\JJ}$ on role $\JJ$ of the global type $G$ in Fig.~\ref{fig:examplegg}  is the  CFSM $M_\JJ$ shown in Figure~\ref{eq:J}
which describes the behaviour of that part of the environment of system $S$ which sends a text and waits for a positive or negative answer.

\begin{figure}[ht]
\hrule

\vspace{4mm}
\begin{center}
      \begin{tikzpicture}
      
      \mkint{}{x9}[][H][ack][M];
      
      \mkint{below=0.5 of x9}{x11}[][M][reset][C];
       
      \mkint{below=0.5 of x11}{x13}[][M][ok][J];
       
       \mkint{right=of x11}{x15}[][C][zero][M];
       
       \mkint{right=of x15}{x16}[][C][notzero][M];
       
       \mkbranch{x15,x16}[x15x16][][-0.7][@][15pt];
       
        \mkint{above =0.18 of x15x16}{x10}[][H][nack][M];
        
        \mkint{below = 0.5 of x15}{x17}[][M][fail][J];
        
        \mkseq{x9}{x11};
        \mkseq{x11}{x13};
        \mkseq{x10}{x15x16};
        \mkseq{x15}{x17};
        
         \mkbranch{x9,x10}[x9x10][][0.8][@][15pt];
         
        \mkint{above = 0.5 of x9x10}{x6}[][M][text][H];
        
        \mkint{above = 0.5 of x6}{x7}[][T][text][M];
        
        
        
        \mkint{above = 0.5 of x7}{x5}[][M][text][T];
        
        \mkseq{x5}{x7};
        
         \mkseq{x7}{x6};
        
        \mkseq{x6}{x9x10};
        
        \mklooptwo[0.5][-91]{x5}{x16}{}{};
        
        
        \mkint{above = 1.3 of x5}{x2}[][J][text][M];
        
        \path[line] (x2) -- (entryx5);
        
        
        \mkmerge{x13,x17}[x13x17][][0.8][@][15pt];
        
        \mklooptwobelow[0.5][-23]{x2}{x13x17}{}{};
        
        \mkint{above = 1.3 of x2}{x0}[][I][trialsNum][C];
        
        \mkseq{x0}{entryx2};
        
        \node[source,above = 0.5 of x0] (src) {};
        
        \mkseq{src}{x0};
        
      \end{tikzpicture}
      \end{center}
\vspace{4mm}
\hrule
\caption{The global type $G$ describing system $S$}
\label{fig:examplegg}
\end{figure}

\vspace{2mm}
Systems of CFSMs obtained as end-point projections of (well-formed)  generalised global types
do enjoy all communication properties of Definitiion~\ref{def:safeness}. They are free from deadlocks, orphan messages and unspecified receptions (Theorem 3.1 in~\cite{DY12}) and they satisfy the progress property
(Theorem 3.3 in~\cite{DY12}). Moreover, systems of CFSMs obtained
as end-point projections of global choreographies (satisfying some well-formedness conditions) 
are strongly deadlock free; see Theorem 1 in~\cite{TG18}.\\

The above mentioned formalisms represent a line of evolution of the general notion of {\em global type} widely investigated in
the literature  \cite{CHY07,CDP12,CDYP16}.
The centralised viewpoint offered by the global type approaches makes them naturally suitable for describing
{\em closed} systems. This prevents a system described/developed by means of global types
to be looked at as a module that can be connected to other systems.
In the present section, on the ground of our results, we address  the problem of
generalising the notion of {\em global type} in order to encompass  the description of {\em open systems}
and, in particular, open systems of CFSMs; so paving the way towards a fruitful interaction between the investigations on open systems carried out in automata theory and those on global types. \\

In our approach, an ``open global type'' -- that we dub ``global type with interface roles'' (GTIR) -- is a
syntactic expression denoting
a number of connected open systems of CFSMs.
According to one's needs, any role can be looked at
as representing (part of) the expected communication behaviour of the environment, or equivalently
as components whose implementation is delegated to an external system\footnote{Differently from what was done in~\cite{BdLH18}, here we do not partition the set of roles into interface and non-interface roles from the beginning. In fact,
 any role can be looked at as an interface according to the current needs of the developer of a system.
(We thank Ivan Lanese for such a suggestion.)}\\


We have no necessity to stick to any particular global type formalism as a basis for our GTIRs,
as long as the local end-point behaviours of a global type $G$ can be interpreted as CFSMs.
So we introduce a {\em parametric} syntax which, given a global type formalism \gt,
%(equipped with a semantic projection function returning a CFSM for each role of a global type $G$),
extends its syntax by essentially
enabling to identify some roles as {\em interface roles} and to 
define a  composition of open global types, semantically interpreted by systems of CFSMs.
We call \gtir\ (\gt-with-\textsc{i}nterface-\textsc{r}oles) the so obtained formalism which, 
by the main results of the present paper in Section~\ref{sect:safetypreservation}, is hence suitable for the composition of open systems which ensures preservation of communication properties.
%In particular
Our approach is  usable for any global type formalism
\gt which satisfies the following assumptions: For each global type $G$ in \gt,

\begin{enumerate}
\item
there is associated a finite set of roles $\roles(G) \subset \roles_\mathfrak{U}$ and a finite set of actions
 $\mathbb{A}(G) \subset \mathbb{A}_\mathfrak{U}$,
 \item
 there is a {\em projection} function, denoted by $\projecton{\_}{\_}$,
 such that for any $\ttp\in\roles(G)$, $\projecton{G}{\ttp}$ is a CFSM
 over $\roles(G)$ and $\mathbb{A}(G)$.
\end{enumerate}

%In the following section we formally define
%Global Types  with Interface Roles (GTIRs)
%by providing their syntax and semantics.
%
%\subsection{GTIRs: syntax and semantics}
%
%The syntax of GTIRs is  defined in terms of  pre-GTIRs.
%A pre-GTIR is either just
%a global type 
%or it is a syntactic expression composed from two pre-GTIRs by connecting
%roles which we consider as interfaces. The non-connected roles can be possibly used for further subsequent connections.
%A GTIR s hence a pre-GTIR where the connected roles are compatible.

In the following we define formally
Global Types  with Interface Roles (GTIRs)
by providing their syntax and semantics.

\begin{definition}[GTIR syntax]
\label{def:pre-gtir}
The set \GTIRset\ of {\em GTIR-expressions} $\GTIR{{\GG}}{\mathbf{I}}$ is defined by simultaneous induction together with
the set of {\em roles} $\roles(\GTIR{\GG}{\mathbf{I}})$
and projections $\projecton{\GTIR{{\GG}}{\mathbf{I}}}{{\tt p}}$ for each ${\tt p} \in \roles(\GTIR{\GG}{\mathbf{I}})$:

\begin{enumerate}[a)]

\item  if $G$ is a global type of \gt then\\
	 $\GTIR{G}{\mathbf{I}}\in \GTIRset$ and
$\roles(\GTIR{\GG}{\mathbf{I}}) = \roles(G)$ and 
$\projecton{\GTIR{{\GG}}{\mathbf{I}}}{{\tt p}} =
                  \projecton{G}{{\tt p}}$ 
                  for each ${\tt p}\in \roles({G})$;
	 
\item 	
	if
	\begin{enumerate}[1)]
	\item  $\GTIR{{\GG}_1}{\mathbf{H}}, \GTIR{{\GG}_2}{\mathbf{K}}\in\GTIRset$ such that
	$\roles(\GTIR{{\GG}_1}{\mathbf{H}}) \cap \roles(\GTIR{{\GG}_2}{\mathbf{K}})
	=\emptyset,$ and
	\item
	$\HH\in\roles(\GTIR{{\GG}_1}{\mathbf{H}}), \; \KK\in\roles(\GTIR{{\GG}_2}{\mathbf{K}})$,
	such that
	 $\HH$ and $\KK$ are interface compatible, i.e.\ $\projecton{\GTIR{{\GG_1}}{\mathbf{I}}}{\HH}\interfacecomp \projecton{\GTIR{{\GG_2}}{\mathbf{I}}}{\KK}$,
	\end{enumerate}
	then\\
	$\GTIR{\GTIR{{\GG}_1}{\mathbf{H}}\connect{\HH}{\KK}\GTIR{{\GG}_2}{\mathbf{K}}}{\mathbf{I}} \in \GTIRset$ 
	and
	$\roles(\GTIR{\GTIR{{\GG}_1}{\mathbf{H}}\connect{\HH}{\KK}\GTIR{{\GG}_2}{\mathbf{K}}}{\mathbf{I}}) = 
	\roles({\GTIR{{\GG}_1}{\mathbf{H}}}) \cup  \roles(\GTIR{{\GG}_2}{\mathbf{K}})$
and\\
 $\projecton{\GTIR{\GTIR{{\GG}_1}{\mathbf{H}}\connect{\HH}{\KK}\GTIR{{\GG}_2}{\mathbf{K}}}{\mathbf{I}}}{{\tt p}} = \projecton{\GTIR{{\GG}_1}{\mathbf{I}}}{{\tt p}}$
 for all ${\tt p}\in \roles(\GTIR{{\GG}_1}{\mathbf{I}})\setminus\{\HH\}$,\\
  $\projecton{\GTIR{\GTIR{{\GG}_1}{\mathbf{H}}\connect{\HH}{\KK}\GTIR{{\GG}_2}{\mathbf{K}}}{\mathbf{I}}}{{\tt p}} = \projecton{\GTIR{{\GG}_2}{\mathbf{I}}}{{\tt p}}$
 for all ${\tt p}\in \roles(\GTIR{{\GG}_2}{\mathbf{I}})\setminus\{\KK\}$,\\
   $\projecton{\GTIR{\GTIR{{\GG}_1}{\mathbf{H}}\connect{\HH}{\KK}\GTIR{{\GG}_2}{\mathbf{K}}}{\mathbf{I}}}{{\tt H}} = \gateway{\projecton{\GTIR{{\GG_1}}{\mathbf{I}}}{\HH},\KK}$,\\
   $\projecton{\GTIR{\GTIR{{\GG}_1}{\mathbf{H}}\connect{\HH}{\KK}\GTIR{{\GG}_2}{\mathbf{K}}}{\mathbf{I}}}{{\tt K}} = \gateway{\projecton{\GTIR{{\GG_2}}{\mathbf{I}}}{\KK},\HH}$.
\end{enumerate}

\end{definition}

\begin{definition}[GTIR semantics]
   The semantics of a GTIR $\GTIR{{\GG}}{\mathbf{I}}$  is the communicating system
  $$\Sem{\GTIR{{\GG}}{\mathbf{I}}} =
 (\projecton{\GTIR{{\GG}}{\mathbf{I}}}{{\tt p}})_{{\tt p}\in \roles(\GTIR{\GG}{\mathbf{I}})}.$$
\end{definition}

It is immediate to check that the operation of  ``connecting'' GTIRs is semantically commutative and associative if roles are not used twice for connections,
i.e. the following holds:\\[2mm]
({\em comm}) \hspace{4pt}
$ \Sem{\GTIR{\GTIR{{\GG}_1}{\mathbf{H}}\connect{\HH}{\KK}\GTIR{{\GG}_2}{\mathbf{K}}}{\mathbf{I}}}
 = \Sem{\GTIR{\GTIR{{\GG}_2}{\mathbf{K}}\connect{\KK}{\HH}\GTIR{{\GG}_1}{\mathbf{H}}}{\mathbf{I}}} $ \\
({\em ass})\hspace{20pt}
 $\Sem{
  \GTIR{\GTIR{\GTIR{{\GG}_1}{\mathbf{H}}\connect{\HH}{\KK}\GTIR{{\GG}_2}{\mathbf{K}}}{\mathbf{I}}
           \connect{\II}{\JJ}
            \GTIR{{\GG}_3}{\mathbf{J}} }
           {\mathbf{I'}}
 }
 =
 \Sem{\GTIR{
                    \GTIR{{\GG}_1}{\mathbf{H}}\connect{\HH}{\KK} 
                                     \GTIR{
                                               \GTIR{{\GG}_2}{\mathbf{K}}  \connect{\II}{\JJ} \GTIR{{\GG}_3}{\mathbf{J}}
                                               }{\mathbf{J'}}
                               }{\mathbf{I'}}
 }  
 $\\
 \hspace*{15mm} if $\HH, \KK, \II, \JJ$ are pairwise different.\\

By the definition of the composition of communicating systems via gateway CFSMs we can easily prove, by structural induction on the form of GTIRs, that their semantics is compositional:

\begin{theorem} For any composed GTIR-expression $\GTIR{\GTIR{{\GG}_1}{\mathbf{H}}\connect{\HH}{\KK}\GTIR{{\GG}_2}{\mathbf{K}}}{\mathbf{I}}$:
$$\Sem{\GTIR{\GTIR{{\GG}_1}{\mathbf{H}}\connect{\HH}{\KK}\GTIR{{\GG}_2}{\mathbf{K}}}{\mathbf{I}}}
= \Sem{   \GTIR{{\GG}_1}{\mathbf{H}}  } \connect{\HH}{\KK} \Sem{\GTIR{{\GG}_2}{\mathbf{K}}}.$$
\end{theorem}

As an immediate consequence we can apply our preservation results of Section~\ref{sect:safetypreservation}
to justify that the validity of communication properties is propagated from smaller GTIRs to larger ones.

\begin{corollary}
\label{cor:safetypres}
Let $P$ be one of the communication properties of Section \ref{sect:cfsm}. Let $\GTIR{\GG}{\mathbf{I}} =\GTIR{\GTIR{{\GG}_1}{\mathbf{H}}\connect{\HH}{\KK}\GTIR{{\GG}_2}{\mathbf{K}}}{\mathbf{I}}$
be the GTIR composed from GTIRs $\GTIR{\GG_1}{\mathbf{H}}$
and $\GTIR{\GG_2}{\mathbf{K}}$ via compatible interface roles $\HH$ and $\KK$.
If both $\Sem{\GTIR{\GG_1}{\mathbf{H}}}$ and $\Sem{\GTIR{\GG_2}{\mathbf{K}}}$ enjoy  the property $P$,
so does $\Sem{\GTIR{\GG}{\mathbf{I}}}$.
\end{corollary} 

Hence, if we use the global types formalisms discussed above, which guarantee certain communication properties,
we can be sure that these
properties hold for composed types as well. %, i.e.\ GTIRs applied to the respective formalism.
Notice that, as a generalisation, multiple connections of global types could also be considered
with a CFSM interpretation along the lines of Section~\ref{sec:mulconn}. 



%\vspace{5mm}
%===============Old version below for comparison\\
%
%\textbf{Rolf:}
%Below is the old definition.
%Problems with this definition are:\\
%1) GTIRs can only be composed expressions,
%hence just a global type is not a GTIR.\\
%%2) By rule (iii) a composed preGTIR can become a GTIR if the roles H and K are compatible.
%%Then, according to ii), the projection of the composed GTIR to role H is not the gateway process
%%but the CFSM of role H. This is strange, since in the semantics it would be the gateway CFSM.\\
%2)
%With the new idea not to declare interfaces beforehand, the definition of what is a GTIR by rule (iii)
%is not sound anymore. The problem is that you may use the same role twice as an interface.
%The following can happen:\\
%You  compose
%G1 $\connect{\HH}{\KK}$ G2. Then you compose  (G1 $\connect{\HH}{\KK}$ G2) $\connect{\HH}{\LL}$ G3.
%So you have used two times H for the composition. In the first composition
%the compatibility relies on the CFSM $M_H$ of H. In the second composition it relies again on $M_H$.
%But it should rely on the gateway CFSM gw($M_H$,K).\\
%3) Also associativity would probaly not hold anymore.\\
%4) As a solution I have proposed the new part above. 
%
%\begin{definition}[GTIR]\hfill
%\label{def:pre-gtir}
%\begin{enumerate}[i)]
%\item
%The set of {\em pre-GTIR} expressions $\GTIR{{\GG}}{\mathbf{I}}$ is defined by simultaneous induction together with
%their sets of roles $\roles(\GTIR{\GG}{\mathbf{I}})$ and
% {\em components}  $\components(\GTIR{\GG}{\mathbf{I}})$:
%\begin{enumerate}[a)]
%\item $\GTIR{G}{\mathbf{I}}\in \preGTIR$ and
%$\roles(\GTIR{\GG}{\mathbf{I}}) = \roles(G)$ and 
%$\components (\GTIR{G}{\mathbf{I}})  = \Set{G}$ if\\
%	 $G$ is a global type of \gt;
%\item $\GTIR{\GTIR{{\GG}_1}{\mathbf{H}}\connect{\HH}{\KK}\GTIR{{\GG}_2}{\mathbf{K}}}{\mathbf{I}} \in \preGTIR$ 
%	and
%	$\roles(\GTIR{\GTIR{{\GG}_1}{\mathbf{H}}\connect{\HH}{\KK}\GTIR{{\GG}_2}{\mathbf{K}}}{\mathbf{I}}) = 
%	\roles({\GTIR{{\GG}_1}{\mathbf{H}}}) \cup  \roles(\GTIR{{\GG}_2}{\mathbf{K}})$
%and
%	$\components(\GTIR{\GTIR{{\GG}_1}{\mathbf{H}}\connect{\HH}{\KK}\GTIR{{\GG}_2}{\mathbf{K}}}{\mathbf{I}}) = 
%	\components({\GTIR{{\GG}_1}{\mathbf{H}}}) \cup  \components(\GTIR{{\GG}_2}{\mathbf{K}})$ if
%	\begin{enumerate}[1)]
%	\item  $\GTIR{{\GG}_1}{\mathbf{H}}, \GTIR{{\GG}_2}{\mathbf{K}}\in\preGTIR$ with
%		$\HH\in\roles(\GTIR{{\GG}_1}{\mathbf{H}}), \; \KK\in\roles(\GTIR{{\GG}_2}{\mathbf{K}})$,
%%	\item $\mathbf{I} = (\mathbf{H}\cup\mathbf{K})\setminus\Set{\HH,\KK}$,
%	\item
%	$\roles(\GTIR{{\GG}_1}{\mathbf{H}}) \cap \roles(\GTIR{{\GG}_2}{\mathbf{K}})
%	=\emptyset.$
%%	\for all $G\in \components(\GTIR{{\GG}_1}{\mathbf{H}}), G'\in\components(\GTIR{{\GG}_2}{\mathbf{K}})$,  
%%		$\roles(G)\cap\roles(G')=\emptyset$.
%	\end{enumerate}
%\end{enumerate}
%	\item 
%Let $\GTIR{{\GG}}{\mathbf{I}}$ be a pre-GTIR and let ${\tt p}\in  \roles(\GTIR{{\GG}}{\mathbf{I}}) $. 
%We define
%$$\projecton{\GTIR{{\GG}}{\mathbf{I}}}{{\tt p}} =
%                  \projecton{G}{{\tt p}} \hspace{3mm}\text{ where } G\in\components(\GTIR{{\GG}}{\mathbf{I}}) \text{ such that } {\tt p}\in \roles({G}).$$
%	
%\item
%	A pre-GTIR $\GTIR{\GTIR{{\GG}_1}{\mathbf{H}}\connect{\HH}{\KK}\GTIR{{\GG}_2}{\mathbf{K}}}{\mathbf{I}}$
%obtained by the composition of two GTIRs $\GTIR{\GG_1}{\mathbf{H}}$
%and $\GTIR{\GG_2}{\mathbf{K}}$ via the interface roles $\HH$ and $\KK$ is a GTIR if\,  $\HH$ and $\KK$ are interface compatible, i.e.\ $\projecton{\GTIR{{\GG_1}}{\mathbf{I}}}{\HH}\interfacecomp \projecton{\GTIR{{\GG_2}}{\mathbf{I}}}{\KK}$.
%\end{enumerate}
%\end{definition}
%
%By the above definition, a pre-GTIR is an expression formed by either a global type in \gt\ or a number of global types in \gt\ ``composed''  via syntactic operators of the form $\ \connect{\HH}{\KK}$.
%These global types are what  we have defined as the {\em components} of the pre-GTIR.
%
%\begin{definition}[GTIR semantics]\hfill\\
%\label{def.semgtir}
%The communicating system $\Sem{\GTIR{{\GG}}{\mathbf{I}}}$ denoted by a GTIR $\GTIR{{\GG}}{\mathbf{I}}$ 
%is inductively defined as follows:
%\begin{itemize}
%\item[-]
%$\Sem{\GTIR{{G}}{\mathbf{I}}} =  (\projecton{G}{\ttp})_{\ttp\in\roles(G)}$   where $G$ is a global type in \gt;
%
%\item[-]
%$\Sem{\GTIR{\GTIR{{\GG}_1}{\mathbf{H}}\connect{\HH}{\KK}\GTIR{{\GG}_2}{\mathbf{K}}}{\mathbf{I}}}
%= \Sem{   \GTIR{{\GG}_1}{\mathbf{H}}  } \connect{\HH}{\KK} \Sem{\GTIR{{\GG}_2}{\mathbf{K}}}.$
%%   \cup   \Set{ \gateway{\projecton{\GTIR{{\GG}_1}{\mathbf{H}}}{{\HH}}, \KK} , \gateway{\projecton{\GTIR{{\GG}_2}{\mathbf{K}}}{{\KK}}, \HH}}$
%\end{itemize}
%\end{definition}
%\noindent
%%CFSMs in $\Sem{\GTIR{{\GG}}{\mathbf{I}}}$ that correspond to interface roles will be called {\em gateway CFSM}.\\
%
%It is immediate to check that the operation of  ``connecting'' GTIRs is semantically commutative and associative,
%i.e. the following holds:\\[2mm]
%({\em comm}) \hspace{4pt}
%$ \Sem{\GTIR{\GTIR{{\GG}_1}{\mathbf{H}}\connect{\HH}{\KK}\GTIR{{\GG}_2}{\mathbf{K}}}{\mathbf{I}}}
% = \Sem{\GTIR{\GTIR{{\GG}_2}{\mathbf{K}}\connect{\KK}{\HH}\GTIR{{\GG}_1}{\mathbf{H}}}{\mathbf{I}}} $ \\
%({\em ass})\hspace{19pt}
% $\Sem{
%  \GTIR{\GTIR{\GTIR{{\GG}_1}{\mathbf{H}}\connect{\HH}{\KK}\GTIR{{\GG}_2}{\mathbf{K}}}{\mathbf{I}}
%           \connect{\II}{\JJ}
%            \GTIR{{\GG}_3}{\mathbf{J}} }
%           {\mathbf{I'}}
% }
% =
% \Sem{\GTIR{
%                    \GTIR{{\GG}_1}{\mathbf{H}}\connect{\HH}{\KK} 
%                                     \GTIR{
%                                               \GTIR{{\GG}_2}{\mathbf{K}}  \connect{\II}{\JJ} \GTIR{{\GG}_3}{\mathbf{J}}
%                                               }{\mathbf{J'}}
%                               }{\mathbf{I'}}
% }  
% $\\
%
% 
%Let now $P$ be one of the communication properties of Section \ref{sect:cfsm}. By our preservation results we
%get immediately the following corollaries.
%
%
%
%\begin{corollary}
%\label{cor:safetypres}
%Let $\GTIR{\GG}{\mathbf{I}} =\GTIR{\GTIR{{\GG}_1}{\mathbf{H}}\connect{\HH}{\KK}\GTIR{{\GG}_2}{\mathbf{K}}}{\mathbf{I}}$
%be the GTIR composed from GTIRs $\GTIR{\GG_1}{\mathbf{H}}$
%and $\GTIR{\GG_2}{\mathbf{K}}$ via compatible interface roles $\HH$ and $\KK$.
%If both $\Sem{\GTIR{\GG_1}{\mathbf{H}}}$ and $\Sem{\GTIR{\GG_2}{\mathbf{K}}}$ enjoy  the property $P$,
%so does $\Sem{\GTIR{\GG}{\mathbf{I}}}$.
%\end{corollary}
%
%
%As a consequence, by induction on the pairwise composition of GTIRs,
%we obtain the following result.
%
%\begin{corollary}
%Let $\GTIR{\GG}{\mathbf{I}}$ be a GTIR  such that, for any global graph $G\in \components (\GTIR{G}{\mathbf{I}})$,\\
%the system $(\projecton{G}{\ttp})_{\ttp\in\roles(G)}$ enjoy the property $P$.
%Then so does $\Sem{\GTIR{\GG}{\mathbf{I}}}$.
%\end{corollary}
%
%%It is immediate to extend our GTIR sysntax and semantics with multiple connections.
%
%===End of old version







 
 
 


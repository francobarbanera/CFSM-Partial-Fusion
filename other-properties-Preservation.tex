%!TEX root = Main-asynchCFSM-multicomp.tex

%Given a transition sequence $\xi$, we now define the subsequence of $\xi$ made only
%of transitions having a given role as
%sender or receiver.
%
%\begin{definition}
%Let $\xi$ be a transition sequence for a system S and let $\ttp\in\roles(S)$.\\
%We define $\restrictup{\xi}{\ttp}\in \textit{Act}^*$  by 
%$$\restrictup{\xi}{\ttp} = \left\{\begin{array}{ll}
%                                                \varepsilon & \text{if } \xi=s  \;\; (\text{i.e. } |\xi|=0) \\
%                                               \restrictup{\xi'}{\ttp}\cdot\elle & 
%                                                         \text{if } \xi=\xi'\lts{\elle}s \text{ and } [ \elle= \_\ttp?\_  \vee \elle= \ttp\_!\_ ]\\
%                                               \restrictup{\xi'}{\ttp}  & \text{otherwise}
%                                               %\text{if } \xi=\xi'\lts{\elle}s \text{ and } [ \elle\neq \_\ttp?\_  \wedge \elle\neq \ttp\_!\_ ]
%                                      \end{array} \right. 
%$$
%\end{definition}
%
%Notice that by definitions of $\gateway{\cdot}$ and  $\restrictup{(\cdot)}{(\cdot)}$,  
%if $\xi$ is a transition sequence leading to an $s\in RS(S)$, then
% $\restrictup{\xi}{\HH}$ is made (but for its last element, in case $|\restrictup{\xi}{\HH}|$ is odd) of consecutive pairs of elements of $\textit{Act}$  of the form
%$\KK\HH?a \,\, \HH\tts!a$ or $\tts\HH?a \,\,\HH\KK!a$, with $\tts\neq\KK$.
%Moreover, in case $|\restrictup{\xi}{\HH}|$ is odd, it ends with an action of the form
%$\KK\HH?a$ or $\tts\HH?a$, with $\tts\neq\KK$.
%Similarly for $\restrictup{\xi}{\KK}$.\\
%
%We define now two functions, one enabling to map traces of ${M_\HH}$ (i.e. the gateway $\gateway{M^1_\HH, \KK}$)
%  into the corresponding traces of ${M^1_\HH}$
%and one enabling to map traces of ${M_\KK}$  (i.e. the gateway $\gateway{M^2_\KK, \HH}$) into the corresponding traces of ${M^2_\KK}$.
%Traces ending in elements of $\widehat{Q_\HH}$ or $\widehat{Q_\KK}$ will be treated differently
%for what concerns their last elements.
%
%
%\begin{definition}
%Given $\varphi\in\textit{Act}^*$, we define $\symb{\HH\KK}{\varphi}\in (\Set{?,!}\times\mathbb{A})^* $  by 
%$$\symb{\HH\KK}{\varphi} = \left\{\begin{array}{ll}                                             
%                                                \varepsilon & \text{if }  \varphi  = \varepsilon  \text{ or } \varphi = \tts\HH?\_ \text{ with } \tts\neq\KK    \\
%                                                !a & \text{if }  \varphi = \KK\HH?a \\
%                                               ?a\cdot\symb{\HH\KK}{\varphi'} & 
%                                                      \text{if } \varphi= \tts\HH?a \cdot \HH\KK!a \cdot \varphi' \text{ with } \tts\neq\KK\\
%                                                !a\cdot\symb{\HH\KK}{\varphi'}   & 
%                                                         \text{if } \varphi= \KK\HH?a \cdot \HH\tts!a \cdot \varphi'  \text{ with } \tts\neq\KK\\
%                                               \textit{undefined}  & \text{otherwise } 
%                                      \end{array} \right. 
%$$
%The function $\symb{\KK\HH}{\varphi}$ is defined by simply exchanging the roles of $\HH$ and $\KK$ in the above definition.
%\end{definition}
%
%\begin{lemma}
%\label{lem:notwomarks}
%Let $\phi\in\lang{M^1_\HH}^\mC$.\\
%Then there are no two messages $a,c\in \mathbb{A}$ such that \\
%$$\phi\cdot !a\in\lang{M^1_\HH}^\mC \mathrm{~and~}  \phi\cdot ?c\in\lang{M^1_\HH}^\mC.$$
%Similarly for $\lang{M^2_\KK}^\mC$
%\end{lemma}
%\begin{proof}
%Easy, by induction on the length of $\phi$, 
%using the
%?!-determinism and no mixed states assumption for
%$M^1_\HH$ imposed by compatibility.
%
%\end{proof}

%\begin{lemma}
%\label{lem:inlangs}
%Let $\xi$ be a transition sequence for a system
%$S={S_{1}}\connect{\HH}{\KK} {S_{2}}$ starting from the initial state. Then
%$$\symb{\HH\KK}{\restrictup{\xi}{\HH}}\in\lang{M^1_\HH}^\mC \text{ and }
%\symb{\KK\HH}{\restrictup{\xi}{\KK}}\in\lang{M^2_\KK}^\mC$$
%\end{lemma}
%\begin{proof}
%We prove only $\symb{\HH\KK}{\restrictup{\xi}{\HH}}\in\lang{M^1_\HH}^\mC$, since 
%$\symb{\KK\HH}{\restrictup{\xi}{\KK}}\in\lang{M^2_\KK}^\mC$ can be proved in a similar way.
%Let $\xi$ be\\
%\centerline{
%$s_0\lts{\elle_1}s_1\lts{\elle_2} \ldots \lts{\elle_{n-1}}s_{n-1}\lts{\elle_n}s_n$
%}
%where $s_i=({\vec{q_i}},{\vec{w_i}})$.\\
%We can proceed by  induction on $n$.\\
%\underline{Base case} $n=0$. \\
%This case is trivial since $\xi = \restrictup{\xi}{\HH}=\varepsilon$. .\\
%\underline{Inductive case} $n\neq 0$.\\
%Let $\xi'$ be the sequence $s_0\lts{\elle_1} \ldots \lts{\elle_{n-1}}s_{n-1}$.
%We distinguish now two possible cases:
%\begin{description}
%\item 
%${q_n}_\HH\in\widehat{Q_\HH}$.\\
%If  $\elle_n = \KK\HH?a$ then, by definition of $\restrictup{\cdot}{\HH}$,
%we have that $\restrictup{\xi}{\HH}= \restrictup{\xi'}{\HH}\cdot\KK\HH?a$. 
%Then, by definition of $\symb{\HH\KK}{\cdot}$, 
%$\symb{\HH\KK}{\restrictup{\xi}{\HH}}=\symb{\HH\KK}{\restrictup{\xi'}{\HH}}\cdot !a$ with
%$\symb{\HH\KK}{\restrictup{\xi'}{\HH}}\in\lang{M^1_\HH}^\mC$ by the induction hypothesis.
%We can now obtain the thesis since, by definition of $\gateway{\cdot}$, we have that, for a certain $q'$ belonging to the states of $M^1_\HH$, $({q_{n-1}}_\HH, \KK\HH?a, {q_n}_\HH), ({q_{n}}_\HH, \HH\tts!a, q') \in \delta_\HH$ and
% $({q_{n-1}}_\HH, \HH\tts!a, q') \in \delta^1_\HH$
% where $\tts \in \roles_1$. Hence $\symb{\HH\KK}{\restrictup{\xi}{\HH}} = \symb{\HH\KK}{\restrictup{\xi'}{\HH}}\cdot !a
%\in \lang{M^1_\HH}^\mC$. \\
% If  $\elle_n = \tts\HH?a$ then, by definition of $\restrictup{\cdot}{\HH}$ and $\symb{\HH\KK}{\restrictup{\cdot}{\HH}}$, we have that $\symb{\HH\KK}{\restrictup{\xi}{\HH}}=\symb{\HH\KK}{\restrictup{\xi'}{\HH}}$
% and hence the thesis follows immediately from the induction hypothesis.\\
% All the other possible forms of $\elle_n$ do not involve $\HH$ and hence $\restrictup{\xi}{\HH}=\restrictup{\xi'}{\HH}$.
% So the thesis follows immediately from the induction hypothesis.
%\item
%${q_n}_\HH\not\in\widehat{Q_\HH}$\\
%It is possible to proceed similarly to the previous case, distinguishing the possible forms of 
%$\elle_n$ involving $\HH$.
%\end{description}
%\end{proof}
%
%
%
%The next definition and the subsequent lemma are used to prove Proposition \ref{lem:halfduplex}  and Corollary \ref{lem:prefix} below.
%In the lemma we shall  check (besides other things) that
%the application of the function $\symb{\_\,\_}{\cdot}$ on a transition sequence yelds a sequence of messages all
%prefixed by `?' and that such a sequence does coincide with
% the content of a buffer. In order to formalize such an equality, we define below a function that inserts a `?' in front
% of any message in a buffer.
%\begin{definition}
%We define $\qm :\messages^* \rightarrow (\Set{?}\times\messages)^*$ by
%$$\qm(\varepsilon) = \varepsilon \hspace{12mm} \qm(a\cdot w') = ?a\cdot\qm(w')$$
%\end{definition}
%
%
%
%\begin{lemma}%\hfill\\
%\label{lem:prefixaux}
%Let $s= (\vec{q},\vec{w}) \in RS(S)$ be a reachable configuration of
%$S={S_{1}}\connect{\HH}{\KK} {S_{2}}$, and
%let $\xi$ be a transition sequence of length $n$ leading to $s\in RS(S)$ from the initial state
%such that, for all $0\leq i\leq n$, ${w_i}_{\HH\KK}\neq \varepsilon \implies {w_i}_{\KK\HH}= \varepsilon$.
%\begin{enumerate}[i)]
%\item
%\label{lem:prefix-iii}
%If $w_{\HH\KK}= w_{\KK\HH}=\varepsilon$, then
% $\Dual{\symb{\HH\KK}{\restrictup{\xi}{\HH}}}=\symb{\KK\HH}{\restrictup{\xi}{\KK}}$;
%
%\item
%\label{lem:prefix-i}
%
%If $w_{\HH\KK}\neq\varepsilon$  then 
%\begin{enumerate}[a)]
%\item 
%$\Dual{\symb{\KK\HH}{\restrictup{\xi}{\KK}}}$
% is a strict prefix of
%$\symb{\HH\KK}{\restrictup{\xi}{\HH}}$;
%\item
%$\symb{\HH\KK}{\restrictup{\xi}{\HH}}\setminus \Dual{\symb{\KK\HH}{\restrictup{\xi}{\KK}}} = \qm({{w}_{\HH\KK}})$.
%\end{enumerate}
%
%\item
%\label{lem:prefix-ii}
%
%If $w_{\KK\HH}\neq\varepsilon$ then 
%\begin{enumerate}[a)]
%\item 
%$\Dual{\symb{\HH\KK}{\restrictup{\xi}{\HH}}}$
% is a strict prefix of
%$\symb{\KK\HH}{\restrictup{\xi}{\KK}}$;
%
%\item
%$\symb{\KK\HH}{\restrictup{\xi}{\KK}}\setminus \Dual{\symb{\HH\KK}{\restrictup{\xi}{\HH}}}
% = \qm({{w}_{\KK\HH}})$
%\end{enumerate}
%\end{enumerate}
%\end{lemma}
%
%\begin{proof}
%Let $\xi$ be the following transition sequence leading to $s\in RS(S)$ from the initial state\\
%\centerline{
%$s_0\lts{\elle_1}s_1\lts{\elle_2} \ldots \lts{\elle_{n-1}}s_{n-1}\lts{\elle_n}s_n=s$}
%where $s_i=(\vec{q_i},\vec{w_i})$.\\
%We show  (\ref{lem:prefix-iii}), (\ref{lem:prefix-i}) and (\ref{lem:prefix-ii}) by simultaneous
% induction over $|\xi|$.\\
%\underline{Base case} $|\xi|=n=0$. \\
%It is immediate to check that  ${w_0}_{\KK\HH}={w_0}_{\HH\KK}=\varepsilon$
%and $\restrictup{\xi}{\HH}=\restrictup{\xi}{\KK}=\varepsilon$. 
%
%Then (\ref{lem:prefix-iii}) trivially holds,
%whereas (\ref{lem:prefix-i}) and (\ref{lem:prefix-ii}) are vacuously satisfied.\\
%\underline{Inductive case} $|\xi|=n\neq 0$.\\
%In case the action $\elle_n$ does not involve neither $\HH$ nor $\KK$, the thesis descends immediately
%from the induction hypothesis on $\xi' \equiv
%s_0\lts{\elle_1} \ldots \lts{\elle_{n-1}}s_{n-1}$, since $ {w_{n-1}}_{\HH\KK}  = w_{\HH\KK}$,
%$ {w_{n-1}}_{\KK\HH}  = w_{\KK\HH}$,  $\symb{\KK\HH}{\restrictup{\xi'}{\KK}}= \symb{\KK\HH}{\restrictup{\xi}{\KK}}$
% and $\symb{\HH\KK}{\restrictup{\xi'}{\HH}}= \symb{\HH\KK}{\restrictup{\xi}{\HH}}$.\\
%Otherwise, we distinguish two cases: either $\elle_n$ corresponds to an action performed by $\HH$ or
%by an action  performed by $\KK$. Let us consider only the first case, since the second one can be treated in the same way.\\
%Now we need to consider the following further possibilities concerning the form
%of $\elle_n$.
%
%
%\begin{description}
%\item
%$\elle_n =\HH\KK!a$.\\
%In such a case, we can infer that ${w_n}_{\HH\KK}= {w_{n-1}}_{\HH\KK}\cdot  a\neq\varepsilon$.
%In this case, (\ref{lem:prefix-iii}) and also  (\ref{lem:prefix-ii}), since by assumption ${w_{n}}_{\KK\HH} = \varepsilon$, when ${w_{n}}_{\HH\KK} \neq \varepsilon$,
% are vacuously satisfied and only item (\ref{lem:prefix-i}) has to be proved.\\
%
%Since $\elle_n =\HH\KK!a$, we have that g
%\begin{equation}
%\label{eq:weq}
%\restrictup{\xi'}{\KK}=\restrictup{\xi}{\KK}
%\end{equation}
%and hence $\symb{\KK\HH}{\restrictup{\xi'}{\KK}} = \symb{\KK\HH}{\restrictup{\xi}{\KK}}$. 
%Moreover, by definition of $\gateway{\cdot}$ and $\symb{\HH\KK}{\cdot}$, 
%\begin{equation}
%\label{eq:xipxi}
%\symb{\HH\KK}{\restrictup{\xi}{\HH}} = \symb{\HH\KK}{\restrictup{\xi'}{\HH}}\cdot ?a
%\end{equation}
%
%
%
%By the hypothesis
% $\forall 0\leq i\leq n. {w_i}_{\HH\KK}\neq \varepsilon \implies {w_i}_{\KK\HH}= \varepsilon$,
%we have to consider only the following subcases:
%\begin{description}
%\item
%      ${w_{n-1}}_{\HH\KK} = {w_{n-1}}_{\KK\HH} = \varepsilon$ and ${w_n}_{\HH\KK} = a$.\\
%By the induction hypothesis for (\ref{lem:prefix-iii}),we get
%$$\Dual{\symb{\HH\KK}{\restrictup{\xi'}{\HH}}}=\symb{\KK\HH}{\restrictup{\xi'}{\KK}} = \symb{\KK\HH}{\restrictup{\xi}{\KK}} $$
%Hence, 
%$$\symb{\HH\KK}{\restrictup{\xi'}{\HH}} = \Dual{\symb{\KK\HH}{\restrictup{\xi}{\KK}}}$$
%Then, by (\ref{eq:xipxi}), we get
%$$\symb{\HH\KK}{\restrictup{\xi}{\HH}} = \symb{\HH\KK}{\restrictup{\xi'}{\HH}}\cdot ?a = 
% \Dual{\symb{\KK\HH}{\restrictup{\xi}{\KK}}} \cdot ?a$$
%Thus we obtain the thesis, namely
%\begin{enumerate}[a)]
%\item 
%$\Dual{\symb{\KK\HH}{\restrictup{\xi}{\KK}}}$
% is a strict prefix of
%$\symb{\HH\KK}{\restrictup{\xi}{\HH}}$;
%\item
%$\symb{\HH\KK}{\restrictup{\xi}{\HH}}\setminus \Dual{\symb{\KK\HH}{\restrictup{\xi}{\KK}}} = ?a = \qm(a) = \qm({{w_n}_{\HH\KK}})$
%\end{enumerate}
%
%\item
%      ${w_{n-1}}_{\HH\KK}\neq\varepsilon$ and ${w_{n}}_{\HH\KK}= {w_{n-1}}_{\HH\KK}\cdot a$\\
%Let us hence assume ${w_{n-1}}_{\HH\KK}$ to be of the form $\phi\cdot b$.\\
%By the induction hypothesis for (\ref{lem:prefix-i}) we have
%\begin{enumerate}[a)]
%\item 
%$\Dual{\symb{\KK\HH}{\restrictup{\xi'}{\KK}}}$
% is a strict prefix of
%$\symb{\HH\KK}{\restrictup{\xi'}{\HH}}$;
%\item
%$\symb{\HH\KK}{\restrictup{\xi'}{\HH}}\setminus \Dual{\symb{\KK\HH}{\restrictup{\xi'}{\KK}}}) = \qm({{w_{n-1}}_{\HH\KK}})$
%\end{enumerate}
%
%     By the above, by (\ref{eq:weq}) and (\ref{eq:xipxi}) and by the fact that
% ${{w_{n}}_{\HH\KK}}= {{w_{n-1}}_{\HH\KK}}\cdot a$ we obtain
% \begin{itemize}
%\item[c)]
%$\Dual{\symb{\KK\HH}{\restrictup{\xi}{\KK}}}$
% is a strict prefix of
%$\mathsf{init}(\symb{\HH\KK}{\restrictup{\xi}{\HH}})$ with $\mathsf{last}(\symb{\HH\KK}{\restrictup{\xi}{\HH}}) = ?a$;
%\item[d)]
%$\mathsf{init}(\symb{\HH\KK}{\restrictup{\xi}{\HH}})\setminus \Dual{\symb{\KK\HH}{\restrictup{\xi}{\KK}}} = \qm(\mathsf{init}({\vec{w_{n}}_{\HH\KK}}))$  with $\mathsf{last}({\vec{w_{n}}_{\HH\KK}}) = ?a $
%\end{itemize}
%Out of the above the thesis descends immediately.
%\end{description}
%
%
%\item
%       $\elle_n =\HH\tts!a$ with $\tts \in \roles_1$ (hence $\tts\neq\KK$).\\
%Since $\elle_n =\HH\tts!a$, we have  
%\begin{equation}
%\label{eq:weq2}
%\restrictup{\xi'}{\KK}=\restrictup{\xi}{\KK}
%\end{equation}
%and 
%\begin{equation}
%\label{eq:weqw}
%{w_{n-1}}_{\KK\HH} = {w_{n}}_{\KK\HH} \text{ and } {w_{n-1}}_{\HH\KK} = {w_{n}}_{\HH\KK}
%\end{equation}
%
%Moreover, by definition of $\gateway{\cdot}$ and $\symb{\HH\KK}{\cdot}$, 
%\begin{equation}
%\label{eq:xipxi2}
%\symb{\HH\KK}{\restrictup{\xi}{\HH}} = \symb{\HH\KK}{\restrictup{\xi'}{\HH}}
%\end{equation}
%
%The thesis hence follows by the induction hypothesis.
%
%
%\item
%       $\elle_n =\KK\HH?a$\\
%Since $\elle_n =\KK\HH?a$, we have 
%\begin{equation}
%\label{eq:weq3}
%\restrictup{\xi'}{\KK}=\restrictup{\xi}{\KK}
%\end{equation}
%Moreover, by definition of $\gateway{\cdot}$ and $\symb{\HH\KK}{\cdot}$, 
%\begin{equation}
%\label{eq:xipxi3}
%\symb{\HH\KK}{\restrictup{\xi}{\HH}} = \symb{\HH\KK}{\restrictup{\xi'}{\HH}}\cdot !a
%\end{equation}
%
%By the hypothesis
% $\forall 0\leq i\leq n. {w_i}_{\HH\KK}\neq \varepsilon \implies {w_i}_{\KK\HH}= \varepsilon$,
%we have to consider only the following subcases:
%
%\begin{description}
%\item
%      ${w_{n-1}}_{\KK\HH} = a\cdot {w_{n}}_{\KK\HH}$ with ${w_{n}}_{\KK\HH}\neq \varepsilon$\\
%In this case, (\ref{lem:prefix-iii}) and (\ref{lem:prefix-i}) are vacuously satisfied. For what concerns
%(\ref{lem:prefix-ii}),
%by the induction hypothesis we have
%\begin{enumerate}[a)]
%\item 
%\label{l:aa}
%$\Dual{\symb{\HH\KK}{\restrictup{\xi'}{\HH}}}$
% is a strict prefix of
%$\symb{\KK\HH}{\restrictup{\xi'}{\KK}}$;
%\item
%\label{l:bb}
%$\symb{\KK\HH}{\restrictup{\xi'}{\KK}}\setminus \Dual{\symb{\HH\KK}{\restrictup{\xi'}{\HH}}} = \qm({{w_{n-1}}_{\KK\HH}})$
%\end{enumerate}
%
%     By the above, using  (\ref{eq:weq3}), (\ref{eq:xipxi3}), we  get
%\begin{itemize}
%\item[c)]
%\label{l:cc}
%$\Dual{\init(\symb{\HH\KK}{\restrictup{\xi}{\HH}})}
%\text{ is a strict prefix of }
%\symb{\KK\HH}{\restrictup{\xi}{\KK}}$
%with $\last(\Dual{\symb{\HH\KK}{\restrictup{\xi}{\HH}}})= ?a$;
%\item[d)]
%\label{l:dd}
%$\symb{\KK\HH}{\restrictup{\xi}{\KK}}\setminus \Dual{\init(\symb{\HH\KK}{\restrictup{\xi}{\HH}})} =  \qm({w_{n-1}}_{\KK\HH}) =?a\cdot\qm({{w_{n}}_{\KK\HH}})$
%\end{itemize}
%
%and then, by c) and  d) above  the thesis descends immediately.
%
%\item
%      ${w_{n-1}}_{\KK\HH} = a\cdot {w_{n}}_{\KK\HH}$ with ${w_{n}}_{\KK\HH}={w_{n}}_{\HH\KK} = \varepsilon$\\
%In this case, (\ref{lem:prefix-i})  and (\ref{lem:prefix-ii}) are vacuously satisfied. For what concerns
%(\ref{lem:prefix-iii}),
%by the induction hypothesis for (\ref{lem:prefix-ii}) we have
%\begin{enumerate}[a)]
%\item 
%\label{l:aa}
%$\Dual{\symb{\HH\KK}{\restrictup{\xi'}{\HH}}}$
% is a strict prefix of
%$\symb{\KK\HH}{\restrictup{\xi'}{\KK}}$;
%\item
%\label{l:bb}
%$\symb{\KK\HH}{\restrictup{\xi'}{\KK}}\setminus \Dual{\symb{\HH\KK}{\restrictup{\xi'}{\HH}}} = \qm({{w_{n-1}}_{\KK\HH}}) = ?a$
%\end{enumerate}
%     By the above, using  (\ref{eq:weq3}), (\ref{eq:xipxi3}), we  get
%\begin{itemize}
%\item[c)]
%\label{l:ccc}
%$\Dual{\init(\symb{\HH\KK}{\restrictup{\xi}{\HH}})}
%\text{ is a strict prefix of }
%\symb{\KK\HH}{\restrictup{\xi}{\KK}}$ 
%with $\last(\Dual{\symb{\HH\KK}{\restrictup{\xi}{\HH}}})= ?a$;
%\item[d)]
%\label{l:ddd}
%$\symb{\KK\HH}{\restrictup{\xi}{\KK}}\setminus\Dual{\init(\symb{\HH\KK}{\restrictup{\xi}{\HH}})} =  ?a$
%\end{itemize}
%and then, by c) and  d) above we can infer the thesis, namely
%$$\Dual{\symb{\HH\KK}{\restrictup{\xi}{\HH}}} = 
%\symb{\KK\HH}{\restrictup{\xi}{\KK}}$$
%
%\end{description}
%
%
%\item
%       $\elle_n =\tts\HH?a$ with  $\tts \in \roles_1$ (hence $\tts\neq\KK$)\\
%Since $\elle_n = \tts\HH?a$, we have  
%\begin{equation}
%\label{eq:weq2}
%\restrictup{\xi'}{\KK}=\restrictup{\xi}{\KK}
%\end{equation}
%and 
%\begin{equation}
%\label{eq:weqw}
%{w_{n-1}}_{\KK\HH}= {w_{n}}_{\KK\HH} \text{ and } {w_{n-1}}_{\HH\KK}= {w_{n}}_{\HH\KK}
%\end{equation}
%
%Moreover, by definition of $\gateway{\cdot}$ and $\symb{\HH\KK}{\cdot}$, 
%\begin{equation}
%\label{eq:xipxi2}
%\symb{\HH\KK}{\restrictup{\xi}{\HH}} = \symb{\HH\KK}{\restrictup{\xi'}{\HH}}
%\end{equation}
%
%The thesis hence follows by the induction hypothesis.
%
%\end{description}
%
%\end{proof}
%
%
%The following proposition essentially shows that in gateway-connected systems any pair of FIFO channels  connecting
%gateways is such that in each reachable configuration at least one of the two buffers is empty, that is they
% can be replaced by a half-duplex channel.
%
%\begin{proposition}%\hfill\\
%\label{lem:halfduplex}
%Let $s= (\vec{q},\vec{w}) \in RS(S)$ be a reachable configuration of
%$S={S_{1}}\connect{\HH}{\KK} {S_{2}}$.
%Then $w_{\HH\KK}$ and  $w_{\KK\HH}$ cannot be both non-empty. That is
%$${w}_{\HH\KK}\neq \varepsilon \implies {w}_{\KK\HH}= \varepsilon$$
%\end{proposition}
%
%\begin{proof}
%Towards a contradiction, we assume the thesis not to hold.
%We then take, among all the transition sequences $\zeta$ leading (from the initial state) to a state $s_{\zeta}= (\vec{q_\zeta},\vec{w_\zeta}) \in RS(S)$
%such that 
%\begin{equation}
%\label{tchyp}
%{w_\zeta}_{\HH\KK}\neq \varepsilon  \text{ and } {w_\zeta}_{\KK\HH} \neq \varepsilon,
%\end{equation}
%a sequence having a minimal length.
% Let the following sequence $\xi$ be such a sequence.
%$$s_0\lts{\elle_1}s_1\lts{\elle_2} \ldots \lts{\elle_{n-1}}s_{n-1}\lts{\elle_n}s_n$$
%where $s_i=({\vec{q_i}},{\vec{w_i}})$. And let $s_{\xi}$ be $s=(\vec{q},\vec{w}) \in RS(S)$.
%Since ${w_0}_{\HH\KK} = {w_0}_{\KK\HH} = \varepsilon$, we have that  $|\xi| > 0$.
%\\
%By the minimality of $|\xi |$, we can infer that one of the following two cases necessarily holds:
%either    ${w_{n-1}}_{\KK\HH} = \varepsilon$ or ${w_{n-1}}_{\HH\KK} = \varepsilon$.
%%We can assume the first case to hold since the other one can be treated similarly.
%Without loss of generality, assume
%  $w_{{n-1}_{\KK\HH}} = \varepsilon$.\\
%In this case, as a consequence of (\ref{tchyp}), we have that ${w_{n-1}}_{\HH\KK}= {w_\xi}_{\HH\KK} \neq \varepsilon$.\\
%So we are assuming that 
%\begin{equation}
%\label{allnonbothnonempty}
%{w_i}_{\HH\KK}\neq \varepsilon \implies {w_i}_{\KK\HH}= \varepsilon \text{~~ for all } 0\leq i \leq n-1
%\end{equation}
%and
%\begin{equation}
%\label{ngetsnonempty}
%{{w_{n-1}}}_{\HH\KK} \neq\varepsilon  \text{ and } {{w_{n-1}}}_{\KK\HH} = \varepsilon \text{ and } {w_{n}}_{\KK\HH}\neq \varepsilon
%\end{equation}
%
%
%To get a contradiction, the idea is the following: First, since ${{w_{n-1}}}_{\HH\KK} \neq\varepsilon$,
%\ref{lem:prefixaux}(\ref{lem:prefix-i})  implies, that the next action of  $M_\KK$ in configuration 
%$s_{n-1}$ can only be the consumption of an element $c$ of the channel ${w_{n-1}}_{\HH\KK}$.
% On the other hand, since ${w_{n-1}}_{\KK\HH}= \varepsilon$ and  ${w_{n}}_{\KK\HH} \neq \varepsilon$, to progress from $s_{n-1}$ to $s_n$  $M_\KK$ must put an element $a$ into the buffer ${w_{n-1}}_{\KK\HH}$. But both is not possible.
%Let us now do the formal proof of the contradiction. \\
%By Lemma \ref{lem:inlangs}, 
%$\symb{\HH\KK}{\restrictup{(\upto{\xi}{n-1})}{\HH}}\in \lang{M^1_\HH}^\mC$ and 
%$\symb{\KK\HH}{\restrictup{(\upto{\xi}{n-1})}{\KK}} \in \lang{M^2_\KK}^\mC$.\\
%Moreover, since we have (\ref{allnonbothnonempty}) and (\ref{ngetsnonempty}), by Lemma \ref{lem:prefixaux} we can infer that  
%\begin{enumerate}[a)]
%\item 
%\label{en:a}
%$\Dual{\symb{\KK\HH}{\restrictup{\upto{\xi}{n-1}}{\KK}}}$
% is a strict prefix of
%$\symb{\HH\KK}{\restrictup{\upto{\xi}{n-1}}{\HH}}$;
%\item
%\label{en:b}
%$\symb{\HH\KK}{\restrictup{\upto{\xi}{n-1}}{\HH}}\setminus \Dual{\symb{\KK\HH}{\restrictup{\upto{\xi}{n-1}}{\KK}}} = \qm({{w_{n-1}}_{\HH\KK}})$.
%\end{enumerate}
%
%Recall that by definitions of $\gateway{\cdot}$ and  $\restrictup{(\cdot)}{(\cdot)}$,  
% we have that $\restrictup{\xi}{\HH}$ is made (but for its last element, in case $|\restrictup{\xi}{\HH}|$ is odd) of consecutive pairs of elements of $\textit{Act}$ either of the form
%$\KK\HH?a \,\, \HH\tts!a$, with $\tts\neq\KK$, or $\tts\HH?a \,\,\HH\KK!a$, with $\tts\neq\KK$.
%Moreover, in case $|\restrictup{\xi}{\HH}|$ is odd, it ends with an action either of the form
%$\KK\HH?a$ or $\tts\HH?a$, with $\tts\neq\KK$.
%Similarly for $\restrictup{\xi}{\KK}$.\\
%
%
%Hence, as a consequence of (\ref{en:a}) and (\ref{en:b}) above, there exists a message $c$ such that 
%$$\Dual{\symb{\KK\HH}{\restrictup{\upto{\xi}{n-1}}{\KK}}}\cdot ?c \in \lang{M^1_\HH}^\mC$$
%Now, from (\ref{ngetsnonempty}) (in particular ${w_{n-1}}_{\KK\HH} = \varepsilon$ and ${w_{n}}_{\KK\HH} \neq \varepsilon$), we can infer also that, for a certain message $a$,
%$$\symb{\KK\HH}{\restrictup{(\upto{\xi}{n-1})}{\KK}}\cdot ?a \in \lang{M^2_\KK}^\mC.$$
%By definition of compatibility we have $\lang{M^1_\HH}^\mC = \Dual{\lang{M^2_\KK}^\mC}$ and hence
% $$\Dual{\symb{\KK\HH}{\restrictup{(\upto{\xi}{n-1})}{\KK}}\cdot ?a} = \Dual{\symb{\KK\HH}{\restrictup{(\upto{\xi}{n-1})}{\KK}}}\cdot !a \in \lang{M^1_\HH}^\mC$$
% So we have both $\Dual{\symb{\KK\HH}{\restrictup{\upto{\xi}{n-1}}{\KK}}}\cdot ?c \in \lang{M^1_\HH}^\mC$
% and $ \Dual{\symb{\KK\HH}{\restrictup{(\upto{\xi}{n-1})}{\KK}}}\cdot !a \in \lang{M^1_\HH}^\mC$, which, by Lemma \ref{lem:notwomarks}
% is a contradiction.
%\end{proof}

%The next corollary is an immediate consequence of Lemma \ref{lem:prefixaux}  and Proposition \ref{lem:halfduplex}. It is the key for getting our preservation results in the next section.
%
%\begin{corollary}%\hfill\\
%\label{lem:prefix}
%Let $s= (\vec{q},\vec{w}) \in RS(S)$ be a reachable configuration of
%$S={S_{1}}\connect{\HH}{\KK} {S_{2}}$, and
%let $\xi$ be a transition sequence leading to $s\in RS(S)$ from the initial state.
%\begin{enumerate}[i)]
%\item
%\label{lem:prefixcor-iii}
%If $w_{\KK\HH}= w_{\HH\KK}=\varepsilon$, then
% $\Dual{\symb{\HH\KK}{\restrictup{\xi}{\HH}}}=\symb{\KK\HH}{\restrictup{\xi}{\KK}}$;
%
%\item
%\label{lem:prefixcor-i}
%
%If $w_{\HH\KK}\neq\varepsilon$  then 
%\begin{enumerate}[a)]
%\item 
%$\Dual{\symb{\KK\HH}{\restrictup{\xi}{\KK}}}$
% is a strict prefix of
%$\symb{\HH\KK}{\restrictup{\xi}{\HH}}$;
%\item
%$\symb{\HH\KK}{\restrictup{\xi}{\HH}}\setminus \Dual{\symb{\KK\HH}{\restrictup{\xi}{\KK}}} = \qm({{w}_{\HH\KK}})$.
%\end{enumerate}
%
%\item
%\label{lem:prefixcor-ii}
%
%If $w_{\KK\HH}\neq\varepsilon$ then 
%\begin{enumerate}[a)]
%\item 
%$\Dual{\symb{\HH\KK}{\restrictup{\xi}{\HH}}}$
% is a strict prefix of
%$\symb{\KK\HH}{\restrictup{\xi}{\KK}}$;
%
%\item
%$\symb{\KK\HH}{\restrictup{\xi}{\KK}}\setminus \Dual{\symb{\HH\KK}{\restrictup{\xi}{\HH}}}
% = \qm({{w}_{\KK\HH}})$
%\end{enumerate}
%\end{enumerate}
%\end{corollary}
%\begin{proof}
%Immediate by Lemma \ref{lem:prefixaux} and Proposition \ref{lem:halfduplex}.
%\end{proof}










\subsection{No-orphan-message preservation}


% \begin{lemma}
%\label{lem:wempty}
%%Let $S={S_{1}}\connect{\HH}{\KK} {S_{2}}$.
%If $s= (\vec{q},\vec{w}) \in RS(S)$ is a reachable configuration of $S={S_{1}}\connect{\HH}{\KK} {S_{2}}$ 
%such that ${q}_\KK$ is final, then ${w}_{\HH\KK} = \varepsilon$.
%%both  states $\vec{q}_\HH$ and $\vec{q}_\KK$ are final, then $\vec{w}_{\HH\KK} = \vec{w}_{\KK\HH} = \varepsilon$.
%The same holds by exchanging $\HH$ and $\KK$.
%\end{lemma}
%
%\begin{proof}
%By Fact. \ref{fact:uniquesending}(\ref{fact:uniquesending-i}), 
%%$q_\HH\notin \widehat{Q_\HH}$ and 
%$q_\KK\notin \widehat{Q_\KK}$.
%Let now $\xi$ be a transitions sequence leading to $s\in RS(S)$ from the initial state, say\\
%\centerline{
%$s_0\lts{\elle_1}s_1\lts{\elle_2} \ldots \lts{\elle_{n-1}}s_{n-1}\lts{\elle_n}s_n=s$
%}
%Towards a contradiction, let us assume $\vec{w}_{\HH\KK}\neq \varepsilon$ 
%%(the case  $\vec{w}_{\KK\HH}\neq \varepsilon$ can be treated similarly). 
%Hence, by Corollary \ref{lem:prefix} we get
%\begin{enumerate}[a)]
%\item 
%\label{l:aaaa}
%$\Dual{\symb{\KK\HH}{\restrictup{\xi}{\KK}}}$
% is a strict prefix of
%$\symb{\HH\KK}{\restrictup{\xi}{\HH}}$;
%\item
%\label{l:bbbb}
%$\symb{\HH\KK}{\restrictup{\xi}{\HH}}\setminus \Dual{\symb{\KK\HH}{\restrictup{\xi}{\KK}}} = \qm({{w}_{\HH\KK}})$.
%\end{enumerate}
%Now, by ?!-determinism of 
%%$M^1_\HH$ and 
%$M^2_\KK$ and by 
%%$q_\HH\notin \widehat{Q_\HH}$ and 
%$q_\KK\notin \widehat{Q_\KK}$, we have that $\vec{q}_\KK$ is the unique state of $M^2_\KK$ recognising the string 
%$\symb{\KK\HH}{\restrictup{\xi}{\KK}}\in \lang{M^2_\KK}^\mC$.\\
%% Let now $q$ be the unique state of $M^2_\KK$ 
%%recognizing the string $\Dual{\symb{\KK\HH}{\restrictup{\xi}{\KK}}}$.\\
%Now, by (\ref{l:aaaa}) and (\ref{l:bbbb}) above and knowing, by Lemma \ref{lem:inlangs}, that $\symb{\HH\KK}{\restrictup{\xi}{\HH}}\in\lang{M^1_\HH}^\mC $, 
%there exists a message $a$ such that
%$\Dual{\symb{\KK\HH}{\restrictup{\xi}{\KK}}}\cdot ?a \in \lang{M^1_\HH}^\mC$.
%Hence, by compatibility, $\symb{\KK\HH}{\restrictup{\xi}{\KK}}\cdot !a \in \lang{M^2_\KK}^\mC$.
%Contradiction, since $\vec{q}_\KK$ is final.
%\end{proof}



\begin{lemma}%\hfill\\
\label{lem:restrRSom}
%Let $S = \MC(\Set{S_i}_{i\in I}, \cs)$ and 
Let $s= (\vec{q},\vec{w}) \in \RS(S)$ be an orphan-message configuration of $S$.
Then either
\begin{itemize}
\item
 there exists $i\in I$ such that $\restrict{s}{i}\in \RS(S)$ and $\restrict{s}{i}$ is an  orphan-message configuration of $S_i$;
or 
\item
$\restrict{s}{\cs}\in \RS(\cs)$ and $\restrict{s}{\cs}$ is an  orphan-message configuration of $\cs$.
\end{itemize}
\end{lemma}

\begin{proof}
Let $s= (\vec{q},\vec{w}) \in \RS(S)$ be an orphan-message configuration of $S$, 
that is $\vec{q}$ is final and $\vec{w}\neq \vec{\varepsilon}$.
Since $\vec{q}$ is final we get, by  \cref{fact:uniquesending}(\ref{fact:uniquesending-i},  that, 
for each $i\in I$, $q_{\HH_i}\not\in \widehat{Q_{\HH_i}}$.
So, by~\cref{lem:nohatrestrict}, $\restrict{s}{\cs}\in \RS(\cs)$
and $\restrict{s}{i}\in \RS(S_i)$ for each $i\in I$. 
We have to consider now only the following two cases:
\begin{description}
\item
$\exists i\in I.\exists \ttp,\ttq\in \roles_i$ such that  $w_{\ttp\ttq}\neq\varepsilon$.\\
In such a case we immediately get, by definition of projection, that $\restrict{s}{i}$ is an  orphan-message configuration of $S_i$.
\item
$\exists \ttu,\ttv\in \Set{\HH_i}_{i\in I} \cup \rolescsint$ with  $w_{\ttu\ttv}\neq\varepsilon$ \\
Let $\restrict{s}{\cs}=({\vec{q'}},{\vec{w'}})$. Since $\vec{q}$ is final,
by definition of projection (\cref{def:projectedconf}), $\vec{q'}$ is final as well.
Moreover, for each pair $\ttu,\ttv\in \roles_{\cs}$ with $\ttu\neq\ttv$, 
we have that $w'_{\ttu\ttv} = w_{\tilde{\tilde{\ttu}}\tilde{\tilde{\ttv}}}$.
%$w'_{\KK_j\KK_v} = w_{\HH_j\HH_v}$, for each ${j,v\in I}$ with $j\neq v$.
This implies that $\vec{w'}\neq \vec{\varepsilon}$.
Hence, $\restrict{s}{\cs}$ is an orphan-message configuration of $\cs$.
\end{description}

%
%OLD proof
%
%By hypothesis,  let $s= (\vec{q},\vec{w}) \in RS(S)$ be an orphan-message configuration for $S$, 
%that is $\vec{q}$ is final and $\vec{w}\neq \vec{\varepsilon}$.
%Since $\vec{q}$ is final we get, by Fact \ref{fact:uniquesending},  that, 
%for each $j\in I$, $q_{\HH_j}\not\in \widehat{Q_{\HH_j}}$.
%So, by Lemma \ref{lem:nohatrestrict}, $\restrict{s}{\cs}\in RS(\cs)$
%and, for each $v\in I$, $\restrict{s}{v}\in RS(S_v)$ .
%
%We have to consider now only the following two cases.
%\begin{description}
%\item
%$\exists v\in I.\exists \ttp,\ttq\in \roles_v$ such that  $w_{\ttp\ttq}\neq\varepsilon$\\
%In such a case we immediately get that $\restrict{s}{v}$ is an  orphan-message configuration of $S_v$.
%\item
%$\exists \ttp,\ttq\in \Set{\HH_i}_{i\in I}$ with  $w_{\ttp\ttq}\neq\varepsilon$ \\
%If $\restrict{s}{\cs}=({\vec{q}\,'},{\vec{w}'})$,
%by definition of projections we have that 
%$w'_{\KK_j\KK_v} = w_{\HH_j\HH_v}$, for each ${j,v\in I}$ such that $j\neq v$.
%This implies that $\vec{w}'\neq \vec{\varepsilon}$.
%That is $\restrict{s}{\cs}$ is an orphan-message configuration for $\cs$.
%\end{description}
\end{proof}


\begin{corollary}[Preservation of orphan-message-freedom]%\hfill\\
\label{prop:nomPreservation}
Let $S = \MC(\Set{S_i}_{i\in I}, \cs)$ such that, for each $i\in I$, 
$S_i$ is orphan-message free and also $\cs$ is orphan-message free.
% $\RS(S_{i})$ has no orphan-message configuration and also $\cs$
%has no orphan-message configuration. 
Then $S$ is orphan-message free.
%$\RS(S)$ has no orphan-message configuration.
\end{corollary}
\begin{proof}
By contradiction, let us assume there is an $s\in \RS(S)$ which is an orphan-message configuration. Then we get
a contradiction by Lemma \ref{lem:restrRSom}.
\end{proof}


\subsection{Preservation of no unspecified reception}

%The following lemma is crucial for proving our preservation results for absence of unspecified receptions and progress.
%
%
%\begin{lemma}%\hfill\\
%\label{lem:getright}
%Let $s= (\vec{q},\vec{w}) \in RS(S)$ be a reachable configuration of
%$S={S_{1}}\connect{\HH}{\KK} {S_{2}}$ such that 
%all the transitions from $q_\KK$ in $\delta_\KK$ are of the form $(q_\KK,\HH\KK?\_,\_)$.
%Then
%$$w_{\HH\KK}= a\cdot w' \implies  \exists (q_\KK,\HH\KK?a,\_)\in\delta_\KK$$
%The same property holds by exchanging $\HH$ and $\KK$.
% \end{lemma}
% 
% \begin{proof}
%Let $\xi$ be a transition sequence leading to $s$ from the initial state, in particular let $\xi$ be\\
%\centerline{
%$s_0\lts{\elle_1}s_1\lts{\elle_2} \ldots \lts{\elle_{n-1}}s_{n-1}\lts{\elle_n}s_n=s$
%}
%where $s_i=({\vec{q_i}},{\vec{w_i}})$.\\
%Moreover, let us assume $w_{\HH\KK}= a\cdot w'$.
%Now, by Corollary \ref{lem:prefix}(\ref{lem:prefixcor-i}) 
%we have that
%\begin{enumerate}[a)]
%\item 
%$\Dual{\symb{\KK\HH}{\restrictup{\xi}{\KK}}}$
% is a strict prefix of
%$\symb{\HH\KK}{\restrictup{\xi}{\HH}}$;
%\item
%$\symb{\HH\KK}{\restrictup{\xi}{\HH}}\setminus \Dual{\symb{\KK\HH}{\restrictup{\xi}{\KK}}}
%= \qm({w_{\HH\KK}}) = ?a\cdot  \qm({w'}) $.
%\end{enumerate}
%Moreover $\symb{\HH\KK}{\restrictup{\xi}{\HH}}\in\lang{M^1_\HH}^\mC$ and
%$\symb{\KK\HH}{\restrictup{\xi}{\KK}}\in\lang{M^2_\KK}^\mC$ by Lemma \ref{lem:inlangs}.
%
%Now, let $j$ be the greatest index such that ${q_j}_\HH\not\in\widehat{Q_\HH}$ and 
%$\symb{\HH\KK}{\restrictup{\upto{\xi}{j}}{\HH}} = \Dual{\symb{\KK\HH}{\restrictup{\xi}{\KK}}}$.
%Then necessarily $s_j\lts{\tts\HH?a} s_{j+1}$.
%% and  $s_{j'}\lts{\HH\KK!a} s_{j'+1}$ for a $j'$ such that $j<j'\leq n$. 
%Now, by ?!-determinism of $M^1_\HH$ and $M^2_\KK$, we have that ${q_j}_\HH$ is the unique state of $M^1_\HH$ recognising the string 
%$\symb{\HH\KK}{\restrictup{\upto{\xi}{j}}{\HH}}$ and ${q}_\KK$ is the unique state of $M^2_\KK$ 
%recognising the string $\symb{\KK\HH}{\restrictup{\xi}{\KK}} = \Dual{\symb{\HH\KK}{\restrictup{\upto{\xi}{j}}{\HH}}}$.
%Obviously, $\symb{\KK\HH}{\restrictup{\xi}{\KK}}\cdot !a =
%\Dual{\symb{\HH\KK}{\restrictup{\upto{\xi}{j}}{\HH}}\cdot ?a}$. Because 
% $\symb{\HH\KK}{\restrictup{\upto{\xi}{j}}{\HH}}\cdot ?a \in \lang{M^1_\HH}^\mC$
%and, by compatibility,  $\lang{M^1_\HH}^\mC=  \Dual{\lang{M^2_\KK}^\mC}$ we then know that 
%$\symb{\KK\HH}{\restrictup{\xi}{\KK}}\cdot !a \in \lang{M^2_\KK}^\mC$.
%Hence, by definition of $\symb{\KK\HH}{\cdot}$, $\exists ({q}_\KK,\HH\KK?a,\_)\in\delta_\KK$.\\
% The very same argument can be used to show the statement with $\HH$ and $\KK$ exchanged. 
% \end{proof}
% 
% We are now ready to prove preservation of no unspecified reception.
%Intuitively this holds, since, by Lemma~\ref{lem:getright}, no ``wrong'' elements can be put
%in a gateway-connecting channel, if the interface machines are compatible.   
 
%Let $S = \MC(\Set{S_i}_{i\in I}, \cs)$ such that the interfaces of each $S_i$,
%i.e.\ the CFSMs $M_{h_i}$ in $S_i$, have no mixed state.

\begin{proposition}[Preservation of reception-error freedom]%\hfill\\
\label{prop:nurPreservation}
Let $S = \MC(\Set{S_i}_{i\in I}, \cs)$ such that, for each $i\in I$, no element in $\RS(S_{i})$ and also no element in
$\RS(\cs)$ is an unspecified reception configuration.
Moreover, let the interfaces of each $S_i$,
i.e.\ the CFSMs $M_{\hh_i}$ in $S_i$, have no mixed state.
Then there is no unspecified reception configuration in $\RS(S)$.
% OLD version
%Let $S = \MC(\Set{S_i}_{i\in I}, \cs)$ such that
%\bfr the interfaces of the $S_i$'s have no mixed state. \efr
%Moreover, for each $i\in I$, let no element in $RS(S_{i})$ be
%an unspecified reception configuration, and let that hold for $RS(\cs)$ as well.
%Then there is no unspecified reception configuration in $RS(S)$.
\end{proposition}



\begin{proof}
By contradiction, let us assume there is an $s= (\vec{q},\vec{w})\in \RS(S)$ which is an unspecified reception configuration.
So, let $\ttr \in \roles$ and let ${q}_\ttr$  be the receiving state of $M_\ttr$ prevented from 
receiving any message from any of its buffers, which all are not empty (Definition \ref{def:safeness}(\ref{def:safeness-ur})).
We consider two main cases $\ttr \not\in\rolescsint$ and  $\ttr\in\rolescsint$.\\[2mm]
$\bullet$ $\ttr \not\in\rolescsint$\\
In particular, let $z\in I$ such that $\ttr \in \roles_z$. Now we  take into account the following possible cases where, for the sake of readability,
we set $\HH=\HH_z$:

\begin{description}
\item 
${q}_\HH\not\in \widehat{Q_\HH}$.\\ 
By~\cref{lem:nohatrestrict}(\ref{lem:nohatrestrict-a}) we get $\restrict{s}{z}\in \RS(S_z)$.
% and $\restrict{s}{\cs}\in RS(\cs)$.
We distinguish now two further subcases.
\begin{description}
\item 
$\ttr \neq \HH$\\
Then $\restrict{s}{z}\in \RS(S_z)$ is an unspecified reception configuration of
$S_z$. Contradiction!
\item 
$\ttr = \HH$\\
Since ${q}_\ttr(= {q}_\HH)$ is a receiving state,
by definition of gateway it follows that
 the set
of all outgoing transitions from $q_\HH$ in $\delta_\HH$ is of the form 
$$\Set{({q}_\HH,\tts_j\HH?\msg[a]_j,\widehat{q_j})}_{j=1..m}$$
where $\widehat{q_j}\in  \widehat{Q_\HH}$ and 
 $\tts_j \in \roles_z \cup \{\HH_i\}_{i \in I\setminus\Set{z}}\cup\rolescsint.$
%$\tts_j \in \roles_z \cup \{\HH_i\}_{i \in I}, \tts_j \neq \HH.$ 

By definition of unspecified reception configuration,  we have hence that for all $j=1..m$, 
\begin{equation} \label{eq:wur}
\mid w_{\tts_j\HH}\mid > 0 
\text{ and } w_{\tts_j\HH}\not\in  \msg[a]_j \cdot \mathbb{A}^*  %\mathbb{A}^*\cdot a_j
\end{equation}
Now, due to the no-mixed-state assumption, it suffices to
consider the following two possibilities:\\

\begin{description}
\item
\underline{$\diamond$}
{\it For each  $j=1..m$, $\tts_j \not\in \{\HH_i\}_{i \in I\setminus\Set{z}}\cup\rolescsint$}\\ %\{\HH_i\}_{i \in I}.$}\\
In this case we can infer from~\cref{fact:uniquesending}\ref{fact:uniquesending-iiib}) that, for each $j=1..m$, there exists $q'_j \in Q_{\HH}$
such that $(q_\HH,\tts_j\HH?\msg[a]_j,q'_j)\in\delta^z_\HH.$
 By definition of projection (\cref{def:projectedconf}) it follows that, for all $j=1..m$,
$w'_{\tts_j\hh}=w_{\tts_j\hh}$, where 
$\restrict{s}{z}=(\vec{q'},\vec{w'})$. 
 This implies that $\restrict{s}{z}\in \RS(S_z)$ is an  unspecified reception configuration of $S_z$. Contradiction!\\
%
\item
\underline{$\diamond$} {\it  For each  $j=1..m$,
 $\tts_j\in \{\HH_i\}_{i \in I\setminus\Set{z}}\cup\rolescsint$.  
%$v_j\in I$ such that $\tts_j = \HH_{v_j}$
} \\
In this case, by~\cref{fact:uniquesending}\ref{fact:uniquesending-iiia}), we have that, for each $j=1..m$, 
$$(q_\HH,\tts_j\HH?\msg[a]_j,\widehat{q_j}), (\widehat{q_j},\HH\ttw_j!\msg[a]_j,q'_j)\in\delta_\HH $$
where $\ttw_j \in \roles_z, q'_j \in Q_\HH$ and 
$(q_{\HH},\HH\ttw_j!\msg[a]_j,q'_j)\in\delta^z_\HH.$

Moreover, again by~\cref{fact:uniquesending}\ref{fact:uniquesending-iiia}), %by definition of multi-composition w.r.t.\ connection policy $\cs$, 
we have that, for each $j=1..m$,
$$(\dot{q}_\HH,\widetilde{\tts_j}\KK_z?\msg[a]_j,\dot{q'_j})\in\delta^\cs_{\KK_z} $$
We consider now the possible two subcases.\\
$-$ ${q}_{\HH_i}\not\in\widehat{Q_{\HH_i}}$ for each $i\in I$.\\
Then, by~\cref{lem:nohatrestrict}(\ref{lem:nohatrestrict-b}) we get $\restrict{s}{\cs}\in \RS(\cs)$. 
If $\restrict{s}{\cs} = (\vec{p},\vec{w'})$, by definition of projection we have that,
 for each $i \in   I$, $p_{\kk_i} = \dot{q_{\HH_i}}$  and,
 for each $\ttu \in   \rolescsint$, $p_{\pu} = q_{\pu}$. Moreover,
  for each pair $\ttp,\ttq\in \roles_{\cs}$ with $\ttp\neq\ttq$, 
 $w'_{\ttp\ttq} = w_{\tilde{\tilde{\ttp}}\tilde{\tilde{\ttq}}}$.

%$p_{\KK_z}=\dot{q}_\HH$ and $w'_{\KK_x\KK_y} = w_{\HH_x\HH_y}$ for each $x,y\in I$ such that $x\neq y$.
The last equalities and  (\ref{eq:wur}) imply that, for each $j = 1..m$,
$$
\mid w_{\widetilde{\tts_j}\KK_z}\mid > 0 
\text{ and } w_{\widetilde{\tts_j}\KK_z}\not\in  \msg[a]_j \cdot\mathbb{A}^*.
$$
 Thus $\restrict{s}{\cs} \in \RS(\cs)$ is an unspecified reception configuration of $\cs$. Contradiction!\\
 $-$ There exists $k\in I$ such that ${q}_{\HH_k}\in\widehat{Q_{\HH_k}}$.\\
 By Lemma \ref{lem:addendum}, there exists $s'= (\vec{q''},\vec{w''})\in \RS(S)$ 
 such that $s \to^* s'$,
 $|w_{\ttp\ttq}| \leq |w''_{\ttp\ttq}|$ for all $\ttp\ttq \in C$ and
$q''_{\HH_j}\not\in\widehat{Q_{\HH_j}}$ for each $j\in I$. 
Moreover, $q''_{\HH}=q_{\HH}$ since we have assumed ${q}_\HH\not\in \widehat{Q_\HH}$.
By~\cref{lem:nohatrestrict}(\ref{lem:nohatrestrict-b}) we get $\restrict{s'}{\cs}\in \RS(\cs)$. By reasoning as in the previous case, we get that 
$\restrict{s'}{\cs}$ is an unspecified reception configuration of $\cs$. Contradiction!
\end{description}
\end{description}
\item 
${q}_\HH\in \widehat{Q_\HH}$.\\ 
Then, by \cref{fact:uniquesending}(\ref{fact:uniquesending-i}),
${q}_\HH\in \widehat{Q_\HH}$ is a sending state such that $({q}_\HH,\HH\tts!\msg[a],q'_\HH)\in{\delta}_\HH$ with $q'_\HH \in Q_{\HH}$ and 
$\tts \in \roles_z \cup \Set{\HH_i}_{i\in I\setminus\Set{z}}\cup\rolescsint$. 
Since $q_\ttr$ is a receiving state, it is impossible that $\ttr=\HH$.
So, let $\ttr\neq\HH$. We recall that we are dealing with the case when also $\ttr\not\in\rolescsint$.
It is now immediate to check that  there exists $s'\in \RS(S)$ such that
$s\lts{\HH\tts!\msg[a]}s'=(\vec{q'},\vec{w'})$ with $q'_\HH$ and $\tts$ as above.
Since $q'_\HH\not\in \widehat{Q_\HH}$ it
follows, by~\cref{lem:nohatrestrict}(\ref{lem:nohatrestrict-a}), that $\restrict{s'}{z} \in \RS(S_z)$.
Moreover, we have that 
\begin{enumerate}[a)]
\item
\label{l:aa}
$\forall \ttp\neq\HH.\ q'_\ttp = {q}_\ttp$ and,
in particular, $q'_\ttr = {q}_\ttr$;
\item
\label{l:bb}
$\forall \ttp\ttq \neq \HH\tts.\ w'_{\ttp\ttq} = {w}_{\ttp\ttq}$;
\item
\label{l:cc}
$w'_{\HH\tts} = {w}_{\HH\tts}\cdot \msg[a]$.
\end{enumerate}
We consider now the following two possible subcases:
\begin{description}
\item
 $\tts \in\Set{\HH_v}_{v\in I\setminus\Set{z}}\cup\rolescsint$.  \\
%$\tts = \HH_v$ for some $v\in I$ with $v \neq z$.\\
Then $\tts \neq \ttr$, since $\ttr \in \roles_z$.
From ($\ref{l:aa}$) and ($\ref{l:bb}$) above it follows that  $q'_\ttr = {q}_\ttr$
and, since  $\tts \neq \ttr$, $w'_{\ttp\ttr} = {w}_{\ttp\ttr}$ for all $\ttp \in \roles_z.$ Consequently, $\restrict{s'}{z} \in \RS(S_z)$ is an unspecified reception configuration of $S_z$. Contradiction!
\item
%$\tts=\ttr$ (and hence $\tts \in \roles_1$)\\
$\tts \in \roles_z$\\
If $\tts \neq \ttr$ we get a contradiction as in the previous case.\\
If $\tts = \ttr$, then $\HH$ sends the message $\msg[a]$ to the buffer $w_{\HH\ttr}$. Since $q_\ttr$ is the receiving state of $M_\ttr$ prevented from receiving any message from any of its buffers, which all are not empty in configuration $s$, the sending of $\msg[a]$ extends $w_{\HH\ttr}$ which still has a wrong element on its first position. Then, by ($\ref{l:aa}$) and ($\ref{l:bb}$) above $\restrict{s'}{z}$ is an unspecified reception configuration of $S_z$.  Contradiction!
\end{description}
\end{description}
We can now proceed with the second of the two main cases, namely\\[2mm]
$\bullet$ $\ttr \in\rolescsint$\\
We consider the two following possible subcases:
\begin{description}
\item
$q_{\hh_i}\not\in\widehat{Q_{\hh_i}}$ for each $i\in I$.\\
Then $\restrict{s}{\cs}$ is defined and $\restrict{s}{\cs} \in \RS(\cs)$. 
Let $\restrict{s}{\cs} = (\vec{p},\vec{w'})$.
By definition of projection (\cref{def:projectedconf}) it follows that, for each $\ttu\in\rolescsint$,
$p_{\ttu} = q_{\ttu}$ and, for each $\ttu\in\roles_{\cs}$ with $\ttu\neq\ttr$,   $w'_{\ttu\ttr} = w_{\tilde{\tilde{\ttu}}\tilde{\tilde{\ttr}}}= w_{\tilde{\tilde{\ttu}}\ttr}$ (the last equality holds by definition of $\widetilde{\widetilde{\cdot}}$). Consequently, $\restrict{s'}{\cs} \in \RS(\cs)$ is an unspecified reception configuration of $\cs$ for the participant $\ttr$. Contradiction!
\item

$q_{\hh_i}\in\widehat{Q_{\hh_i}}$ for some $i\in I$.\\
 By Lemma \ref{lem:addendum}, there exists $s'= (\vec{q''},\vec{w''})\in \RS(S)$ 
 such that $s \to^* s'$,
 $|w_{\ttp\ttq}| \leq |w''_{\ttp\ttq}|$ for all $\ttp\ttq \in C$ and
$q''_{\HH_j}\not\in\widehat{Q_{\HH_j}}$ for each $j\in I$. 
Now, since $\ttr\in\rolescsint$ and also $s'$ is necessarily a unspecified reception configuration of $S$ for the participant $\ttr$, we can proceed as in the previous case.
\end{description}
\end{proof}


% OLD proof
%\begin{proof}
%By contradiction, let us assume there is an $s= (\vec{q},\vec{w})\in RS(S)$ which is an unspecified reception configuration.
%So, let $\ttr \in \roles$ and let ${q}_\ttr$  be the receiving state of $M_\ttr$ prevented from 
%receiving any message from any of its buffers (Definition \ref{def:safeness}(\ref{def:safeness-ur})).
%In particular, let $z\in I$ such that $\ttr \in \roles_z$. \\
%Now we  take into account the following possible cases where, for the sake of readability,
%we set $\HH=\HH_z$:
%
%\begin{description}
%\item 
%${q}_\HH=\widehat{q}\not\in \widehat{Q_\HH}$.\\ 
%By~\cref{lem:nohatrestrict} we get $\restrict{s}{z}\in RS(S_z)$ and $\restrict{s}{\cs}\in RS(\cs)$.
%\brc The latter is not true with the current formulation of ~\cref{lem:nohatrestrict}(ii) because we need as an assumption $({q}_{\HH_i}\not\in\widehat{Q_{\HH_i}}$ for each $i\in I)$ and not only ${q}_\HH=\widehat{q}\not\in \widehat{Q_\HH}$.
%To repair this, I think we must remove the assumption $({q}_{\HH_i}\not\in\widehat{Q_{\HH_i}}$ for each $i\in I)$ and we can do this (as you have already remarked one time.). Of course, the proof must then be reconsidered.
%\erc
%We distinguish now two further subcases.
%\begin{description}
%\item 
%$\ttr \neq \HH$\\
%We get a contradiction by the hypothesis that $RS(S_{z})$ does not contain any unspecified reception configuration.
%\item 
%$\ttr = \HH$\\
%Since ${q}_\ttr(= {q}_\HH)$ is a receiving state,
%by definition of gateway it follows that
% the set
%of all the outgoing transitions from $q_\HH$ in $\delta_\HH$ is of the form 
%$$\Set{({q}_\HH,\tts_j\HH?a_j,\widehat{q_j})}_{j=1..m}$$
%By definition of unspecified reception configuration,  we have hence that for all $j=1..m$, 
%\begin{equation} \label{eq:wur}
%\mid w_{\tts_j\HH}\mid > 0 
%\text{ and } w_{\tts_j\HH}\not\in \bmr \msg[a]_j \cdot \mathbb{A}^* \emr %\mathbb{A}^*\cdot a_j
%\end{equation}
%Now, the following further possibilities have to be taken into account\\
%\underline{$\diamond$} {\it  $\tts_j\neq\HH_v$ for each $j=1..m$ and each $v\in I$.}\\
%By Fact \ref{fact:uniquesending}(\ref{fact:uniquesending-iii}) and definition of gateway, we have that  
%$$[(q_\HH,\tts_j\HH?a_j,\widehat{q_j})\in\delta_\HH  ~~\wedge ~~
%\forall v\in I. \tts_j\neq\HH_v] \iff  
%(q_\HH,\tts_j\HH?a_j,q_j)\in\bmr\delta^z_\HH\emr$$
% This implies $\restrict{s}{z}$ to be an  unspecified reception configuration for $S_z$. Contradiction.\\
%\underline{$\diamond$} {\it  For each  $j=1..m$, there exists $v\in I$ such that $\tts_j = \HH_{v_j}$} \\
%In this case, by definition of gateway, we can infer that 
%$$(q_\HH,\HH_{v_j}\HH?a_j,\widehat{q_j}), (\widehat{q_j},\HH\ttu_j!a_j,q'_j)\in\delta_\HH $$
%Moreover, by definition of connection policy, we have that
%$$(\dot{q}_\HH,\KK_{v_j}\KK_z?a_j,\dot{q'_j})\in\delta^\cs_{\KK_z} $$
%We consider now the possible two subcases.\\
%$-$ ${q}_{\HH_k}\not\in\widehat{Q_{\HH_k}}$ for each $k\in I$.\\ 
%If $\restrict{s}{\cs} = (\vec{p},\vec{w}')$, by definition of projection we have that
%$p_{\KK_v}=\dot{q}_\HH$
%\brc should be $p_{\KK_z}$ \erc
%and $w'_{\KK_x\KK_y} = w_{\HH_x\HH_y}$ for each $x,y\in I$ such that
%$x\neq y$. This last equality and  (\ref{eq:wur}) imply that, for each $j = 1..m$,
%$$
%\mid w_{\KK_{v_j}\KK_z}\mid > 0 
%\text{ and } w_{\KK_{v_j}\KK_z}\not\in \bmr \msg[a]_j \cdot\mathbb{A}^*.\emr%\mathbb{A}^*\cdot a_j.
%$$
% So $\restrict{s}{\cs}$ is an unspecified reception configuration for $\cs$
% by Lemma \ref{lem:nohatrestrict}. Contradiction.\\
% $-$ There exists $k\in I$ such that ${q}_{\HH_k}\in\widehat{Q_{\HH_k}}$.\\
% By Lemma \ref{lem:addendum}, there exists $s'= (\vec{q}\,'',\vec{w}'')$ 
% such that $s \lts{}^* s'$ (so $s'\in RS(S)$) and
%${q''}_{\HH_j}\not\in\widehat{Q_{\HH_j}}$ for each $j\in I$.
%Hence, by Lemma \ref{lem:nohatrestrict}, we get $\restrict{s'}{\cs}\in RS(\cs)$ with
%$q''_{\HH}=q_{\HH}$.
%\brc
%We need additionally that the channels in s' are the same or extensions of
%those in s. 
%\erc
%By reasoning as in the previous case, we get that 
%$\restrict{s'}{\cs}$ is an unspecified reception configuration for $\cs$. Contradiction.
%\\
%\underline{$\diamond$} 
%No other case is possible because of the no mixed state condition.
%%which is part of the compositionality condition. 
%% do get a contradiction by Lemma \ref{lem:getright}.
%\end{description}
%%
%\item 
%${q}_\HH=\widehat{q}\in \widehat{Q_\HH}$.\\ 
%By Fact \ref{fact:uniquesending}(\ref{fact:uniquesending-i}),
%${q}_\HH\in \widehat{Q_\HH}$ is a sending state such that $({q}_\HH,\HH\tts!a,{q'}_\HH)\in{\delta}_\HH$. Hence it is impossible that $\ttr=\HH$.
%So, let $\ttr\neq\HH$.
%It is now immediate to check that  there exists an element $s'\in RS(S)$ such that
%$s\lts{\HH\tts!a}s'=(\vec{q'},\vec{w'})$ with ${q'}_\HH\not\in \widehat{Q_\HH}$
%and $\tts \in \roles_z \cup \Set{\HH_i}_{i\in I}$.
%It hence follows, by Lemma \ref{lem:nohatrestrict}, that $\restrict{s'}{z} \in RS(S_z)$.
%Moreover, we have that 
%\begin{enumerate}[a)]
%\item
%\label{l:aa}
%$\forall \ttp\neq\HH.\ {q'}_\ttp = {q}_\ttp$;
%\item
%\label{l:bb}
%$\forall \ttp\ttq \neq \HH\tts.\ {w'}_{\ttp\ttq} = {w}_{\ttp\ttq}$;
%\item
%\label{l:cc}
%${w'}_{\HH\tts} = a\cdot{w}_{\HH\tts}$.
%\end{enumerate}
%We consider now the following two possible subcases:
%\begin{description}
%\item
%%$\tts\neq\ttr$ (and hence $\tts = \KK$)\\
%$\tts = \HH_v$ for some $v\in I$.\\
%By ($\ref{l:aa}$) and ($\ref{l:bb}$) above it follows that  also $\restrict{s'}{z} \in RS(S_z)$ is an unspecified reception configuration. Contradiction.
%\item
%%$\tts=\ttr$ (and hence $\tts \in \roles_1$)\\
%$\tts \in \roles_z$\\
%If $\tts \neq \ttr$, by ($\ref{l:aa}$) and ($\ref{l:bb}$) above it follows that  also $\restrict{s'}{z} \in RS(S_z)$ is an unspecified reception configuration. Contradiction.\\
%If $\tts = \ttr$, then $\HH$ sends the message $a$ to the buffer $w_{\HH\ttr}$. Since $q_\ttr$ is the receiving state of $M_\ttr$ prevented from receiving any message from any of its buffers, which all are not empty in configuration $s$, the sending of $a$ extends $w_{\HH\ttr}$ which still has a wrong element on its first position. Then, by ($\ref{l:aa}$) and ($\ref{l:bb}$) above $\restrict{s'}{z}$ is an unspecified reception configuration of $S_z$.  Contradiction.
%\end{proof}

%$\spadesuit\spadesuit\spadesuit\spadesuit\spadesuit\spadesuit$

\subsection{Progress preservation}

%For ensuring the progress property for $S = \MC(\Set{S_i}_{i\in I}, \cs)$ it is not enough that
%$\cs$ and each $S_i$ (for $i\in I$) do enjoy the progress property.
%In fact we can simply get 
%[[Insert counterexample: $\cs$ with 4 participants; two continuously exchanging messages and the
%other two stuck]]

\begin{proposition}[Progress preservation]%\hfill\\
\label{lem:restrRS}
Let $S = \MC(\Set{S_i}_{i\in I}, \cs)$ such that $\cs$ and each $S_i$ (for all $i\in I$) satisfy 
%do enjoy 
the progress property. 
 Moreover, let the interfaces of each $S_i$,
i.e.\ the CFSMs $M_{\hh_i}$ in $S_i$, have no mixed state.
Then also $S$ satisfies the progress property. 
\end{proposition}

\begin{proof}
The proof is by contradiction.
Let us assume $S$ does not enjoy the progress property, namely that there exists 
 $s= (\vec{q},\vec{w}) \in \RS(S)$ such that
 \begin{equation}
 \label{eq:snotprogr}
 \text{$s\notlts{}\hspace{2mm}$ and $\hspace{2mm}\vec{q}$ is not final, i.e.\
 $\exists \ttr\in\roles. ~ q_\ttr \text{ is not final in } M_\ttr$.}
\end{equation}
We distinguish between the two following cases $\ttr\not\in\rolesorch_{\cs}$ and $\ttr\in\rolesorch_{\cs}$:\\[2mm]
$\bullet$ $\ttr\not\in\rolesorch_{\cs}$.\\
In particular, there exists $z \in I$ and $\ttr \in \roles_z$ such that $q_\ttr$ is not final in $M_\ttr$.


By $s\notlts{}$  and by \cref{fact:uniquesending}(\ref{fact:uniquesending-i}), we have that
for each $i\in I$, $q_{\HH_i}\not\in\widehat{Q_{\HH_i}}$. Otherwise
there would be an output transition from some $q_{\HH_i}$, contradicting $s\notlts{}$.
So, by~\cref{lem:nohatrestrict}, we get $\restrict{s}{i}\in \RS(S_i)$ for each $i\in I$, as well as $\restrict{s}{\cs}\in \RS(\cs)$. 
In particular, $\restrict{s}{z}\in \RS(S_z).$

Now we distinguish two cases:
\begin{description}
\item \emph{Case 1: $q_{\HH_z}$ is final in $M_{\HH_z}$.}\\
Then $q_{\HH_z}$ is final in $M^z_{\HH_z}$.
Consequently, $s\notlts{}$ and $\restrict{s}{z}\in \RS(S_z)$ implies $\restrict{s}{z}\notlts{}$. Moreover, we know that $q_\ttr$ is not final in $M_\ttr$ which is
an element of the system $S_z$. Hence, $S_z$ does not enjoy the progress property. Contradiction!

\item \emph{Case 2: $q_{\HH_z}$ is not final in $M_{\HH_z}$.}\\
From above we know  $q_{\HH_z}\not\in\widehat{Q_{\HH_z}}$ and hence,
by construction of gateways,
there must be an input transition in $M_{\HH_z}$ starting in $q_{\HH_z}$.
We distinguish two subcases:

\begin{description}
\item \emph{Case 2.1:
There exists a transition from $q_{\HH_z}$ in $\delta_{\HH_{z}}$ of the form
$(q_{\HH_z},\tts\HH_z?\_,\_)$ with $\tts\in\roles_z$.}\\
By the no-mixed-state assumption and by construction of gateways
it follows that all transitions in $\delta_{\HH_{z}}$ with source state
$q_{\HH_z}$ have this form.
%\brr ALSO here I must be more careful in the following part.
%I inserted the modified part after that.\\
%Since $s\notlts{}$ and $\restrict{s}{z}\in RS(S_z)$, we have $\restrict{s}{z}\notlts{}$. Moreover, $q_{\HH_z}$ is not final in $M_{\HH_z}$ and $q_{\HH_z}\not\in\widehat{Q_{\HH_z}}$.
%Therefore $q_{\HH_z}$ is also not final in $M^z_{\HH_z}$. 
%Thus $S_z$ does not enjoy the progress property. Contradiction!
%\err
Then (i) also all transitions in $\delta^z_{\HH_{z}}$ with source state
$q_{\HH_z}$ have this form and (ii) there exists at least one such transition.
Since $s\notlts{}$ and $\restrict{s}{z}\in \RS(S_z)$ we obtain, as a consequence of (i), that $\restrict{s}{z}\notlts{}$. As a consequence of (ii) $q_{\HH_z}$ is also not final in $M^z_{\HH_z}$. 
Thus $S_z$ does not enjoy the progress property. Contradiction!
%
\item \emph{Case 2.2:
There exists a transition from $q_{\HH_z}$ in $\delta_{\HH_{z}}$ of the form
$(q_{\HH_z},\ttu\HH_z?\_,\_)$ with $\ttu\in\Set{\hh_i}_{ \in I\setminus\Set{z}}\cup\rolesorch_{\cs}$.}\\
By the no-mixed-state assumption and by construction of gateways
it follows that all transitions in $\delta_{\HH_{z}}$ with source state
$q_{\HH_z}$ have this form. 
Then, by definition of gateway and orchestrated connection policy, all transitions
in $\delta^{\cs}_{\KK_{z}}$ have the form
$(\dot{q}_{\KK_z},\widetilde{\ttu}\KK_z?\_,\_)$ and there exists at least one such
transition.
Let now $\restrict{\vec{s}}{\cs}= (\vec{q'},\vec{w'})$.
By definition of projection,
$w'_{\ttv\KK_{z}}=w_{\widetilde{\widetilde{\ttv}}\HH_{z}}$  for each $\ttv\in\roles_{\cs}$.
So, since  $s \notlts{}$ and $\restrict{s}{\cs}\in \RS(\cs)$, we can infer that $\restrict{s}{\cs}\notlts{}$.
Moreover, $\dot{q}_{\KK_z}$ is not final in $M_{\KK_z}$.
Thus $S_{\cs}$ does not enjoy the progress property. Contradiction!
\end{description}
\end{description}
 
$\bullet$ $\ttr\in\rolesorch_{\cs}$.\\
As in the previous case, $s\notlts{}$  and \cref{fact:uniquesending}(\ref{fact:uniquesending-i})
implies that, for each $i\in I$, $q_{\HH_i}\not\in\widehat{Q_{\HH_i}}$. Otherwise
there would be an output transition from some $q_{\HH_i}$, contradicting $s\notlts{}$.
So, by~\cref{lem:nohatrestrict}, we get $\restrict{s}{\cs}\in \RS(\cs)$.
We now proceed by considering the following two possible cases.
\begin{description}
\item
{\em There exists a $j\in I$ such that $q_{\hh_j}$ is not final in $M_{\hh_j}$}\\
Trivially $\hh_j\in\roles_j$. We can hence simply proceed as in the first main case, where it is shown that we can get to a contradiction by assuming $s\notlts{}$ and the 
presence of a participant belonging to $\roles_j$, for some $j$, in a non final state.  
 \item
{\em For all $j\in I$, $q_{\hh_j}$ is final in $M_{\hh_j}$}\\
By definition of projection, it immediately follows that $q_{\ttr}$ is not final in $\cs$.
We can hence get a contradiction with the progress property of $\cs$ since we can show that
$\restrict{s}{\cs}\notlts{}$. Toward a contradiction, let
$\restrict{s}{\cs}\lts{\elle}$ for some $\elle$. 
By definition of gateways and from the hypothesis of the present case 
we have that for all $j\in I$, $\dot{q_{\kk_j}}$ is final in $M^{\cs}_{\kk_j}$.
Hence, necessarily, $\elle=\ttu\ttr?\_$ where $\ttu\in\rolesorch_{\cs}$, i.e. 
$\ttu\not\in\Set{\kk}_{i\in I}$. 
This would then imply,  by definitions of projection, composition and transition,  that
$s\lts{\elle}$, so contradicting $s\notlts{}$.
\end{description}

\end{proof}


%%%%%%%%%%%%OLD PROOF%%%%%%%%%%%%%%%%
%\begin{proof}
%By contraposition, let us assume $S$ not to enjoy the progress property, namely that there exists 
% $s= (\vec{q},\vec{w}) \in RS(S)$ such that
% \begin{equation}
% \label{eq:snotprogr}
% \text{$s\notlts{}\hspace{2mm}$ and $\hspace{2mm}\vec{q}$ is not final.}
%\end{equation}
%%It is immediate to check that this can be equivalently rephrased as
%%\begin{equation}
%% \label{eq:snotprogrequiv}
%%\text{$\vec{q}$ is not final and  $\exists\ttr\in\roles$ such that $\vec{q}_\ttr$ is a receiving state of $M_\ttr$}
%%\end{equation}
%
%
%By $s\notlts{}$  and by Fact \ref{fact:uniquesending}(\ref{fact:uniquesending-i}), we have that
%for each $v\in I$ $q_{\HH_v}\not\in\widehat{Q_\HH}$. Otherwise
%there will be an output transition from each $q_{\HH_v}$, contradicting $s\notlts{}$.
%So, by Lemma \ref{lem:nohatrestrict}, we get $\restrict{s}{v}\in RS(S_v)$ for each $v\in I$,
%as well as $\restrict{s}{\cs}\in RS(\cs)$.
%We show in the following that either
%
%\begin{equation}
% \label{eq:snotprogr1}
%\text{$\exists z\in I.\ \restrict{\vec{s}}{z}\notlts{}\hspace{2mm}$ and $\hspace{2mm}\restrict{\vec{q}}{z}$ is not final;}
%\end{equation}
%or 
%\begin{equation}
% \label{eq:snotprogr2}
%\text{$\restrict{\vec{s}}{\cs}\notlts{}\hspace{2mm}$ and $\hspace{2mm}\restrict{\vec{q}}{\cs}$ is not final.}
%\end{equation}
%
%Once we have shown (\ref{eq:snotprogr1}) or (\ref{eq:snotprogr2}) either $S_v$ or $\cs$ does not enjoy the progress property and we are done.
%
%
%\noindent
% We distinguish now the following possible cases according to  whether the $q_{\HH_v}$'s are final or not.
% \begin{description}
%\item For each $v\in I$ $q_{\HH_v}$ is final in $M_{\HH_v}$, and hence in $M^v_\HH$: \\
%Then from $s\notlts{}$ it immediately follows that, for each $v\in I$, $\restrict{s}{v}\notlts{}$. Moreover, since there exists $\ttr\in\roles$ such that $q_\ttr$ is not final,
% we can infer that there exists $\ttr\in\roles_z$, for a certain $z\in I$,
% such that $\ttr\neq{\HH_z}$ and $q_\ttr$ is not final, contradicting the progress property of $S_z$.
%% or there exists $\ttr\in\roles_2$ such that $\ttr\neq\KK$ and $q_r$ is not final.
%% a receiving state of $M^1_\ttr$.
%% This means that either  $\restrict{s}{1}$  is a deadlock state of  $S_1$ or  $\restrict{s}{2}$  is a deadlock state of  $S_2$ (or both).
%%
%\item There exists $v\in I$ such that $q_{\HH_v}$ is not final in $M_{\HH_v}$, and hence in $M^v_\HH$.\\
%Now, let $\Set{v_j}_{j=1..m}$ be all the indexes in $I$ such that
%$q_{\HH_{v_j}}$ is not final in $M_{\HH_{v_j}}$.
%In such a case, by definition of gateway and 
%by the no mixed state condition on $M^v_{\HH_v}$,
%%imposed by composability,
% we need  to 
%take into account only the following   subcases concerning the shapes of the transitions from the $q_{\HH_{v_j}}$'s  in the $\delta_{\HH_{v_{j}}}$'s.
%
%\underline{$\diamond$} 
%{\em  For each $j=1..m$, the transitions from $q_{\HH_{v_j}}$ in $\delta_{\HH_{v_j}}$ are of the form 
%$(q_{\HH_{v_j}},\HH_{k}\HH_{v_j}?\_,\_)$.}\\
%In such a case, by definition of connection policy and gateway also 
%the transitions from $\dot q_{\KK_{v_j}}$ in $\delta^\cs_{\KK_{v_j}}$ have the shape.
%$(\dot q_{\KK_{v_j}},\KK_{k}\KK_{v_j}?\_,\_)$
% Let now $\restrict{\vec{s}}{\cs}= (\vec{q}',\vec{w}')$.
%We  have that, by definition of projection,
% $w'_{\KK_l\KK_{l'}}=w_{\HH_l\HH_{l'}}$  for each pair $l,l'\in I$.
%So, since  $s \notlts{}$, we can infer that $\restrict{\vec{s}}{\cs}\notlts{}$ and 
%hence (\ref{eq:snotprogr2}) holds.
%
%\underline{$\diamond$} 
%{\em  There exists $j\in\set{1,..,m}$, such that the transitions from $q_{\HH_{v_j}}$
%in $\delta_{\HH_{v_j}}$ are of the form 
%$(q_{\HH_{\tts_j}},\tts_k\HH_{v_j}?\_,\_)$ with $\tts\in\roles\HH_{v_j}$.}\\
%By definition of gateway we have that also the transitions from $q_{\HH_{v_j}}$ in $\delta^{v_j}_{\HH_{v_j}}$ have the same form.
%Moreover,
%by letting $\restrict{\vec{s}}{v_j}= (\vec{q}',\vec{w}') \in RS(S)$,
%we have that, by definition of projection,  
%$w'_{\ttp\ttq}=w_{\ttp\ttq}$ for each pair of participants $\ttp,\ttq\in\roles\HH_{v_j}$.
%So, since  $\notlts{}$, we have also that $\restrict{\vec{s}}{v_j}\notlts{}$ and 
%hence (\ref{eq:snotprogr1}) holds for $z=v_j$.

%
%=============
%
% \item Both $q_\HH$ and $q_\KK$ are non final: \\
%In such a case, by definition of $\gateway{\cdot}$ and by the no mixed state condition on $M^1_\HH$ and $M^2_\KK$
%imposed by compatibility,  we need  to 
%take into account only the following  further possible subcases concerning the shapes of the transitions from  $q_\HH$  in $\delta_\HH$ and from  $q_\KK$ in $\delta_\KK$.
%
%\underline{$\diamond$} 
%{\em  All the transitions from $q_\HH$ in $\delta_\HH$ are of the form $(q_\HH,\KK\HH?\_,\_)$ and
%all the transitions from $q_\KK$ in $\delta_\KK$ are of the form $(q_\KK,\tts\KK?\_,\_)$ with $\tts \in \roles_2$ (and hence $\tts\neq\HH$).}\\
%Since all the transitions from $q_\KK$ in $\delta_\KK$ are of the form $(q_\KK,\tts\KK?\_,\_)$ with $\tts \in \roles_2$ (and hence $\tts\neq\HH$),
%  we can infer, from the definition of  $\gateway{\cdot}$, that also  all the transitions from $q_\KK$ in $\delta^2_\KK$ are of the form $(q_\KK,\tts\KK?\_,\_)$. Hence from $s\notlts{}$ we can  
%obtain that $\restrict{s}{2}\notlts{}$ as well and then  (\ref{eq:snotprogr2}).
%%that $\restrict{s}{2}$ is a deadlock state.
%(Notice that, instead, $\restrict{s}{1}\lts{}$, since, by definition of $\gateway{\cdot}$, all the transitions from 
%$q_\HH$ in $\delta^1_\HH$ are of the form $(q_\HH,\HH\_!\_,\_)$)\!
%\footnote{This fact clearly prevents us from having the preservation of progress in the case where, instead of connecting roles of
%different systems, we connect roles belonging to the same system. This ``self connection'' is equivalent to having multiple connections 
%(see discussion in Section \ref{sec:mulconn}). }
%.
%
%\underline{$\diamond$} 
%{\em All the transitions from $q_\HH$ in $\delta_\HH$ are of the form $(q_\HH,\tts\HH?\_,\_)$ with $\tts \in \roles_1$ (and hence $\tts\neq\KK$). and
%all the transitions from $q_\KK$ in $\delta_\KK$ are of the form $(q_\KK,\HH\KK?\_,\_)$.}\\
%This case can be treated similarly to the previous one.
%
%
%\underline{$\diamond$} 
%{\em  All the transitions from $q_\HH$ in $\delta_\HH$ are of the form $(q_\HH,\tts\HH?\_,\_)$ with $\tts\in\roles_1$ and
%all the transitions from $q_\KK$ in $\delta_\KK$ are of the form $(q_\KK,\tts\KK?\_,\_)$ with $\tts\in\roles_2$.}\\
%This case can be treated similarly to the previous ones.
%
%
%\underline{$\diamond$}
%{\em  All the transitions from $q_\HH$ in $\delta_\HH$ are of the form $(q_\HH,\KK\HH?\_,\_)$. and
%all the transitions from $q_\KK$ in $\delta_\KK$ are of the form $(q_\KK,\HH\KK?\_,\_)$.}\\
%We now consider the possible shapes of $w_{\HH\KK}$ and $w_{\KK\HH}$.
%\begin{description}
%\item
% $w_{\HH\KK}=w_{\KK\HH}=\varepsilon$.\\
%This subcase cannot occur. Towards a contradiction, let us assume it to be possible.\\
%Let now $\xi$ be a transition sequence leading to $s\in RS(S)$ from the initial state, in particular let $\xi$ be\\
%\centerline{
%$s_0\lts{\elle_1}s_1\lts{\elle_2} \ldots \lts{\elle_{n-1}}s_{n-1}\lts{\elle_n}s_n=s$
%}
% By Lemma \ref{lem:inlangs},
% $\symb{\HH\KK}{\restrictup{\xi}{\HH}}\in\lang{M^1_\HH}^\mC $ and $\symb{\KK\HH}{\restrictup{\xi}{\KK}}\in\lang{M^2_\KK}^\mC$. Moreover, by Corollary \ref{lem:prefix}(\ref{lem:prefixcor-iii}),
%  $\Dual{\symb{\HH\KK}{\restrictup{\xi}{\HH}}}=\symb{\KK\HH}{\restrictup{\xi}{\KK}}$.
%  Since all the transitions from $q_\HH$ in $\delta_\HH$ are of the form $(q_\HH,\KK\HH?\_,\_)$. and
%all the transitions from $q_\KK$ in $\delta_\KK$ are of the form $(q_\KK,\HH\KK?\_,\_)$,
%we can infer, by definition of $\gateway{\cdot}$, that  $\symb{\HH\KK}{\restrictup{\xi}{\HH}}\cdot !a\in\lang{M^1_\HH}^\mC $ and $\symb{\KK\HH}{\restrictup{\xi}{\KK}}\cdot !b\in\lang{M^2_\KK}^\mC$ for certain $a$ and $b$.
%
%We have then that $\Dual{\symb{\HH\KK}{\restrictup{\xi}{\HH}}\cdot ?b} = {\symb{\KK\HH}{\restrictup{\xi}{\KK}}\cdot !b} \in\lang{M^2_\KK}^\mC$
%and hence, since by compatibility $\Dual{\lang{M^1_\HH}^\mC} = \lang{M^2_\KK}^\mC$, also that
% $\symb{\HH\KK}{\restrictup{\xi}{\HH}}\cdot ?b\in \lang{M^1_\HH}^\mC$.\\
% To have both $\symb{\HH\KK}{\restrictup{\xi}{\HH}}\cdot !a\in\lang{M^1_\HH}^\mC $ and 
% $\symb{\HH\KK}{\restrictup{\xi}{\HH}}\cdot ?b\in \lang{M^1_\HH}^\mC$
%does contradict the no mixed state condition of compatibility since,
% by ?!-determinism of $M^1_\HH$, ${q}_\HH$ is the unique state of $M^1_\HH$ 
%recognising the string $\symb{\HH\KK}{\restrictup{\xi}{\HH}}$. 
%
%\item 
%$w_{\HH\KK}\neq\varepsilon$.\\
%This subcase cannot occur, otherwise, by Lemma \ref{lem:getright}, we would get $s\lts{}$.
%
%\item 
%$w_{\KK\HH}\neq\varepsilon$.\\
%This subcase cannot occur, otherwise, by Lemma \ref{lem:getright}, we would get $s\lts{}$.
%\end{description}
%
%
%\item $q_\HH$ is final and $q_\KK$ is non final: \\
%By the no mixed state condition imposed by compatibility, we need to take into account two further subcases.\\
%\underline{$\diamond$} 
%{\em  All the transitions from $q_\KK$ in $\delta_\KK$ are of the form $(q_\KK,\tts\KK?\_,\_)$ with $\tts\neq\HH$}.\\
%As done in a previous case,  we can infer, from the definition of  $\gateway{\cdot}$, that also  all the transitions from $q_\KK$ in $\delta^2_\KK$ are of the form $(q_\KK,\tts\KK?\_,\_)$. Hence from $s\notlts{}$ we can  
%obtain that $\restrict{s}{2}\notlts{}$ as well, and then  (\ref{eq:snotprogr2}).\\
%\underline{$\diamond$} 
%{\em  All the transitions from $q_\KK$ in $\delta_\KK$ are of the form $(q_\KK,\HH\KK?\_,\_)$}.\\
%Since $\vec{q}_\HH$ is final, by Lemma \ref{lem:wempty}, $\vec{w}_{\KK\HH} = \varepsilon$. Thus, it remains to consider the following two subcases:
% \begin{description}
% \item
% ${w}_{\HH\KK}=\varepsilon$.\\
% This case cannot occur. Towards a contradiction, let us assume it to be possible.\\
%Let $\xi$ be a transition sequence leading to $s\in RS(S)$ from the initial state, in particular let $\xi$ be\\
%\centerline{
%$s_0\lts{\elle_1}s_1\lts{\elle_2} \ldots \lts{\elle_{n-1}}s_{n-1}\lts{\elle_n}s_n=s$
%}
% By Lemma \ref{lem:inlangs},
% $\symb{\HH\KK}{\restrictup{\xi}{\HH}}\in\lang{M^1_\HH}^\mC $ and $\symb{\KK\HH}{\restrictup{\xi}{\KK}}\in\lang{M^2_\KK}^\mC$. \\
% Moreover, by Corollary \ref{lem:prefix}(\ref{lem:prefixcor-iii}),
%  $\Dual{\symb{\HH\KK}{\restrictup{\xi}{\HH}}}=\symb{\KK\HH}{\restrictup{\xi}{\KK}}$.
%  Since all the transitions from $q_\KK$ in $\delta_\KK$ are of the form $(q_\KK,\HH\KK?\_,\_)$,
%we can infer, by definition of $\gateway{\cdot}$, that  $\symb{\KK\HH}{\restrictup{\xi}{\KK}}\cdot !a\in\lang{M^2_\KK}^\mC$ for a certain $a$.
%
%We have then that $\Dual{\symb{\HH\KK}{\restrictup{\xi}{\HH}}\cdot ?a}\in\lang{M^2_\KK}^\mC$
%and hence, since by compatibility $\Dual{\lang{M^1_\HH}^\mC} = \lang{M^2_\KK}^\mC$, also that
% $\symb{\HH\KK}{\restrictup{\xi}{\HH}}\cdot ?a\in \lang{M^1_\HH}^\mC$.\\
% To have both $\symb{\HH\KK}{\restrictup{\xi}{\HH}}\cdot ?a\in\lang{M^1_\HH}^\mC $ and 
% $\symb{\HH\KK}{\restrictup{\xi}{\HH}}\in \lang{M^1_\HH}^\mC$
%does contradict $q_\HH$ to be final, since,
% by ?!-determinism of $M^1_\HH$, ${q}_\HH$ is the unique state of $M^1_\HH$ 
%recognising the string $\symb{\HH\KK}{\restrictup{\xi}{\HH}}$. 
% \item
% $w_{\HH\KK}\neq\varepsilon$.\\
%Then, by Lemma \ref{lem:getright}, we get $s \lts{}$. Contradiction to the assumption of no progress in $s$.
%\end{description}
%
%
%\item $q_\KK$ is final and $q_\HH$ is non final: \\
%The same argument of the previous case does apply.
%\end{description}
%\end{proof}











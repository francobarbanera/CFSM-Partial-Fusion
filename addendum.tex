

\begin{remark}
\label{rem:dircomm}
{\em
As noted in \cite[Sect.5.3]{BdLH19}, trying to get rid of gateways (essentially by bypassing them
and by directly connecting the participants which indirectly communicate via the gateways)
would have several drawbacks: not only the need of renaming the channels of the directly connected participants, but in general also their complete redesign.
This would hence result in a non-conservative approach to multicomposition.
Of course this does not also imply investigations on direct communication composition
not to be worth pursuing, in particular for modular approaches to system development.
\brc The last sentence sounds strange. \erc.
\finex}
\end{remark}


We are currently considering a generalisation of the compatibility relation used 
in some of the above mentioned papers to guarantee safety for binary composition.
A possible and decidable candidate for that being the following one.\\
%  \begin{definition}[Multicompatibility]
{\bf Definition} (Multicompatibility) {\em  Let $\cp=(M_{\kk_i})_{i\in I}$ be a CS.
\vspace{-1mm}
  \begin{enumerate}[i)]
  \item
    Let   $\vec{p}= (p_{\KK_i})_{i\in I}$
  and $\vec{q}= (q_{\KK_i})_{i\in I}$  be  two control states of $\cp$,
   we say that $\vec{p}$ and $\vec{q}$ are {\em multicompatible} in $\cp$, 
   notation $\vec{p}\mc{\cp}\vec{q}$,
  whenever, for each  $j\in I$, the following holds\\
%  \begin{enumerate}[a)]
%  \item 
  \centerline{$ (p_{\KK_j},\_\,\KK_j?a,\tilde p)\in\delta_j \implies \exists v\in I\ s.t. \ j\neq v \qand (q_{\KK_v},\KK_v\,\_!,\tilde q)\in \delta_z \qand \vec{p'}\mc{\cp}\vec{q'}$}
  where $\vec{p'}$ and $\vec{q'}$  are s.t. $p'_{\KK_z} = \left\{\begin{array}{lll}
                                                                                p_{\KK_z} & \text{if} & z\neq j\\
                                                                                \tilde p   & \text{if} & z= j
                                                                                \end{array}
                                                                       \right.$
                                                                       and\
                                                  $q'_{\KK_z} = \left\{\begin{array}{lll}
                                                                                q_{\KK_z} & \text{if} & z\neq v\\
                                                                                \tilde q   & \text{if} & z= v
                                                                                \end{array}
                                                                       \right.$   
%   \item
%   for each $j\in I$
%  $$ (q_{\KK_j},\_\,\KK_j?a,\tilde q)\in\delta_j \implies \exists v\in I\ s.t. \ j\neq v \qand (p_{\KK_v},\KK_v\,\_!,\tilde p) \qand \vec{p}\mc{\cp}\vec{q'}$$
%  where$ \vec{q}$ is such that                        
%  \end{enumerate}
  \item The CFSMs  of the set $\Set{M_{\kk_i}}_{i\in I}$ are {\em multicompatible}
  whenever $\vec{p_0}\mc{\cp}\vec{p_0}$, where
  $\vec{p_0}$ is the control state of $\cp$ containing all initial states.
  \end{enumerate}
  }
  
  
  A weak notion of deadlock configuration, and hence a stronger notion of deadlock-freeness, has been suggested in~\cite{TG18}. 

To distinguish it from the notion above, we call it \emph{strong deadlock-freeness}, as done in \cite{BdLH19}.

  \begin{definition}[Strong deadlock-freeness]\label{def:strong-df}%\hfill\\
\label{def:wdf}
Let $S$ be a communicating system, and let $s= (\vec{q},\vec{w})$ be a configuration of $S$.
$s$ is a {\em weak deadlock configuration} of $S$ if $s\, \not\!\!\lts{}$ and either
\begin{enumerate}[a)]
\item $\exists \ttr\in\roles$ such that $q_\ttr \lts{\ttr\tts?\msg[a]} q'_\ttr$ , or
\item
\label{def:safeness-wnotem}
$\vec{w}\neq\vec{\varepsilon}$ 
\end{enumerate}
i.e.\ $s$ is stuck and at the same time either a machine is still waiting for a message
or there is a message waiting in a buffer which cannot be consumed (or both).
We say that $S$ is {\em strongly deadlock-free} whenever, for any $s\in \RS(S)$, $s$ is not a weak deadlock configuration.
\end{definition}

As it is natural to expect, a stuck configuration made of final states and empty buffers is not a weak deadlock configuration.
But any orphan-message configuration is a weak deadlock configuration.
Moreover, a stuck configuration made of final states and non empty buffers is a weak
deadlock configuration, but not a lock. Hence, whereas lock-freeness implies deadlock freeness, it does   not imply strong deadlock-freeness.


It can be shown that the strong deadlock-freeness property is equivalent to the properties of progress and orphan-message freeness. Moreover,
progress is just the same as  strong deadlock-freeness when alternative (\ref{def:safeness-wnotem}) of Definition
\ref{def:wdf}
%\ref{def:safeness}(\ref{def:safeness-i}) 
is omitted.


\begin{proposition}[\cite{BdLH19}]
\label{prop:freeprogorph}
Let $S$ be a communicating system.
\begin{enumerate}[i)]
\item
If $S$ is strongly deadlock-free, then $S$ is deadlock-free.
\item
There is no progress in a reachable configuration $s= (\vec{q},\vec{w}) \in \RS(S)$ if and only if\\
\centerline{
 1. $s\, \not\!\!\lts{}$ and\quad 2. $\exists \ttr\in\roles$ such that $q_\ttr \lts{\ttr\tts?\msg[a]} q'_\ttr$.
}
%\begin{enumerate}[1.]
%\item
% $s\, \not\!\!\lts{}$ and
%\item 
%\label{label:a}
%$\exists \ttr\in\roles$ such that $q_\ttr \lts{\ttr\tts?\msg[a]} q'_\ttr$.
%\end{enumerate}
\item
\label{prop:freeprogorph-ii}
$S$ is strongly deadlock-free if and only if $S$ is orphan-message free and satisfies the progress property.
\end{enumerate}
\end{proposition}

\begin{corollary}[Preservation of strong deadlock-freeness]%\hfill\\
\label{prop:dfPreservation}
Let $S = \MC(\Set{S_i}_{i\in I}, \cp)$ and let $\cp$ and each $S_i$ (for $i\in I$) be strongly deadlock-free.
 Moreover, let the interfaces of each $S_i$,
i.e.\ the CFSMs $M_{h_i}$ in $S_i$, have no mixed state.
Then also $S$  is strongly deadlock-free. 
%Let $S_1$ and $S_2$ be strongly deadlock-free.
%Then $S = {S_{1}} \connect{\HH}{\KK} {S_{2}}$ is strongly deadlock-free.
\end{corollary}
\begin{proof}
By  \cref{prop:freeprogorph}(\ref{prop:freeprogorph-ii}),  Corollary \ref{prop:nomPreservation} and  \cref{lem:restrRS}.
\end{proof}

  %\begin{proof}
%i) is obvious.\\
%ii)
%$\Rightarrow$: Assume that there is no progress in $s= (\vec{q},\vec{w})$. Then $s\, \not\!\!\lts{}$
%and there exists $\ttr\in\roles$ such that $q_\ttr$ is not final.
%Thus no sending action is possible in $q_\ttr$.
%Therefore there must exist a transition $q_\ttr \lts{\ttr\tts?a} q'_\ttr$ and thus (\ref{label:a}) holds.\\ 
%$\Leftarrow$: The converse direction is clear, since $s\, \not\!\!\lts{}$ and, by (\ref{label:a}),
%there exists $\ttr\in\roles$ and $q_\ttr \lts{\ttr\tts?a} q'_\ttr$. Thus $q_\ttr$ is not final.\\
%iii)
%$\Rightarrow$:
%Assume that there exists an orphan-message configuration $s \in \RS(S).$
%Since for all $\ttp\in\roles, q_\ttp$ is final, we have $s\, \not\!\!\lts{}$.
%Moreover, $\vec{w}\neq\vec{\varepsilon}$ since $s$ is an orphan-message configuration.
%Thus $s$ is a weak deadlock configuration.\
%Now assume that there exists a configuration $s \in \RS(S)$ with no progress.
%Then, using (ii), $s$ is a weak deadlock configuration.\\  
%$\Leftarrow$: Assume that there is a weak deadlock configuration $s \in \RS(S).$
%Then $s\, \not\!\!\lts{}$.
%If a) of Def. \ref{def:wdf} holds then, according to (ii), there is no progress in $s$.
%If a) does not hold, then $\forall \ttp\in\roles. ~ q_\ttp \text{ is final}$ and,
%since $s$ is a weak deadlock configuration, b) in Def.~\ref{def:strong-df} must hold, i.e.\
%$\vec{w}\neq \vec{\varepsilon}$. In this case $s$ is an orphan-message configuration. 
%\end{proof}

  
  
%  \end{definition}



\begin{figure*}[ht]\centering
\vspace{-16mm}
 \hspace{0mm}$
 \begin{tikzpicture}[node distance=1.5cm,scale=1]
        \node (square-h) [draw,minimum width=0.8cm,minimum height=0.8cm] {\large $\hh$};
        \node (square-k) [draw,minimum width=0.8cm,minimum height=0.8cm, right of = square-h, xshift=14mm] {\large $\kk$};
        \node (square-v)  [draw,minimum width=0.8cm,minimum height=0.8cm, below of = square-h, yshift=-14mm] {\large $\ttv$};
        \node (square-w)  [draw,minimum width=0.8cm,minimum height=0.8cm, below of = square-k, yshift=-14mm] {\large $\ttw$};
        \node (square-r)  [draw,minimum width=0.8cm,minimum height=0.8cm, below left of = square-k, yshift=-2mm] {\large $\ttr$};
        \node (square-q)  [draw,minimum width=0.8cm,minimum height=0.8cm, below of = square-h, yshift=-3mm, xshift=-4mm] {\large $\ttq$};
        \node (square-p)  [draw,minimum width=0.8cm,minimum height=0.8cm, right of = square-r,  yshift=-2mm] {\large $\ttp$};
        \draw[-stealth]  (square-w) to[out=75,in=-105]  node {$\msg[b]$} (square-p);
        \draw[-stealth]  (square-p) to[out=35,in=45]  node {$\msg[e]$} (square-r);
        \draw[-stealth]  (square-q) to[out=35,in=45]  node {$\msg[b]$} (square-v);
        \draw[-stealth]  (square-r) to[out=-35,in=-25]  node {$\msg[b]$} (square-v);
        \draw[-stealth]  (square-v) to[out=25,in=-85]  node {$\msg[a]$} (square-r);
        \draw[-stealth]  (square-w) to[out=145,in=-75]  node {$\msg[c]$} (square-h);
        \draw[-stealth]  (square-p) to[out=-145,in=115]  node {$\msg[a]$} (square-w);
        \draw[-stealth]  (square-h) to  node {$\msg[a]$} (square-q);
        \draw[-stealth]  (square-r) to  node {$\msg[f]$} (square-p);
        \draw[-stealth]  (square-r) to[out=145,in=165]  node {$\msg[a]$} (square-k);
        \draw[-stealth]  (square-k) to[out=145,in=55]  node {$\msg[b]$} (square-q);
 \end{tikzpicture}
 \qquad\qquad
  \begin{tikzpicture}[node distance=1.5cm,scale=1]
        \node (square-h) [draw,minimum width=0.8cm,minimum height=0.8cm] {\large $\hh$};
        \node (square-k) [draw,minimum width=0.8cm,minimum height=0.8cm, right of = square-h, xshift=14mm] {\large $\kk$};
        \node (square-v)  [draw,minimum width=0.8cm,minimum height=0.8cm, below of = square-h, yshift=-14mm] {\large $\ttv$};
        \node (square-w)  [draw,minimum width=0.8cm,minimum height=0.8cm, below of = square-k, yshift=-14mm] {\large $\ttw$};
        \node (square-r)  [draw,minimum width=0.8cm,minimum height=0.8cm, below left of = square-k, yshift=-2mm] {\large $\ttr$};
        \node (square-q)  [draw,minimum width=0.8cm,minimum height=0.8cm, below of = square-h, yshift=-3mm, xshift=-4mm] {\large $\ttq$};
        \node (square-p)  [draw,minimum width=0.8cm,minimum height=0.8cm, right of = square-r,  yshift=-2mm] {\large $\ttp$};
        \draw[-stealth]  (square-w) to[out=75,in=-105]  node {$\msg[b]$} (square-p);
        \draw[-stealth]  (square-p) to[out=35,in=45]  node {$\msg[e]$} (square-r);
        \draw[-stealth]  (square-q) to[out=35,in=45]  node {$\msg[b]$} (square-v);
        \draw[-stealth]  (square-r) to[out=-35,in=-25]  node {$\msg[b]$} (square-v);
        \draw[-stealth]  (square-v) to[out=25,in=-85]  node {$\msg[a]$} (square-r);
        \draw[-stealth]  (square-w) to[out=145,in=-75]  node {$\msg[c]$} (square-h);
        \draw[-stealth]  (square-p) to[out=-145,in=115]  node {$\msg[a]$} (square-w);
        \draw[-stealth]  (square-h) to  node {$\msg[a]$} (square-q);
        \draw[-stealth]  (square-r) to  node {$\msg[f]$} (square-p);
        \draw[-stealth]  (square-r) to[out=145,in=165]  node {$\msg[a]$} (square-k);
        \draw[-stealth]  (square-k) to[out=145,in=55]  node {$\msg[b]$} (square-q);
        %
                \node [state] (h-a) [above of = square-h, draw=none] {};
        \node [state] (h-c) [left of = square-h, draw=none] {};
        \draw [-stealth] (h-a) --  node {$\msg[a]$} (square-h);
        \draw [-stealth] (square-h) --  node {$\msg[c]$} (h-c);
                \node [state] (k-b) [above of = square-k, draw=none] {};
        \node [state] (k-a) [right of = square-k, draw=none] {};
        \draw [-stealth] (k-b) --  node {$\msg[b]$} (square-k);
        \draw [-stealth] (square-k) --  node {$\msg[a]$} (k-a);
                \node [state] (v-b) [left of = square-v, draw=none] {};
        \node [state] (v-a) [below of = square-v, draw=none] {};
        \draw [-stealth] (v-a) --  node {$\msg[a]$} (square-v);
        \draw [-stealth] (square-v) --  node {$\msg[b]$} (v-b);
                \node [state] (w-c) [right of = square-w, draw=none] {};
        \node [state] (w-a) [below of = square-w, draw=none] {};
        \node [state] (w-b) [above right of = square-w, draw=none] {};
        \draw [-stealth] (w-c) --  node {$\msg[c]$} (square-w);
        \draw [-stealth] (square-w) --  node {$\msg[a]$} (w-a);
        \draw [stealth-] (square-w) --  node {$\msg[b]$} (w-b);
        %
        \draw (0,0.4)[dotted,thick]  --  (-0.09,-0.4); % carrying a inside h  
        \draw (-0.4,0)[dotted,thick]  --  (0.1,-0.4); % carrying c inside h
        %
        \draw (2.9,0.4)[dotted,thick]  --  (2.5,0.25); % carrying b inside k
        \draw (2.5,0.1)[dotted,thick]  --  (3.3,0); % carrying a inside k
        %
        \draw (0.4,-2.7)[dotted,thick]  --  (0,-3.3); % carrying a inside v
        \draw (0.4,-2.5)[dotted,thick]  --  (-0.4,-2.9); % carrying q's b inside v
        \draw (0.4,-3.1)[dotted,thick]  --  (-0.4,-2.9); % carrying b inside v
        %
        \draw (2.7,-2.5)[dotted,thick]  --  (2.9,-3.3); % carrying a inside w
        \draw (2.5,-2.6)[dotted,thick]  --  (3.3,-2.9); % carrying c inside w
        \draw (3.01,-2.5)[dotted,thick]  to[out=-90,in=-90]  (3.3,-2.5); % carrying b inside w
 \end{tikzpicture}
  \vspace{-8mm}
 $   \caption{\label{fig:intinfra} Composition via ``interfacing infrastructure''.}
\end{figure*}
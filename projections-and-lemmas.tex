%!TEX root = Main-asynchCFSM-multicomp.tex

\section{Projections and Related Technical Results}
\label{sect:safetypreservation}


In this section we provide some assumptions, notations, definitions, and technical lemmas 
accessory to the subsequent proofs of property preservation.
% We begin with some general assumption and notations.



\medskip

\noindent
\textbf{General assumptions:}

\noindent
We assume given a
system $S$  obtained via orchestrated multicomposition:
$$S= (M_\ttp)_{\ttp\in\roles} = \MC(\Set{S_i}_{i\in I}, \cs)$$
where 
\begin{enumerate}[a)]
\item
$\Set{S_i}_{i\in I}$
is a set of communicating systems, composable with respect to the set of interfaces
$H = \Set{\hh_i}_{i\in I}$;
\item 
$S_i= (M^i_\ttp)_{\ttp\in \roles_i}$ for each $i\in I$;
\item
$\cs = (M^{\cs}_\ttu)_{\ttu\in \roles_{\cs}}$, with $\roles_{\cs} = \Set{\kk_i}_{i\in I}\cup\rolesorch_{\cs}$, is an orchestrated connection policy for $H$ (complying with a given connection model
$\cm$ for $H$,  though this is only of methodological relevance and irrelevant for the proofs);
\item
$\roles = (\bigcup_{i\in I} \roles_i)\cup \rolescsint$.
%where $\rolescsint = \roles_{\cs}\setminus\Set{\kk_i}_{i\in I}$\\
\end{enumerate}
 Notice that, by definition of orchestrated multicomposition, we have that, for each $i\in I$ and $\ttp\in\roles_i\setminus\Set{\HH_i}$, $M^i_\ttp = M_\ttp$. 
 Moreover, for each $\ttu\in\rolesorch_{\cs}$, we have that $M_\ttu = \widetilde{\widetilde{M^{\cs}_\ttu}}$. In the following, for the sake of readability and when no
ambiguity arises, we shall refer to $\widetilde{\widetilde{M^{\cs}_\ttu}}$ as simply $M_\ttu$ when
$\ttu\in\rolesorch_{\cs}$.
 

 We also recall that, for each $i\in I$, $M_{\hh_i} = M^i_{\HH_i}\gts M^{\cs}_{\KK_i}$, where we recall that $\kk_i=\dot{\hh_i}$.



%\smallskip
%\noindent
%We emphasise again that no condition is required for $\cm$, being our result indipendent 
%from the connection policy considered.
%We recall that, by the composability condition, 
% $\roles=\bigcup_{i\in I}\roles_i$ with  $\roles_n\cap\roles_m=\emptyset$ for
%each $n,m\in I$ such that $n\neq m$.

\medskip


\noindent
\textbf{Notations:} \\
%The  participants 
%of $S$ are $\roles=\bigcup_{i\in I}\roles_i$ with  $\roles_n\cap\roles_m=\emptyset$ for
%each $n,m\in I$ such that $n\neq m$.\\
The set of channels of $S$ is $C=\Set{\ttp\ttq \mid \ttp,\ttq\in \roles, \ttp\neq\ttq}$.\\
The set of transitions of $M_\ttp$ in $S$ is denoted by $\delta_\ttp$;\footnote{Note that,
for $\HH \in \Set{\hh_i}_{i\in I}$, we use the notation
$\delta_{\HH}$ for what is denoted by $\widehat\delta$ 
in the definition of $M_{\HH}\gts M_{\KK}$ (see \cref{def:gatewaycs}).}\\
Given $i\in I$:

the set of channels of $S_i$ is $C_i=\Set{\ttp\ttq \mid \ttp,\ttq\in \roles_i, \ttp\neq\ttq}$;
%$C=\bigcup_{i\in I} C_i\cup \Set{\ttp\ttq \mid \ttp,\ttq\in H\cup\rolescsint, \ttp\neq\ttq}$
%--$C=\bigcup_{i\in I} C_i\cup\Set{\HH_i\HH_j}_{i,j\in I, i\neq j}$. \\

the set of transitions of $M^i_\ttp$ in $S_i$ will be denoted by $\delta^i_\ttp$
 where $\delta^i_\ttp = \delta_\ttp$ for each $\ttp\in\roles_i\setminus\Set{\HH_i}$;\\
%The set of transitions of $M_{\KK_i}$ will be denoted by $\delta^\cs_{\KK_i}$. \\
Given $\ttu\in\roles_{\cs}$, the set of transitions of 
$M^{\cs}_{\ttu}$ in $\cs$ will be denoted by $\delta^\cs_{\ttu}$.\\

\noindent
Notice that, for each $i\in I$ and $\ttp\in\roles_i\setminus\Set{\HH_i}$, by definition of orchestrated multicomposition, $\delta^i_\ttp = \delta_\ttp$. 
 Moreover, for each $\ttu\in\rolesorch_{\cs}$, $\delta_\ttu = \widetilde{\widetilde{\delta^{\cs}_\ttu}}$ (see \cref{def:tildetildem}). For the sake of readability, and when no ambiguity arises,
in the following we shall refer to $\widetilde{\widetilde{\delta^{\cs}_\ttu}}$ as simply $\delta_\ttu$.



%\smallskip
%\noindent
%Notice that, for each $i\in I$, $\delta_\ttp = \delta^i_\ttp$ for each $\ttp\in\roles_i\setminus\Set{\HH_i}$.


%\subsection{Technical notions and results}

%A key notion for our main result is that of {\em  projection\/} of a reachable configuration of a composed system. 

%We shall prove that if a projection does not contain any fresh state
%introduced by gateway construction then it is a configuration in the corresponding system.


 
\begin{definition}[Configuration projections]\label{def:projectedconf}
Let $s= (\vec{q},\vec{w})\in \RS(S)$ where $\vec{q}=(q_\ttp)_{\ttp\in\roles}$
and $\vec{w} = (w_{\ttp\ttq})_{\ttp\ttq\in C}$.
For each $i\in I$, the projection $\restrict{s}{i}$ of $s$
to $S_i$ is defined by
$$\restrict{s}{i}=(\restrict{\vec{q}}{i},\restrict{\vec{w}}{i})$$
where $\restrict{\vec{q}}{i} = (q_\ttp)_{\ttp\in\roles_i}$ and 
$\restrict{\vec{w}}{i} =  (w_{\ttp\ttq})_{\ttp\ttq\in C_{i}}$.

%\bfr
%[[ALTERNATIVE]] Let $s= (\vec{q},\vec{w})\in \RS(S)$ where $\vec{q}=(q_\ttp)_{\ttp\in\roles}$
%and $\vec{w} = (w_{\ttp\ttq})_{\ttp\ttq\in C}$.
%For each $i\in I$, the projection $\restrict{s}{i}$ of $s$
%to $S_i$ is defined by
%$$\restrict{s}{i}=(\restrict{\vec{q}}{i},\restrict{\vec{w}}{i})$$
%where  $\restrict{\vec{q}}{i} = (p_{\ttp})_{i\in  \roles_{i}}$ is the configuration of $S_i$
%such that,  for each $\ttp\in\roles_i$,
%$$p_{\ttp} = \left\{\begin{array}{l@{\qquad}l}
%                                                      q_\ttp & \text{if } \ttp\neq\hh_i  \text{ or }  q_\ttp\not\in\widehat{Q_{\hh_i}}\\[1mm]
%                                                      p' & \text{if } p'\lts{\ttr\,\hh_i?\_}q_{\ttp} \text{ and } \ttr\not\in\roles_i\\ 
%                                                      
%                                                      p' & \text{if }  q_{\ttp}\lts{\hh_i\ttr!\_}p' \text{ and } \ttr\not\in\roles_i\\
%                              \end{array}\right.
%$$ 
%
%and 
%$\restrict{\vec{w}}{i} =  (w_{\ttp\ttq})_{\ttp\ttq\in C_{i}}$.
%\efr

\noindent
The projection $\restrict{s}{\cs}$ of $s= (\vec{q},\vec{w})$ on $\cs$  is defined
if ${q}_{\HH_i}\not\in \widehat Q_{\HH_i}$  for each $i \in I$ and
then
$$\restrict{s}{\cs}=(\restrict{\vec{q}}{\cs},\restrict{\vec{w}}{\cs})$$
where\\
 - $\restrict{\vec{q}}{\cs} = (p_{\ttu})_{i\in  \roles_{\cs} }$ is the configuration of $\cs$ such that,  $\forall i \in   I.\,p_{\kk_i} = \dot{q_{\HH_i}}$  and  $\forall \ttu \in   \rolescsint.\,p_{\pu} = q_{\pu}$;\\
- $\restrict{\vec{w}}{\cs} =  (w'_{\ttu\ttv})_{\ttu,\ttv\in \roles_{\cs},\ttu\neq\ttv}$
%(w'_{\ttp\ttq})_{\ttp,\ttq\in \Set{\KK_i}_{i\in I},\ttp\neq\ttq}$  
is such that, for each pair $\ttu,\ttv\in \roles_{\cs}$ with $\ttu\neq\ttv$, 
 $w'_{\ttu\ttv} = w_{\tilde{\tilde{\ttu}}\tilde{\tilde{\ttv}}}$.
%$w'_{\ttp\ttq} = w_{\ttp\ttq}$.
% ${i,j\in I}$ with $i\neq j$, $w'_{\KK_i\KK_j} = w_{\HH_i\HH_j}$.
%We also define 
%$$\restrict{s}{\cs}=(\restrict{\vec{q}}{\cs},\restrict{\vec{w}}{\cs})$$
%where
%\bfc
%Since we are using 
%\cref{lem:nohatrestrict} in its present version, I guess that
%the definition of $\restrict{\vec{s}}{\cs}$ does not need to be as complex as it is now, and
%it could be made as simple as the one of $\restrict{s}{i}$.
%Namely
%$$\restrict{\vec{s}}{\cs} = (\restrict{\vec{q}}{\cs},\restrict{\vec{w}}{\cs})$$
%where $\restrict{\vec{q}}{\cs} = (p_{\kk_i})_{i\in I}$ is such that, for each $i\in I$, \\
%${\qquad}$ 
%$p_{\KK_i} = \left\{ \begin{array}{ll}
%                                  \\[-6mm]
%                                  \dot{q_{\HH_i}} &  \text{ if }  {q}_{\HH_i}\not\in \widehat Q_{\HH_i}\\
%                                  {q}_{\HH_i} &  \text{ if }  {q}_{\HH_i}\in \widehat Q_{\HH_i}\\
%                                 \end{array}
%                       \right.$\\
%                        and where
%$\restrict{\vec{w}}{\cs} =  (w_{\kk_i\kk_j})_{i,j\in I, i\neq j}$ 
%is such that, for each $i,j\in I, i\neq j$\\
%${\qquad}$
% $w_{\kk_i\kk_j}= w_{\hh_i\hh_j}$.\\
%\efc
%
%\brc
%I don't understand the case ``${q}_{\HH_i}  \text{ if }  {q}_{\HH_i}\in \widehat Q_{\HH_i}$''.
%Is this a typo and you mean $\dot{q}_{\HH_i}  \text{ if }  {q}_{\HH_i}\in \widehat Q_{\HH_i}$?\\
%Anyway I understand that $\cs$-projection for the case ${q}_{\HH_i}\in \widehat Q_{\HH_i}$
%is irrelevant in the sequel, since when doing projections to $\cs$ we always assume,
%I guess, that 
%${q}_{\HH_i}\not\in \widehat Q_{\HH_i}$ for all $i \in I$. Let us check this.
%If this is true then $\cs$-projections for the case  ${q}_{\HH_i}\in \widehat Q_{\HH_i}$ are
%meaningless and could be misunderstood by the reader.
%Then I would propose to define projections to $\cs$ only for configurations $s$
%with ${q}_{\HH_i}\not\in \widehat Q_{\HH_i}$ for all $i \in I$. The definition would then be very simple:
%\erc

%\noindent
%We also define the projection of $s= (\vec{q},\vec{w})$ to $\cs$ in case,  for each $i \in I$,
%${q}_{\HH_i}\not\in \widehat Q_{\HH_i}$. Namely,
%$$\restrict{s}{\cs}=(\restrict{\vec{q}}{\cs},\restrict{\vec{w}}{\cs})$$
%where $\restrict{\vec{q}}{\cs} = (p_{\kk_i})_{i\in I}$ is such that, for each $i \in I$,
% $p_{\kk_i} = \dot{q_{\HH_i}}$ 
%%  \brc
%% BTW: I think the notation $\dot{q}_{\HH_i}$ is not correct and should be
%% $\dot{q}_{\dot{\HH}_i}$.
%%\erc
%%\bfc
%%The dot is over the whole $q_{\HH_i}$. 
%%Writing $\dot{q}_{\dot{\HH}_i}$ would be the same as writing $\dot{q}_{\kk_i}$, which is not
%%what we mean.\\
%%\efc
%%\brc Okay. It is indeed subtle. Also in~\cref{def:cps}, $\dot{q}_0$ should be $\dot{q_0}$
%%and $\dot{q}'$ should be $\dot{q'}$.
%%I did the changes. And I think in all connection policy examples we must change
%%the state names $0, 1$ etc.\ to $\dot{0}, \dot{1}$?  \erc
%
%\noindent
%and where
%$\restrict{\vec{w}}{\cs} =  (w'_{\ttp\ttq})_{\ttp,\ttq\in \Set{\KK_i}_{i\in I},\ttp\neq\ttq}$  
%is such that, for each pair ${i,j\in I}$ with $i\neq j$,
%$w'_{\KK_i\KK_j} = w_{\HH_i\HH_j}$.
 


%%%%%%%%--OLD DEFINITION of projection on K--%%%%%%%%%%%%%%
%$\restrict{\vec{q}}{\cs} = (p_{\KK_i})_{i\in I}$ is such that, for each $i\in I$,\\
%
%
%${\qquad}$ 
%$p_{\KK_i} = \left\{ \begin{array}{ll}
%                                  \\[-6mm]
%                                  \dot{q}_{\HH_i} &  \text{ if }  {q}_{\HH_i}\not\in \widehat Q_{\HH_i}\\
%                                  \dot{q'} &  \text{ if }  {q}_{\HH_i}\in \widehat Q_{\HH_i} \text{ and } \text{\em (*)}\\
%                                  \dot{q''}           &  \text{ if }  {q}_{\HH_i}\in \widehat Q_{\HH_i} \text{ and } \text{\em (**)}
%                                 \end{array}
%                       \right.$\\
%                       
%${\qquad}$ \text{\em (*)} if, in  $\widehat \delta_{\HH_i}$,  the outgoing transition from ${q}_{\HH_i}$ is of the form 
%$({q}_{\HH_i}, {\HH_i}\tts!\_,q')$ with $\tts\in\roles_i$; 
%
%${\qquad\;}$\text{\em (**)}  if,  in  $\widehat \delta_{\HH_i}$,  the incoming transition to ${q}_{\HH_i}$ is of the form 
%$(q'', \tts{\HH_i}?\_,{q}_{\HH_i}) $ with $\tts\in\roles_i$;\\

%${\qquad}$ where $\widehat Q$  are the additional states and  $\widehat \delta$  are the  transitions of $M_{\HH_j}\gts M_{\KK_j}$ (see \cref{def:gatewaymc}); \\
%and 
%
%$\restrict{\vec{w}}{\cs} =  (w'_{\ttp\ttq})_{\ttp,\ttq\in \Set{\KK_i}_{i\in I},\ttp\neq\ttq}$ is such that, for each ${j,v\in I}$ with $j\neq v$, 
%\  $w'_{\KK_j\KK_v} = w_{\HH_j\HH_v}$.
\end{definition}

%\medskip
%\noindent
%Notice that $\restrict{s}{i}$ is not necessarily a configuration of $S_i$, because of the possible presence of the additional states introduced by the gateways construction.
%%  
%-----------------------------------------------------------------------------
% COMMENT TO BE TAKEN INTO ACCOUNT IN THE FUTURE  |
%-----------------------------------------------------------------------------
%\bmc better to add an example \emc \bfc A meaningful example could be obtained by completely describing 
%the multicomposition of the systems of \cref{eq:JK} (i.e. by showing also the the gateways for
%$\HH_2$, $\HH_3$ and $\HH_4$), taking a reachable configuration (represented by coloring
%the corresponding states in the multicomposition system) and showing the various projections. I am not sure, however, it to be worth doing in the present version of the paper. \efc 
%
%
%Instead, $\restrict{s}{\cp}$, when defined, is a configuration of $\cs$ by construction.\\


%\bfr It is not difficult to check that the projections above are well defined. \efr  
 Notice that a projection $\restrict{s}{i}$ is not necessarily a configuration of $S_i$, because of the possible presence of the additional states introduced by the gateway construction.
 
 
 \begin{example}[A reachable configuration and two projections of it] {\em
We consider the following reachable configuration of the orchestrated composition illustrated in 
\cref{fig:orchsyscomp}
$$s=\big((4_{\ttp},2_{\hh_1},0_{\hh_4},0_{\tts},0_{\ttb},0_{\ttd},0_{\ttq},0_{\hh_2},1_{\hh_3},1_{\ttr}),\, \vec{w}\big)
$$
where \quad $w_{\ttp\hh_1}=\langle\msg[nc]\rangle$ \quad $w_{\hh_1\ttb}=\langle\msg[react]\rangle$ \quad $w_{c}=\varepsilon\, (\forall c\not\in\Set{\ttp\hh_1,\hh_1\ttb})$.\\
This configuration is graphically depicted on top of \cref{fig:orchsyscompproj}, where, for the sake of readability,
empty channels are not indicated.


 
\begin{figure*}[h!t]
$$
\begin{array}{c@{\hspace{-2mm}}c}
     \begin{array}{cc} %%% SYSTEM 1
            \begin{tikzpicture}[mycfsm]
		  % 
		  \node[state, initial, initial where = above, initial text={$\ptp[\ttp]$}] (0) {$0$};
		  \node[state] (1) [below right of=0]   {$1$};
		  \node[state] (2) [below left of=1]   {$2$};
		  \node[draw=none,fill=none] (wph1) [below  of=2,yshift=12mm]   {\scalebox{1.3}{$w_{\ttp\hh_1}=\langle\msg[nc]\rangle$}};
		  \node[state] (3) [above left of=2,xshift=-4mm]   {$3$};
		  \node[state] (4) [above left of=2,xshift=4mm, fill=black]   {$4$};
		  % 
		  \path
		  (0) edge [bend left] node[above] {$\ain[h_1][p][][start]$} (1)
		  (1) edge [bend left]  node[below] {$\aout[p][h_1][][react]$} (2)
		  (2) edge [bend left]  node[below] {$\aout[p][h_1][][rc]$} (3)
		  (3) edge[bend left] node[above] {$\ain[h_1][p][][img]$} (0)
		  (2) edge [bend left]  node[above] {$\aout[p][h_1][][nc]$} (4)
		  (4) edge [bend left]  node[below] {$\ain[h_{1}][p][][img]$} (0)
		  ;
		\end{tikzpicture}
&
           \begin{tikzpicture}[mycfsm]
		  % 
		  \node[state] (0) {$0$};
		  \node[draw=none,fill=none] (start) [above  left  = 0.3cm  of zero]{$\hh_1$};
		  \node[draw=none,fill=none] (wh1b) [below  of=0,yshift=-8mm,xshift=10mm]   {\scalebox{1.3}{$w_{\hh_1\ttb}=\langle\msg[react]\rangle$}};
		  \node[state] (hat0) [above right of=0,yshift=-4mm]   {$\widehat 0$};
		  \node[state] (1) [right of=hat0]   {$1$};
		  \node[state] (hat1) [right of=1]   {$\widehat 1$};
		  \node[state] (2) [right of=0, xshift=48mm, fill=black]   {$2$};
		  \node[state] (hat2) [below of=hat1, yshift=2mm]   {$\widehat 2$};
		  \node[state] (hat2p) [below of=hat1, yshift=-6mm]   {$\widehat{2}'$};
		  \node[state] (3) [below of=1,yshift=-6mm]   {$3$};
		  \node[state] (4) [below of=1, yshift=2mm  ] {$4$};
		  \node[state] (hat3) [below of=hat0,yshift=-6mm]   {$\widehat 3$};
		  \node[state] (hat4) [below of=hat0, yshift=2mm  ] {$\widehat 4$};
		  % 
		  \path
		  (start) edge  node {} (0)
		  (0) edge [bend left] node[above] {$\ain[h_3][h_1][][start]$} (hat0)
		  (hat0) edge node[above] {$\aout[h_1][p][][start]$} (1)
		  (1) edge   node[above] {$\ain[p][h_1][][react]$} (hat1)
		  (hat1) edge  [bend left] node[above] {$\aout[h_1][b][][react]$} (2)
		  (2) edge [bend left]  node[below] {$\ain[p][h_1][][rc]$} (hat2p)
		  (hat2p) edge   node[above] {$\aout[h_1][h_2][][rc]$} (3)
		  (3) edge node[above] {$\ain[h_2][h_1][][img]$} (hat3)
		  (hat3) edge[bend left] node[below] {$\aout[h_1][p][][img]$} (0)
		  (2) edge [bend left]  node[above] {$\ain[p][h_1][][nc]$} (hat2)
		  (hat2) edge   node[above] {$\aout[h_1][h_2][][nc]$} (4)
		  (4) edge node[above] {$\ain[h_2][h_1][][img]$} (hat4)
		  (hat4) edge[bend left] node[above] {$\aout[h_1][p][][img]$} (0)
		  ;
		\end{tikzpicture}
    \end{array}
&
    \begin{array}{cc} %%% SYSTEM 4
          \begin{tikzpicture}[mycfsm]
  \node[state, fill=black]           (0)                        {$0$};
  \node[state]           (hat0)          [below right of=0, yshift=5mm]              {$\widehat{0}$};
  \node[draw=none,fill=none] (none) [below  of=hat0,yshift=8mm]   {};
   \node[draw=none,fill=none] (start) [above left = 0.3cm  of 0]{$\HH_4$};
  \node[state]            (1) [above right of=hat0, yshift=-5mm] {$1$};
  \node[state]           (hat0') [above right of=0, yshift=-5mm] {$\widehat{0}'$};
  %\node[state]           (2) [right of=hat0'] {$2$};

   \path  (start) edge node {} (0) 
            (0)  edge    [bend left]               node [above] {${\ttd\hh_4}?{\msg[react]}$} (hat0')
             (hat0')  edge   [bend left]          node [above]  {${\hh_4\tts}!{\msg[react]}$} (1)
             (1)  edge   [bend left]      node [below] {${\tts\HH_4}?{\msg[pars]}$} (hat0)
             (hat0)  edge   [bend left]            node [below] {${\hh_4\hh_2}!{\msg[pars]}$} (0);                 
             \end{tikzpicture}
             &
                   \begin{tikzpicture}[mycfsm]
      %\tikzstyle{every state}=[cnode]
      %\tikzstyle{every edge}=[carrow]
      % 
      \node[state, fill=black] (zero) {$0$};
      \node[state] (one) [below of=zero]   {$1$}; 
      \node[draw=none,fill=none] (none) [below  of=one,yshift=8mm]   {};
            \node[draw=none,fill=none] (start) [above  = 0.3cm  of zero]{$\tts$};
      % 
      \path
      (start) edge node {} (zero) 
      (zero) edge[bend left]  node[above] {$\ain[h_4][s][][react]$} (one)
      (one) edge[bend left] node[above] {$\aout[s][h_4][][pars]$} (zero)
      ;
      \end{tikzpicture}           
      \end{array}
\\[-12mm]
\multicolumn{2}{@{\hspace{35mm}}c}{ %%% ORCHESTRATING SYSTEM
\begin{tikzpicture}[mycfsm]
		  % 
		  \node[state, initial, initial where = left, initial text={$\ptp[b]$}, fill=black] (zero) {$0$};
		  % 
		  \path
		  (zero) edge [loop below,looseness=40] node[below] {$\ain[h_1][b][][react]$} (zero)
		  ;
		\end{tikzpicture}
  \qquad\quad
      \begin{tikzpicture}[mycfsm]
      %\tikzstyle{every edge}=[carrow]
      % 
      \node[state] (zero) [fill=black]{$0$};
      \node[state] (one) [below right of=zero, xshift=-3mm]   {$1$};
      \node[state] (two) [below left  of=zero, xshift=3mm]   {$2$};
      \node[draw=none,fill=none] (start) [above  left= 0.3cm  of zero]{$\ptp[d]$};
      % 
      \path
      (start) edge node {} (zero) 
      (zero) edge[bend left] node[above] {$\ain[h_2][d][][react]$} (one)
      (one) edge[bend left = 40] node[below] {$\aout[d][h_4][][react]$} (two)
      (two) edge[bend left] node[above] {$\aout[d][h_3][][react]$} (zero)
      ;
  \end{tikzpicture}
    }
\\[-6mm]
\begin{array}{cc} %%% SYSTEM 2
\begin{tikzpicture}[mycfsm]
		  % 
		  \node[state, initial, initial where = above, initial text={$\ptp[\ttq]$}, fill=black] (0) {$0$};
		  \node[state] (1) [below right of=0]   {$1$};
		  \node[state] (2) [below left of=1]   {$2$};
		  \node[state] (3) [above left of=2,xshift=-4mm]   {$3$};
		  \node[state] (4) [above left of=2,xshift=4mm]   {$4$};
		  % 
		  \path
		  (0) edge [bend left] node[above] {$\aout[q][h_2][][react]$} (1)
		  (1) edge [bend left]  node[below] {$\ain[h_2][q][][pars]$} (2)
		  (2) edge [bend left]  node[below] {$\ain[h_2][q][][rc]$} (3)
		  (3) edge[bend left] node[above] {$\aout[q][h_2][][img]$} (0)
		  (2) edge [bend left]  node[above] {$\ain[h_2][q][][nc]$} (4)
		  (4) edge [bend left]  node[below] {$\aout[q][h_2][][img]$} (0)
		  (2) edge  node[below] {$\ain[h_2][q][][reset]$} (0)
		  ;
		\end{tikzpicture}
&
\begin{tikzpicture}[mycfsm]
		  % 
		   \node[state]           (0)     [fill=black]                   {$0$};
		  \node[state] (hat0) [above right of=0,yshift=-4mm]   {$\widehat 0$};
		  \node[draw=none,fill=none] (start) [above  left  = 0.3cm  of zero]{$\hh_2$};
		  \node[state] (1) [right of=hat0]   {$1$};
		  \node[state] (hat1) [right of=1]   {$\widehat 1$};
		  \node[state] (2) [right of=0, xshift=48mm]   {$2$};
		  \node[state] (hat2) [below of=hat1, yshift=2mm]   {$\widehat 2$};
		  \node[state] (hat2p) [below of=hat1, yshift=-6mm]   {$\widehat{2}'$};
		  \node[state] (3) [below of=1,yshift=-6mm]   {$3$};
		  \node[state] (4) [below of=1, yshift=2mm  ] {$4$};
		  \node[state] (hat3) [below of=hat0,yshift=-6mm]   {$\widehat 3$};
		  \node[state] (hat4) [below of=hat0, yshift=2mm  ] {$\widehat 4$};
		  % 
		  \path
		   (start) edge  node {} (0)
		  (0) edge [bend left] node[above] {$\ain[q][h_2][][react]$} (hat0)
		  (hat0) edge node[above] {$\aout[h_2][d][][react]$} (1)
		  (1) edge   node[above] {$\ain[h_4][h_2][][pars]$} (hat1)
		  (hat1) edge  [bend left] node[above] {$\aout[h_2][q][][pars]$} (2)
		  (2) edge [bend left]  node[below] {$\ain[h_1][h_2][][rc]$} (hat2p)
		  (hat2p) edge   node[above] {$\aout[h_2][q][][rc]$} (3)
		  (3) edge node[above] {$\ain[q][h_2][][img]$} (hat3)
		  (hat3) edge[bend left] node[below] {$\aout[h_2][h_1][][img]$} (0)
		  (2) edge [bend left]  node[above] {$\ain[h_1][h_2][][nc]$} (hat2)
		  (hat2) edge   node[above] {$\aout[h_2][q][][nc]$} (4)
		  (4) edge node[above] {$\ain[q][h_2][][img]$} (hat4)
		  (hat4) edge[bend left] node[above] {$\aout[h_2][h_1][][img]$} (0)
		  ;
		\end{tikzpicture}
      \end{array}
&
   \begin{array}{c@{\hspace{-2mm}}c} %%% SYSTEM 3
             \begin{tikzpicture}[mycfsm]
  \node[state]           (0)                        {$0$};
  \node[state]           (hat0)          [below right of=0, yshift=5mm]              {$\widehat{0}$};
   \node[draw=none,fill=none] (start) [above left = 0.3cm  of 0]{$\HH_3$};
  \node[state]            (1) [above right of=hat0, yshift=-5mm, fill=black] {$1$};
  \node[state]           (hat0') [above right of=0, yshift=-5mm] {$\widehat{0}'$};
  %\node[state]           (2) [right of=hat0'] {$2$};

   \path  (start) edge node {} (0) 
            (0)  edge    [bend left]               node [above] {${\ttr\hh_3}?{\msg[start]}$} (hat0')
             (hat0')  edge   [bend left]          node [above]  {${\hh_3\hh_1}!{\msg[start]}$} (1)
             (1)  edge   [bend left]      node [below] {${\ttd\HH_3}?{\msg[react]}$} (hat0)
             (hat0)  edge   [bend left]            node [below] {${\hh_3\ttr}!{\msg[react]}$} (0);       \end{tikzpicture}
    &
    \begin{array}{c}
      \begin{tikzpicture}[mycfsm]
      %\tikzstyle{every edge}=[carrow]
      % 
      \node[state] (zero) {$0$};
      \node[state] (one) [below right of=zero, xshift=-6mm, fill=black]   {$1$};
      \node[state] (two) [below left  of=zero, xshift=6mm]   {$2$};
      \node[draw=none,fill=none] (start) [above left = 0.3cm  of zero]{$\ttr$};
      % 
      \path
      (start) edge node {} (zero) 
      (zero) edge[bend left] node[above] {$\aout[r][h_3][][start]$} (one)
      (one) edge[bend left] node[below] {$\ain[h_3][r][][react]$} (two)
      (two) edge[bend left] node[above] {$\aout[r][r'][][greet]$} (zero)
      ;
  \end{tikzpicture}
   \\
      \begin{tikzpicture}[mycfsm]
      %\tikzstyle{every state}=[cnode]
      %\tikzstyle{every edge}=[carrow]
      % 
      \node[state] (zero) {$0$};
      %\node[state] (one) [below of=zero]   {$1$};
      \node[draw=none,fill=none] (start) [above left = 0.3cm  of zero]{$\ttr'$};
      % 
      \path
      (start) edge node {} (zero) 
      (zero) edge[loop right,looseness=40]  node[above] {$\ain[r][\ttr'][][greet]$} (zero)
      %(zero) edge[bend right] node[below] {$\aout[h_4][r][][react]$} (one)
      ;
  \end{tikzpicture}
  \end{array}
    \end{array}
\end{array}
$$
\hrule
\vspace{2mm}
$$
\begin{array}{c} 
  \begin{tikzpicture}[mycfsm]
		  % 
		  \node[state, initial, initial where = above, initial text={$\ptp[\ttp]$}] (0) {$0$};
		  \node[state] (1) [below right of=0]   {$1$};
		  \node[state] (2) [below left of=1]   {$2$};
		  \node[draw=none,fill=none] (wph1) [below  of=1,yshift=8mm,xshift=12mm]   {\scalebox{1.3}{$w_{\ttp\hh_1}=\langle\msg[nc]\rangle$}};
		  \node[state] (3) [above left of=2,xshift=-4mm]   {$3$};
		  \node[state] (4) [above left of=2,xshift=4mm, fill=black]   {$4$};
		  % 
		  \path
		  (0) edge [bend left] node[above] {$\ain[h_1][p][][start]$} (1)
		  (1) edge [bend left]  node[below] {$\aout[p][h_1][][react]$} (2)
		  (2) edge [bend left]  node[below] {$\aout[p][h_1][][rc]$} (3)
		  (3) edge[bend left] node[above] {$\ain[h_1][p][][img]$} (0)
		  (2) edge [bend left]  node[above] {$\aout[p][h_1][][nc]$} (4)
		  (4) edge [bend left]  node[below] {$\ain[h_{1}][p][][img]$} (0)
		  ;
		\end{tikzpicture}
		\begin{tikzpicture}[mycfsm]
		  % 
		  \node[state, initial, initial where = above, initial text={$\ptp[\HH_1]$}] (0) {$0$};
		  \node[state] (1) [below right of=0]   {$1$};
		  \node[state] (2) [below left of=1, fill=black]   {$2$};
		  \node[state] (3) [above left of=2,xshift=-4mm]   {$3$};
		  \node[state] (4) [above left of=2,xshift=4mm]   {$4$};
		  % 
		  \path
		  (0) edge [bend left] node[above] {$\aout[h_1][p][][start]$} (1)
		  (1) edge [bend left]  node[below] {$\ain[p][h_1][][react]$} (2)
		  (2) edge [bend left]  node[below] {$\ain[p][h_1][][rc]$} (3)
		  (3) edge[bend left] node[above] {$\aout[h_1][p][][img]$} (0)
		  (2) edge [bend left]  node[above] {$\ain[p][h_1][][nc]$} (4)
		  (4) edge [bend left]  node[below] {$\aout[h_1][p][][img]$} (0)
		  ;
		\end{tikzpicture}
 \end{array}
 $$
 \hrule
 \vspace{-6mm}
 $$
 \begin{array}{c}
    \begin{array}{c@{\hspace{5mm}}c@{\hspace{5mm}}c@{\hspace{5mm}}c}
    \begin{array}{c}
    \\[4mm]
		\begin{tikzpicture}[mycfsm]
		  % 
		  \node[state, initial, initial where = above, initial text={$\ptp[\KK_1]$}] (0) {$\dot 0$};
		  \node[state] (1) [below right of=0]   {$\dot 1$};
		  \node[state] (2) [below left of=1, fill=black]   {$\dot 2$};
		  \node[draw=none,fill=none] (wk1b) [below  of=2,yshift=8mm]   {\scalebox{1.3}{$w_{\kk_1\ttb}=\langle\msg[react]\rangle$}};
		  \node[state] (3) [above left of=2,xshift=-4mm]   {$\dot 3$};
		  \node[state] (4) [above left of=2,xshift=4mm]   {$\dot 4$};
		  % 
		  \path
		  (0) edge [bend left] node[above] {$\ain[k_3][k_1][][start]$} (1)
		  (1) edge [bend left]  node[below] {$\aout[k_1][b][][react]$} (2)
		  (2) edge [bend left]  node[below] {$\aout[k_1][k_2][][rc]$} (3)
		  (3) edge[bend left] node[above] {$\ain[k_2][k_1][][img]$} (0)
		  (2) edge [bend left]  node[above] {$\aout[k_1][k_2][][nc]$} (4)
		  (4) edge [bend left]  node[below] {$\ain[k_2][ k_1][][img]$} (0)
		  ;
		\end{tikzpicture}
 \end{array}
&
 \begin{array}{c}
		\begin{tikzpicture}[mycfsm]
		  % 
		  \node[state, initial, initial where = above, initial text={$\ptp[\KK_2]$}, fill=black] (0) {$\dot 0$};
		  \node[state] (1) [below right of=0]   {$\dot 1$};
		  \node[state] (2) [below left of=1]   {$\dot 2$};
		  \node[state] (3) [above left of=2,xshift=-4mm]   {$\dot 3$};
		  \node[state] (4) [above left of=2,xshift=4mm]   {$\dot 4$};
		  % 
		  \path
		  (0) edge [bend left] node[above] {$\aout[k_2][d][][react]$} (1)
		  (1) edge [bend left]  node[below] {$\ain[k_4][k_2][][pars]$} (2)
		  (2) edge [bend left]  node[below] {$\ain[k_1][k_2][][rc]$} (3)
		  (3) edge[bend left] node[above] {$\aout[k_2][k_1][][img]$} (0)
		  (2) edge [bend left]  node[above] {$\ain[k_1][k_2][][nc]$} (4)
		  (4) edge [bend left]  node[below] {$\aout[k_2][k_1][][img]$} (0)
		  ;
		\end{tikzpicture}
 \end{array}
&
\begin{array}{c}
   \begin{tikzpicture}[mycfsm]
      %\tikzstyle{every state}=[cnode]
      %\tikzstyle{every edge}=[carrow]
      % 
      \node[state] (zero) {$\dot 0$};
      \node[state] (one) [below of=zero, fill=black]   {$\dot 1$};
      \node[draw=none,fill=none] (start) [above  = 0.3cm  of zero]{$\KK_3$};
      % 
      \path
      (start) edge node {} (zero) 
      (zero) edge[bend left]  node[above] {$\aout[k_3][k_1][][start]$} (one)
      (one) edge[bend left] node[above] {$\ain[d][k_3][][react]$} (zero)
      ;
  \end{tikzpicture}
 \end{array}
 &
  \begin{array}{c}
     \begin{tikzpicture}[mycfsm]
      %\tikzstyle{every edge}=[carrow]
      % 
      \node[state] (zero) [fill=black] {$\dot 0$};
      \node[state] (one) [below of=zero]   {$\dot 1$};
      \node[draw=none,fill=none] (start) [above  = 0.3cm  of zero]{$\KK_4$};
      % 
      \path
      (start) edge node {} (zero) 
      (zero) edge[bend left] node[above] {$\ain[d][k_4][][react]$} (one)
      (one) edge[bend left] node[above] {$\aout[k_4][k_2][][pars]$} (zero)
      ;
      \end{tikzpicture}
 \end{array}
 \end{array}
 \\[-10mm]
\begin{tikzpicture}[mycfsm]
		  % 
		  \node[state, initial, initial where = above, initial text={$\ptp[b]$}, fill=black] (zero) {$0$};
		  % 
		  \path
		  (zero) edge [loop below,looseness=40] node[below] {$\ain[k_1][b][][react]$} (zero)
		  ;
		\end{tikzpicture}
  \qquad\quad
      \begin{tikzpicture}[mycfsm]
      %\tikzstyle{every edge}=[carrow]
      % 
      \node[state] (zero) [fill=black] {$0$};
      \node[state] (one) [below right of=zero, xshift=-3mm]   {$1$};
      \node[state] (two) [below left  of=zero, xshift=3mm]   {$2$};
      \node[draw=none,fill=none] (start) [above  = 0.3cm  of zero]{$\ptp[d]$};
      % 
      \path
      (start) edge node {} (zero) 
      (zero) edge[bend left] node[above] {$\ain[k_2][d][][react]$} (one)
      (one) edge[bend left = 40] node[below] {$\aout[d][k_4][][react]$} (two)
      (two) edge[bend left] node[above] {$\aout[d][k_3][][react]$} (zero)
      ;
  \end{tikzpicture}
 \end{array}
 $$
   \caption{\label{fig:orchsyscompproj}A configuration $s$ (top) and two projections of it: $\restrict{s}{1}$ (middle) and $\restrict{s}{\cs}$ (bottom)}
\end{figure*}

It is possible to check that $s$ is reachable from the initial state after the following actions:\\[1mm] 
- $\ttr$ puts $\msg[start]$ in the $\ttr\hh_3$ channel;\\
- $\msg[start]$ is then taken by $\hh_3$ which, in turn, puts the same message in $\hh_3\hh_1$ that is afterwards taken by $\hh_1$;\\
- after receiving $\msg[start]$, $\hh_1$ puts the same message in $\hh_1\ttp$;\\
-  after taking  $\msg[start]$ from $\hh_1\ttp$, $\ttp$ puts $\msg[react]$ in $\ttp\hh_1$;\\
 - $\msg[react]$ is hence taken by $\hh_1$ which then puts the same message in $\hh_1\ttb$;\\
- in the meantime, $\ttp$ puts $\msg[nc]$ in $\ttp\hh_1$.\\



By \cref{def:projectedconf}, the projection on  $S_1$ (top left of \cref{eq:JK}) is
$$\restrict{s}{1}=\big((4_{\ttp},2_{\hh_1}),\, \vec{\varepsilon}\,\big)$$
as graphically depicted in the middle of \cref{fig:orchsyscompproj}.

Also by \cref{def:projectedconf}, the projection of $s$ on the orchestrated connection policy $\cs$ (the second policy of \cref{ex:connsys}) is
$$\restrict{s}{\cs}=\big((\dot{2}_{\kk_1},\dot{0}_{\kk_2},\dot{1}_{\kk_3},\dot{0}_{\kk_4},\dot{0}_{\ttb},\dot{0}_{\ttd}),\, \vec{w}\big)$$
where  \quad $w_{\kk_1\ttb}=\langle\msg[react]\rangle$ \quad $w_{c}=\varepsilon\, (\forall c\neq\kk_1\ttb)$\\
$\restrict{s}{\cs}$ is graphically depicted on the bottom of \cref{fig:orchsyscompproj}.

It is easy to check that the configuration $s'$ reachable from $s$ in case $\hh_1$ takes the message
$\msg[nc]$ from channel $\ttp\hh_1$ would have a projection on $S_1$ which would not be a configuration of $S_1$,
since the configuration contains an intermediate local state (denoted with a hat). 
Moreover, also the projection of $s'$ on $\cs$ would not be defined.


\finex
}
\end{example}


The following lemma
directly descends from how gateways are built (\cref{def:gatewaymc}).
In particular from the fact that the gateway transformation of a machine $M$ does insert an intermediate state
 between any pair of states of $M$ connected by a transition. By definition, the intermediate state
 possesses exactly one incoming transition and one outgoing transition. 

\begin{lemma}\hfill\\
\label{fact:uniquesending}
Let $s= (\vec{q},\vec{w}) \in \RS(S)$. % be a reachable configuration of$S$ % $= \MC(\Set{S_i}_{i\in I}, \cs)$.
For each $i\in I$, by setting $\HH=\HH_i$  and $\KK=\KK_i$,  the following holds, where
$M_\HH\gts M_\KK= (Q_\HH\cup\widehat{Q_\HH}, \_,\_,  \delta_\HH)$.
\begin{enumerate}
\item
\label{fact:uniquesending-i}
If ${q_\HH} \in\widehat{Q_\HH}$ then
${q_\HH}$ is not final and
 there exists a unique transition $({q_\HH},\_,\_)\in\delta_\HH$.
  Moreover, such a transition is of the form
 $(q_\HH,\HH\tts!\msg[a],q')$ with $q'\not\in\widehat{Q_\HH}$.
% \\Similarly for $\KK$.

\item
\label{fact:uniquesending-ii}
If ${q_\HH}\not\in\widehat{Q_\HH}$ then either $q_\HH$ is final, or any transition $({q_\HH},\_,\_)\in\delta_\HH$
is an input  one, that 
is of the form $({q_\HH},\tts\HH?\msg[a],{q'_\HH})$ with ${q'_\HH}\in\widehat{Q_\HH}$. 
%Similarly for $\KK$.
\item
\label{fact:uniquesending-iii}
If ${q_\HH}\not\in\widehat{Q_\HH}$ then 
             \begin{enumerate}[a)]
\item
\label{fact:uniquesending-iiia}

If $(q_\HH,{\ttu}\HH?\msg[a],{q'_\HH})\in\delta_\HH$ for a $\ttu\in\Set{\hh_i}_{i\in I}\cup\rolescsint$,  then there exists $({q'_\HH},\HH\tts!\msg[a],q''_\HH)\in\delta_\HH$ with $\tts \in \roles_i$ 
(and hence $\tts\not\in\Set{\hh_i}_{i\in I}\cup\rolescsint$). 
Moreover, $(q_\HH,\HH\tts!\msg[a],q''_\HH)\in\delta^i_\HH$ and 
$(\dot{q_\HH},\tilde{\ttu}\kk?\msg[a],\dot{q''_\HH})\in\delta^{\cs}_{\kk}$.

%If $(q_\HH,{\HH_j}\HH?a,{q'_\HH})\in\delta_\HH$ for a $j\in I$,  then there exists $({q'_\HH},\HH\tts!a,q''_\HH)\in\delta_\HH$ with $\tts \in \roles_i$ 
%(and hence, for each  $l\in I$, $\tts \neq \HH_l$)  %$j\in I$, $\tts \neq \HH_j$)
%such that $(q_\HH,\HH\tts!a,q''_\HH)\in\delta^i_\HH$.
\item\label{fact:uniquesending-iiib}

If $(q_\HH,\tts\HH?\msg[a], {q'_\HH})\in\delta_\HH$ with $\tts \in \roles_i$ 
(and hence $\tts\not\in\Set{\hh_i}_{i\in I}\cup\rolescsint$)  then there exists   $({q'_\HH},\HH\ttu!\msg[a],q''_\HH)\in\delta_\HH$,
for a $\ttu\in\Set{\hh_i}_{i\in I}\cup\rolescsint$. 
Moreover, $(q_\HH,\tts\HH?\msg[a],q''_\HH)\in\delta^i_\HH$ and 
$(\dot{q_\HH},\kk\widetilde{\ttu}!\msg[a],\dot{q''_\HH})\in\delta^{\cs}_{\kk}$.

%If $(q_\HH,\tts\HH?\msg[a], {q'_\HH})\in\delta_\HH$ with $\tts \in \roles_i$ 
%(and hence, for each $j\in I$, 
%$\tts \neq \HH_j$)  then there exists   $({q'_\HH},\HH\HH_j!\msg[a],q''_\HH)\in\delta_\HH$,
%for a $j\in I$,  
%such that $(q_\HH,\tts\HH?\msg[a],q''_\HH)\in\delta^i_\HH$.
\end{enumerate}
\end{enumerate}
\end{lemma}

The third item of the following lemma (whose proof depends on the first two items)
states that any sequence of transitions ending with an output action by a gateway
can be ``rearranged'' so that the output action is immediately preceded
by its corresponding reception. 


\begin{lemma}
\label{lem:swap-rolf}
Let $S = \MC(\Set{S_i}_{i\in I}, \cs)$ and $\HH \in \Set{\hh_i}_{i\in I}$.

\begin{enumerate}[i)]
\item\label{lem:swap-rolf-item1}
Let $s,s',s''\in \RS(S)$  such that\\
\hspace*{30mm}
$s\lts{\ttr\HH?\msg[a]}s'\lts{\elle}s''$
 where $\elle$ is not of the form $\HH\_!\_$.\\
Then there exists $s'''\in \RS(S)$ such that $s\lts{\elle}s'''\lts{\ttr\HH?\msg[a]}s''$.

\item\label{lem:swap-rolf-item2}
For $j,k \geq 0$, let $s_j \in \RS(S)$ and $s_{j+k+1} \in \RS(S)$ be reachable from $s_j$ by a sequence of transitions of the form
$s_j  \lts{\ttr\HH?\msg[a]} s_{j+1}\lts{\elle_1}\ldots\, s_{j+k}\lts{\elle_{k}}s_{j+k+1}$
 where, for $x = 1,\ldots,k$,  $\elle_x$ is not of the form $\HH\_!\_$.\\
Then  there exists a sequence of transitions of the form
\\
\hspace*{30mm}
$s_j  \lts{\elle_1}s'_{j+1} \ldots\,  \lts{\elle_k} s'_{j+k} \lts{\ttr\HH?\msg[a]}s_{j+k+1}$.

\item\label{lem:swap-rolf-item3}
Let $s \in \RS(S)$ be reachable from $s_0$ by a sequence of
transitions of the form\\
\hspace*{30mm}
$s_0\lts{}  s_1\,\ldots \lts{} s_{n-2}  \lts{} s_{n-1}\lts{\HH\tts!\msg[a]}s_n=s$.\\
Then $n \geq 2$ and there exists a sequence of transitions of the form
\\
\hspace*{30mm}
$s_0\lts{} s'_1\,\ldots \lts{} s'_{n-2}\lts{\ttr\HH?\msg[a]}s_{n-1}\lts{\HH\tts!\msg[a]}s_n=s$.
\end{enumerate}
\end{lemma}

\begin{proof}
(\ref{lem:swap-rolf-item1})
%For all $\tts\in\roles_j\cup\Set{\hh_i}_{i\in I\setminus \Set{j}}$, the action $\elle$ cannot affect the buffer ${w}_{\HH_j\tts}$.\\
%
 Let $s = (\vec{q},\vec{w}), s' = (\vec{q'},\vec{w'})$ and
 $s'' = (\vec{q''},\vec{w''})$.
 For all $\ttp \in \roles\setminus\{\hh\}$ we have $q'_\ttp = q_\ttp$.
 Since $\elle$ is not of the form $\HH\_!\_$ by assumption and
$\elle$ is also not of the form $\_\HH?\_$ by gateway construction, we have $q''_\HH = q'_\HH$. 
 Then we set $q'''_\HH = q_\HH$ and $q'''_\ttp = q''_\ttp$ for all
$\ttp \in \roles\setminus\{\hh\}$.

Concerning the channels, we know that $w_{\ttr\HH} =  \msg[a]\cdot w'_{\ttr\HH}$ and $w'_{\ttp\ttv} =  w_{\ttp\ttv}$ for all
$\ttp\ttv \neq \ttr\HH$. 
Now we set $w'''_{\ttp\ttv} =  w''_{\ttp\ttv}$ for all $\ttp\ttv \neq \ttr\HH$. 
For defining $w'''_{\ttr\HH}$ we consider three cases for $\elle$: 
%
\begin{description}
\item \underline{$\diamond$}
$\elle = \ttr\HH?\msg[b]$ for some $\msg[b]$.\\
As already said above, this case is not possible by construction of gateways.

\item \underline{$\diamond$}
$\elle \neq \ttr\HH?\msg[b]$ and $\elle \neq \ttr\HH!\msg[b]$
for any $\msg[b]$.\\
Then $w''_{\ttr\HH} =  w'_{\ttr\HH}$. We set
$w'''_{\ttr\HH} = w_{\ttr\HH}$ and $s''' = (\vec{q'''},\vec{w'''})$. 
Thus $s\lts{\elle}s'''\lts{\ttr\HH?\msg[a]}s''$. 
%
\item
\underline{$\diamond$}
$\elle = \ttr\HH!\msg[b]$ for some $\msg[b]$.\\
Then $w''_{\ttr\HH} = w'_{\ttr\HH}\cdot\msg[b]$.
We set $w'''_{\ttr\HH} = w_{\ttr\HH}\cdot\msg[b] = \msg[a]\cdot w'_{\ttr\HH}\cdot\msg[b]$ and $s''' = (\vec{q'''},\vec{w'''})$.
Thus $s\lts{\elle}s'''\lts{\ttr\HH?\msg[a]}s''$. 
\end{description}

(\ref{lem:swap-rolf-item2}) The proof is done by induction on $k$.

{\em Case $k = 0$}.
Then we take $s'_{j+k} = s_j$ and the statement is trivial. 

{\em Case $k>0$}. % \mapsto k+1$}.
Let $s_j  \lts{\ttr\HH?\msg[a]} s_{j+1}\lts{\elle_1}s_{j+2}\,\,\ldots\, \lts{\elle_{k}}s_{j+k+1} \lts{\elle_{k+1}}s_{j+k+2}.$
Then, by part (\ref{lem:swap-rolf-item1}) of the lemma, there exists
$s'_{j+1}$ such that $s_j \lts{\elle_1} s'_{j+1}\lts{\ttr\HH?\msg[a]}s_{j+2}$.
By induction hypothesis, there exists a sequence of transitions 
$s'_{j+1}  \lts{\elle_2} \ldots\,  \lts{\elle_{k+1}} s'_{j+k+1} \lts{\ttr\HH?\msg[a]}s_{j+k+2}$. Thus
$s_j \lts{\elle_1} s'_{j+1}  \lts{\elle_2} \ldots\,  \lts{\elle_{k+1}} s'_{j+k+1} \lts{\ttr\HH?\msg[a]}s_{j+k+2}$. 


(\ref{lem:swap-rolf-item3})
Let $s_0\lts{}  s_1\,\ldots \lts{} s_{n-2}  \lts{} s_{n-1}\lts{\HH\tts!\msg[a]}s_n=s$ with $s = (\vec{q},\vec{w})$.
 Due to the construction of gateways
 ${q_{(n-1)}}_{\HH}\in\widehat{Q_{\HH}}$
 %${q_{(n-1)}}_{\HH_i}\in\widehat{Q_{\HH_i}}$ for each $i\in I$
 and, since ${q_0}_{\HH}\not\in\widehat{Q_{\HH}}$ % for each $i\in I$,
 there must be a transition $s_j \lts{\ttr\HH?\msg[a]}s_{j+1}$ for
 some $0\leq j \leq n-2$. In particular, $n \geq 2$ must hold.
Now we take the largest $j$ with $0\leq j \leq n-2$ such that
 $s_j  \lts{\ttr\HH?\msg[a]} s_{j+1}\lts{\elle_1}\ldots\, s_{j+k}\lts{\elle_{k}}s_{n-1}$
 where, for each $x = 1,\ldots,k$,  $\elle_x$ is not of the form $\HH\_!\_$.
By part (\ref{lem:swap-rolf-item2}) of the lemma, there exists
 $s_j  \lts{\elle_1}s'_{j+1} \ldots\,  \lts{\elle_k} s'_{j+k} \lts{\ttr\HH?\msg[a]} s_{n-1}$. Thus we obtain a sequence
$s_0\lts{}\,\,\ldots \lts{}s_j  \lts{\elle_1}s'_{j+1} \ldots\,  \lts{\elle_k} s'_{n-2}\lts{\ttr\HH?\msg[a]}s_{n-1}\lts{\HH\tts!\msg[a]}s_n=s$.
 
 
\end{proof}



%\begin{lemma}
%\label{lem:swap}
%Let $S = \MC(\Set{S_i}_{i\in I}, \cs)$, $j\in I$ and 
%let $s,s',s''\in \RS(S)$  such that\\
%\centerline{$s\lts{\elle}s'\lts{\hh_j\tts!a}s''$ where
%$\elle$ is not of the form $\_\hh_j?\_$.}
%Then, there exists $s'''\in \RS(S)$ such that $s\lts{\hh_j\tts!a}s'''\lts{\elle}s''$.
%\end{lemma}
%
%\begin{proof}
%%Let us consider just the case $\JJ=\HH$, the other one being similar.
%Since $s\in \RS(S)$ and by definition of gateway (Def. \ref{def:gatewaymc}), $\elle$ cannot be of the form $\_\HH!\_$. 
%\brc Why not? Is there a typo? \erc
%So, for all $\ttr\in\roles_j\cup\Set{\hh_i}_{i\in I\setminus \Set{j}}$, the action $\elle$ cannot affect the buffer ${w}_{\ttr\HH_j}$.\\
%It is now easy to check that, by defining $s'''=(\vec{q'''},\vec{w'''})$ such that  
% $q'''_{\HH_j} = q''_{\HH_j}$,  and $q'''_{\ttp} = q_{\ttp}$  %${q'''}_{\HH_j} = {q''}_{\HH_j}$,  and ${q'''}_{\ttp} = {q}_{\ttp}$
%for $\ttp\neq \HH$,
%and such that
% $w'''_{\HH_j\tts} = w_{\HH_j\tts}\cdot \msg[a]$ and  $w'''_{\ttp\ttq} = w_{\ttp\ttq}$  %${w'''}_{\HH_j\tts} = {w}_{\HH_j\tts}\cdot a$ and  ${w'''}_{\ttp\ttq} = {w}_{\ttp\ttq}$
% for $\ttp\ttq\neq \HH_j\tts$, we get $s\lts{\HH_j\tts!a}s'''\lts{\elle}s''$.
%\end{proof}

The following lemma relates transitions in a composed systems to transitions, if any, in the systems
used to get the composition. In particular, the fourth item specifies that two consecutive
actions of a gateway making a message forwarded, do correspond to a single action in an interface.
\begin{lemma}
\label{lem:indrestrict}
Let $s\in \RS(S)$ be a reachable configuration of $S = \MC(\Set{S_i}_{i\in I}, \cs)$. Then, for each $i\in I$,
\begin{enumerate}[i)]
\item\label{lem:indrestrict1}
$\restrict{s_0}{i}\in \RS(S_i)$.
%\item\label{lem:indrestrict1b}
%\bfr
%Let $s\lts{\elle}s'$ such that $\prt{\elle}\cap\{\HH_i\}_{i \in I}=\emptyset$,
%then one of the following hold:
%\begin{enumerate}[a)]
%\item 
%$\prt{\elle}\subseteq \roles_i$; or
%\item
%$\prt{\elle}\subseteq  \roles_{\cs}\setminus\Set{\kk_i}_{i\in I}$; or
%\item
%$\prt{\elle}\subseteq  \roles_j$ for some $j\in I\setminus\Set{i}$.
%\end{enumerate}
%\efr
\item\label{lem:indrestrict2}
Let $s = (\vec{q},\vec{w})\lts{\elle}s'$
 and let  $\restrict{s}{i}$ be a configuration of $S_i$, i.e.  $q_{\hh_i}\not\in\widehat{Q_{\HH_i}}$. 
If $\elle$ is neither of the form $\_\,\HH_i?\_$ nor of the form $\HH_i\_!\_$,
then either $\restrict{s}{i}\lts{\elle}\restrict{s'}{i}$ or  $\restrict{s}{i}=\restrict{s'}{i}$.

\item\label{lem:indrestrict2cs}
 Let $s = (\vec{q},\vec{w})\lts{\elle}s'$
 and let  $\restrict{s}{\cs}$ be defined, i.e. a configuration of $\cs$.
% , i.e.  $q_{\hh_i}\not\in\widehat{Q_{\HH_i}}$ for each $i\in I$. 
If, for each $i\in I$,  $\elle$ is neither of the form $\_\,\HH_i?\_$ nor of the form $\HH_i\_!\_$,
then either $\restrict{s}{\cs}\lts{\elle}\restrict{s'}{\cs}$ or  $\restrict{s}{\cs}=\restrict{s'}{\cs}$.


\item\label{lem:indrestrict3}
Let $s\lts{\ttr\HH_i?\msg[a]} s'\lts{\HH_i\tts!\msg[a]} s''$.
Then $\ttr \in \roles_i$ implies  $\restrict{s}{i}\lts{\ttr\HH_i?\msg[a]}\restrict{s''}{i}$ ,
 and $\tts \in \roles_i$ implies  $\restrict{s}{i}\lts{\HH_i\tts!\msg[a]}\restrict{s''}{i}$. 
\end{enumerate}
\end{lemma}

\begin{proof}
(\ref{lem:indrestrict1}) is trivially true.

%\noindent
%\bfr
%(\ref{lem:indrestrict1b}) 
%By definition of composition and the fact that the sets $\roles_i$'s and 
%$\roles_{\cs}\setminus\Set{\kk_i}_{i\in I}$ are pairwise disjoint.
%\efr

\noindent
(\ref{lem:indrestrict2}) Let $i\in I$. Besides, let $s = (\vec{q},\vec{w})$ and
$s' = (\vec{q'},\vec{w'})$.  We proceed now by cases, according to the form of
$\elle$. Since the sets $\roles_i$s and $\rolescsint$ are pairwise disjoint,
only the following three cases are possible:

\begin{description}
\item
\underline{$\diamond$}
$\elle$ is of the form $\ttr\tts?\msg[a]$ or $\ttr\tts!\msg[a]$
with $\ttr,\tts \in \roles_i$.\\ 
Since $\elle$ is neither of the form $\_\,\HH_i?\_$ nor of the form $\HH_i\_!\_$
it holds that   
$q_{\HH_i},  q'_{\HH_i} \not\in \widehat{Q_{\HH_i}}$ (by construction of gateways) and, in particular, 
$q_{\HH_i} = q'_{\HH_i}$.
Then, by definitions of transition and projection, we get $\restrict{s}{i}\lts{\elle}\restrict{s'}{i}$.

\item
\underline{$\diamond$}
$\elle$ is of the form $\ttr\tts?\msg[a]$ or $\ttr\tts!\msg[a]$
with $\ttr,\tts \in \roles_j$ for a certain  $j$ such that $j \neq i$.\\
Since $\elle$ is neither of the form $\_\,\HH_i?\_$ nor of the form $\HH_i\_!\_$
it holds that $q_\ttv = q'_\ttv$ for all $\ttv \in \roles_i$
and $w_{\ttu\ttv} = w'_{\ttu\ttv}$ for all $\ttu,\ttv \in \roles_i$.
Hence, $\restrict{s}{i}=\restrict{s'}{i}$.
\item
\underline{$\diamond$}
 $\elle$ is of the form $\ttr\tts?\msg[a]$ or $\ttr\tts!\msg[a]$
with $\ttr,\tts \in \rolescsint\cup\Set{\hh_k}_{k\in I}$. \\ 
As in the previous case, 
since $\elle$ is neither of the form $\_\,\HH_i?\_$ nor of the form $\HH_i\_!\_$
it holds that $q_\ttv = q'_\ttv$ for all $\ttv \in \roles_i$
and $w_{\ttu\ttv} = w'_{\ttu\ttv}$ for all $\ttu,\ttv \in \roles_i$.
Hence, $\restrict{s}{i}=\restrict{s'}{i}$.

\end{description}
%
%
\noindent
(\ref{lem:indrestrict2cs}) 
Let $s = (\vec{q},\vec{w})$ and
$s' = (\vec{q'},\vec{w'})$.  We proceed now by cases, according to the form of
$\elle$. Since the sets $\roles_i$s and $\rolescsint$ are pairwise disjoint,
only the following two cases are possible:

\begin{description}
\item
\underline{$\diamond$}
$\elle$ is of the form $\ttr\tts?\msg[a]$ or $\ttr\tts!\msg[a]$
with $\ttr,\tts \in \roles_j$ for a certain $j\in I$.\\ 
Since $\elle$ is neither of the form $\_\,\HH_i?\_$ nor of the form $\HH_i\_!\_$
for each $i\in I$,
it holds that $q_\ttv = q'_\ttv$ for all $\ttv \in  \{\HH_k\}_{k \in I}\cup\rolescsint$
and $w_{\ttu\ttv} = w'_{\ttu\ttv}$ for all $\ttu,\ttv \in \{\HH_k\}_{k \in I}\cup\rolescsint$.
This implies that also $\restrict{s'}{\cs}$ is defined and $\restrict{s}{\cs}=\restrict{s'}{\cs}$.

%\item
%\underline{$\diamond$}
%$\elle$ is of the form $\ttr\tts?\msg[a]$ or $\ttr\tts!\msg[a]$
%with $\ttr,\tts \in \roles_j $ for a certain  $j$ such that $j \neq i$.\\
%Since $\elle$ is neither of the form $\_\,\HH_i?\_$ nor of the form $\HH_i\_!\_$
%it holds that   
%$q_{\HH_i},  q'_{\HH_i} \not\in \widehat{Q_{\HH_i}}$ (by construction of gateways) and, in particular, 
%$q_{\HH_i} = q'_{\HH_i}$.
%Then, by definitions of transition and projection, we get $\restrict{s}{\cs}\lts{\elle}\restrict{s'}{\cs}$.
\item
\underline{$\diamond$}
 $\elle$ is of the form $\ttr\tts?\msg[a]$ or $\ttr\tts!\msg[a]$
with $\ttr,\tts \in \rolescsint\cup\Set{\hh_k}_{k\in I}$. \\ 
%As in the previous case, 
%since $\elle$ is neither of the form $\_\,\HH_i?\_$ nor of the form $\HH_i\_!\_$
%it holds that $q_\ttv = q'_\ttv$ for all $\ttv \in \roles_i$
%and $w_{\ttu\ttv} = w'_{\ttu\ttv}$ for all $\ttu,\ttv \in \roles_i$.
%Hence, $\restrict{s}{i}=\restrict{s'}{i}$.
Since $\elle$ is neither of the form $\_\,\HH_i?\_$ nor of the form $\HH_i\_!\_$ for each $i\in I$,
and since $\restrict{s}{\cs}$ is defined,
it holds that   
$q_{\HH_i}=  q'_{\HH_i} \not\in \widehat{Q_{\HH_i}}$ for each $i\in I$ (by construction of gateways). Hence also $\restrict{s'}{\cs}$ is defined.
Then, by definitions of transition and projection, we get $\restrict{s}{\cs}\lts{\elle}\restrict{s'}{\cs}$.
\end{description}

%
%
(\ref{lem:indrestrict3}) This item follows from the construction of gateways.
In particular, either $\set{\ttr,\tts} \cap \roles_i=\ttr$ or $\set{\ttr,\tts} \cap \roles_i=\tts$. 
% OLD version of the proof
%
%Easy by definitions of $\restrict{s}{i}$, by definition of $\lts{}$ and by definitions of  gateway and multicomposition 
%(Defs \ref{def:gatewaymc} and \ref{def:multicomposition}).  For Point~\ref{lem:indrestrict3} notice that either $\set{\ttr,\tts} \cap \roles_i=\ttr$ or $\set{\ttr,\tts} \cap \roles_i=\tts$. 
\end{proof}

Notice that $\restrict{s}{i}$ is not necessarily a configuration of $S_i$, because of the possible presence of the additional states introduced by the gateways construction.
However, if reachable configurations of $S=\MC(\Set{S_i}_{i\in I}, \cs)$
do not involve intermediate states of  gateways, 
%$M_\HH = \gateway{M^1_\HH, \KK}$, %$M_\KK = \gateway{M^2_\KK, \HH}$, 
then, by projection,
% by taking into account only the states of machines of  $S_i$  and disregarding the channels between the gateways, %(see Definition \ref{def:restrictedconf} above),
we get reachable configurations of  $S_i$ and $\cs$.  

\begin{proposition}[On reachability of projections]
\label{lem:nohatrestrict}
Let $s= (\vec{q},\vec{w}) \in \RS(S)$. % be a reachable configuration of $S = \MC(\Set{S_i}_{i\in I}, \cs)$.
\begin{enumerate}[i)]
\item
\label{lem:nohatrestrict-a}
 For each $i\in I$, $({q}_{\HH_i}\not\in\widehat{Q_{\HH_i}} \implies 
\restrict{s}{i}\in \RS(S_i))$;
%\brc Shouldn't we use $i$ instead of $k$ as above?\erc
\item
\label{lem:nohatrestrict-b}
$({q}_{\HH_i}\not\in\widehat{Q_{\HH_i}}$ for each $i\in I)$ $\implies$
$\restrict{s}{\cs}\in \RS(\cs)$.
\end{enumerate}
%\begin{enumerate}[i)]
%\item
%\label{lem:nohatrestrict-i}
%${q}_\HH\not\in\widehat{Q_\HH} \implies 
%\restrict{s}{1}\in \RS(S_1)$;
%\item
%\label{lem:nohatrestrict-ii}
%${q}_\KK\not\in\widehat{Q_\KK} \implies 
%\restrict{s}{2}\in \RS(S_2)$.
%\end{enumerate}
\end{proposition}


\begin{proof}
{ (\ref{lem:nohatrestrict-a})}
Let $s = (\vec{q},\vec{w}) \in \RS(S)$ and $i \in I$ such that ${q}_{\HH_i}\not\in\widehat{Q_{\HH_i}}$.
We prove $\restrict{s}{i}\in \RS(S_i)$  by (well-founded) induction on the length $n$ of the transition sequence to reach $s$ from the initial state $s_0$.

{\em Case $n=0$}. Then $\restrict{s}{i} = \restrict{s_0}{i} \in \RS(S_i)$
by~\cref{lem:indrestrict}(\ref{lem:indrestrict1}).


{\em Case $n>0$}.
Then there exists $s_x = (\vec{q_x},\vec{w_x}) \in \RS(S)$ and an action $\elle_x$ over $\roles$
such that $s_x\lts{\elle_x}s$ and $s_x$ is reachable from $s_0$ in $n-1$ steps. We consider the following cases:

\begin{description}
%
\item
\underline{$\diamond$}
$\elle_x$ is neither of the form $\_\,\HH_i?\_$ nor of the form $\HH_i\_!\_$.\\
Then ${q_x}_{\HH_i}={q}_{\HH_i}\not\in\widehat{Q_{\HH_i}}$.
Moreover, by~\cref{lem:indrestrict}(\ref{lem:indrestrict2}),
either $\restrict{s_x}{i}\lts{\elle_x}\restrict{s}{i}$ or  $\restrict{s_x}{i}=\restrict{s}{i}$.\\
Since ${q_x}_{\HH_i}\not\in\widehat{Q_{\HH_i}}$ we can apply the induction hypothesis for $s_x$ and obtain $\restrict{s_x}{i}\in \RS(S_i)$.\\
Hence $\restrict{s}{i}\in \RS(S_i)$.
%
\item
\underline{$\diamond$}
$\elle_x$ is of the form $\_\,\HH_i?\_$.\\
This is not possible since otherwise,
by definition of gateways,
${q}_{\HH_i}\in\widehat{Q_{\HH_i}}$ which is excluded. 
%
\item
\underline{$\diamond$}
$\elle_x$ is of the form $\HH_i\_!\_$.\\
More explicitly, let $s_x\lts{\HH_i\tts!\msg[a]}s$.
%By definition of gateways, ${q_x}_{\HH_i}\in\widehat{Q_{\HH_i}}$.
%Therefore $s_x \neq s_0$ and thus $s$ is reachable from $s_0$ in $n \geq 2$ steps.
Then, by~\cref{lem:swap-rolf}(\ref{lem:swap-rolf-item3}), there exist
transitions
$s_y \lts{\ttr\HH_i?\msg[a]}s_x\lts{\HH_i\tts!\msg[a]}s$
such that $s_y = (\vec{q_y},\vec{w_y})\in \RS(S)$ is reachable from $s_0$ in $n-2$ steps. By definition of gateways, ${q_y}_{\HH_i} \not\in\widehat{Q_{\HH_i}}$.
Hence, we can apply the induction hypothesis for $s_y$ and obtain $\restrict{s_y}{i}\in \RS(S_i)$.
Moreover, by~\cref{lem:indrestrict}(\ref{lem:indrestrict3}) we get a transition
$\restrict{s_y}{i}\lts{\elle}\restrict{s}{i}$. Hence $\restrict{s}{i}\in \RS(S_i)$.
\end{description}
%
%{\bf (\ref{lem:nohatrestrict-a})}[\brc OLD \erc NEW VE\RSION].
%Let $s = (\vec{q},\vec{w}) \in \RS(S)$ and $i \in I$ such that ${q}_{\HH_i}\not\in\widehat{Q_{\HH_i}}$.
%We prove $\restrict{s}{i}\in RS(S_i)$  by (well-founded) induction on the length $n$ of the transition sequence to reach $s$ from the initial state $s_0$.
%
%{\em Case $n=0$}. Then $\restrict{s}{i} = \restrict{s_0}{i} \in RS(S_i)$
%by~\cref{lem:indrestrict}(\ref{lem:indrestrict1}).
%
%{\em Case $n>0$}.
%Then there exists $s_x = (\vec{q_x},\vec{w_x}) \in RS(S)$ and an action $\elle_x$ over $\roles$
%such that $s_x\lts{\elle_x}s$. We consider the following cases:
%
%\begin{description}
%%
%\item
%\underline{$\diamond$}
%$\elle_x$ is neither of the form $\_\,\HH_i?\_$ nor of the form $\HH_i\_!\_$.\\
%Then ${q_x}_{\HH_i}={q}_{\HH_i}\not\in\widehat{Q_{\HH_i}}$.
%Moreover, by~\cref{lem:indrestrict}(\ref{lem:indrestrict2}),
%either $\restrict{s_x}{i}\lts{\elle_x}\restrict{s}{i}$ or  $\restrict{s_x}{i}=\restrict{s}{i}$.\\
%Since ${q_x}_{\HH_i}\not\in\widehat{Q_{\HH_i}}$ we can apply the induction hypothesis for $s_x$ and obtain $\restrict{s_x}{i}\in RS(S_i)$.\\
%Hence $\restrict{s}{i}\in RS(S_i)$.
%%
%\item
%\underline{$\diamond$}
%$\elle_x$ is of the form $\_\,\HH_i?\_$.\\
%This is not possible since otherwise,
%by definition of gateways,
%${q}_{\HH_i}\in\widehat{Q_{\HH_i}}$ which is excluded. 
%%
%\item
%\underline{$\diamond$}
%$\elle_x$ is of the form $\HH_i\_!\_$.\\
% More explicitly, let $s_x\lts{\HH_i\tts!\msg[a]}s$.
%Then, by definition of gateways, ${q_x}_{\HH_i}\in\widehat{Q_{\HH_i}}$.
%Therefore $s_x \neq s_0$ and thus there exists $s_y\lts{\elle_y}s_x\lts{\HH_i\tts!\msg[a]}s$.
%By definition of gateways, $\elle_y$ can not be of the form $\HH_i\_!\_$. 
%Now we distinguish two subcases:
%%
%\begin{itemize}
%\item[(1)]
%$\elle_y$ is not of the form $\_\,\HH_i?\_$.\\
%Then, by~\cref{lem:swap}, there exists $s_z \in RS(S)$ such that
%$s_y\lts{\hh_i\tts!a}s_z\lts{\elle_y}s$.
%In particular, ${q_z}_{\HH_i}={q}_{\HH_i}\not\in\widehat{Q_{\HH_i}}$
%since $\HH_i$ is not involved in $\elle_y$.
%Moreover, by~\cref{lem:indrestrict}(\ref{lem:indrestrict2}),
%either $\restrict{s_z}{i}\lts{\elle_y}\restrict{s}{i}$ or  $\restrict{s_z}{i}=\restrict{s}{i}$.
%We can now apply the induction hypothesis for $s_z$ and obtain $\restrict{s_z}{i}\in RS(S_i)$.
%Hence $\restrict{s}{i}\in RS(S_i)$.
%%
%\item[(2)]
%$\elle_y$ is of the form $\_\,\HH_i?\_$.\\
%Then, by definition of gateways, ${q_y}_{\HH_i} \not\in\widehat{Q_{\HH_i}}$.
%Moreover,  $\elle_y$ must be of the form $\ttr\HH_i?\msg[a]$ and thus we have two transitions
%$s_y\lts{\ttr\HH_i?\msg[a]} s_x\lts{\HH_i\tts!\msg[a]} s$.
%By~\cref{lem:indrestrict}(\ref{lem:indrestrict3}) we get a transition
%$\restrict{s_y}{i}\lts{\elle}\restrict{s}{i}$. 
%We can now apply the induction hypothesis for $s_y$ (which is reached
%from $s_0$ by $n-2$ steps) and obtain $\restrict{s_y}{i}\in RS(S_i)$.
%Hence $\restrict{s}{i}\in RS(S_i)$.
%\end{itemize}
%\end{description}
%%
%{\bf (\ref{lem:nohatrestrict-a})}. [OLD VERSION] \\
%If $s \in RS(S)$, then there exists a transition sequence leading to $s$ from the initial state, say
%$$s_0\lts{}s_1\lts{} \ldots\lts{} s_{n-1}\lts{}s_n=s$$
%where  $s_l = (\vec{q_l},\vec{w_l})$ $(l=0,\ldots,n)$.  
%\\
%Let then $k\in I$ and ${q}_{\HH_k}\not\in\widehat{Q_{\HH_k}}$.
%For the sake of readability, let us set $\hh = \hh_{k}$. 
% We denote by ${q_v}_\HH$ the local state of the machine $M_\HH$  configuration $s_v$.  
%Let now $j \geq 0$ be the smallest index such that ${q_j}_\HH\not\in \widehat{Q_\HH}$ and  ${q_{j+1}}_\HH\in \widehat{Q_\HH}$
%(if there is not such a $j$, then the thesis follows immediately).
%By definitions of  gateway and multicomposition 
%(Defs \ref{def:gatewaymc} and \ref{def:multicomposition})  
%we have that $s_j\lts{\ttr\HH?\msg[a]} s_{j+1}$ for a certain $\ttr$.
%Now let $t$ be the smallest index such that  $t\geq j+1$, ${q_t}_\HH = {q_{j+1}}_\HH$ and ${q_{t+1}}_\HH\not\in \widehat{Q_\HH}$.
%Such an index $t$ does exist because of the hypothesis $q_\HH\not\in\widehat{Q_\HH}$
%(moreover, notice that no self loop transitions are possible out of a state in $\widehat{Q_\HH}$).
%By definition of gateway and multicomposition 
%(Defs \ref{def:gatewaymc} and \ref{def:multicomposition}) 
%we have that $s_{t}\lts{\HH\tts!\msg[a]} s_{t+1}$ for a certain $\tts$.\\
%We can now proceed by induction on the length of the transition sequence
%\centerline{
%$s_j\lts{\ttr\HH?\msg[a]} \ldots \lts{\HH\tts!\msg[a]}s_{t+1}$
%}
%using Lemma \ref{lem:swap}, in order to show that 
%it is possible to build a transition sequence like the following one\\
%\centerline{
%$s_0\lts{}s_1\lts{}\ldots s_j\lts{\ttr\HH?\msg[a]} s_{j+1}\lts{\HH\tts!\msg[a]} s'_{j+2} \lts{} \ldots \lts{}s'_{n-1}\lts{}s_n=s$
%}
%where ${q_{j+2}}_\HH\not\in \widehat{Q_\HH}$.\\
%The iteration of this procedure trivially converges and allow us to get a sequence
%\\
%\begin{equation}
%\label{eq:goodseqa}
%s_0\lts{}\ldots  \lts{}s_n=s
%\end{equation}
%such that any transition of the form $\ttr\HH?a$ is immediately followed by a transition  of the form   $\HH\tts!a$.
%Note that, by definition of gateway, we have also that any transition of the form $\HH\ttr!a$ is immediately preceded by a transition  of the form  $\tts\HH?a$.
%
%Now, by using Lemma \ref{lem:indrestrict}(\ref{lem:indrestrict3}),
% it is possible to get, out of the transition sequence (\ref{eq:goodseqa}),
%a transition sequence $\restrict{s_0}{k} \lts{}^* \restrict{s}{k}$. So $\restrict{s}{k}\in RS(S_k)$.\\
%
%
%{\bf (\ref{lem:nohatrestrict-b})}.
%\bfr
%If $s \in RS(S)$, then there exists a transition sequence leading to $s$ from the initial state, say
%$$s_0\lts{}s_1\lts{} \ldots\lts{} s_{n-1}\lts{}s_n=s$$
%where  $s_l = (\vec{q_l},\vec{w_l})$ $(l=0,\ldots,n)$.  
%\\
%Let now $k\in I$ and ${q}_{\HH_k}\not\in\widehat{Q_{\HH_k}}$.
%For the sake of readability, let us set $\hh = \hh_{k}$. 
% We denote by ${q_v}_\HH$ the local state of the machine $M_\HH$  configuration $s_v$.  
%Let now $j \geq 0$ be the smallest index such that ${q_j}_\HH\not\in \widehat{Q_\HH}$ and  ${q_{j+1}}_\HH\in \widehat{Q_\HH}$
%(if there is not such a $j$, then the thesis follows immediately).
%By definitions of  gateway and multicomposition 
%(Defs \ref{def:gatewaymc} and \ref{def:multicomposition})  
%we have that $s_j\lts{\ttr\HH?\msg[a]} s_{j+1}$ for a certain $\ttr$.
%Now let $t$ be the smallest index such that  $t\geq j+1$, ${q_t}_\HH = {q_{j+1}}_\HH$ and ${q_{t+1}}_\HH\not\in \widehat{Q_\HH}$.
%Such an index $t$ does exist because of the hypothesis $q_\HH\not\in\widehat{Q_\HH}$
%(moreover, notice that no self loop transitions are possible out of a state in $\widehat{Q_\HH}$).
%By definition of gateway and multicomposition 
%(Defs \ref{def:gatewaymc} and \ref{def:multicomposition}) 
%we have that $s_{t}\lts{\HH\tts!\msg[a]} s_{t+1}$ for a certain $\tts$.\\
%We can now proceed by induction on the length of the transition sequence
%\centerline{
%$s_j\lts{\ttr\HH?\msg[a]} \ldots \lts{\HH\tts!\msg[a]}s_{t+1}$
%}
%using Lemma \ref{lem:swap}, in order to show that 
%it is possible to build a transition sequence like the following one\\
%\centerline{
%$s_0\lts{}s_1\lts{}\ldots s_j\lts{\ttr\HH?\msg[a]} s_{j+1}\lts{\HH\tts!\msg[a]} s'_{j+2} \lts{} \ldots \lts{}s'_{n-1}\lts{}s_n=s$
%}
%where ${q_{j+2}}_\HH\not\in \widehat{Q_\HH}$.\\
%The iteration of this procedure trivially converges and allow us to get a sequence
%\begin{equation}
%\label{eq:goodseqa}
%s_0\lts{}\ldots  \lts{}s_n=s
%\end{equation}
%such that any transition of the form $\ttr\HH?a$ is immediately followed by a transition  of the form   $\HH\tts!a$.
%Note that, by definition of gateway, we have also that any transition of the form $\HH\ttr!a$ is immediately preceded by a transition  of the form  $\tts\HH?a$.
%
%\efr
%Let us now assume ${q}_{\HH_k}\not\in\widehat{Q_{\HH_k}}$ for each $k\in I$.
%Then, by applying for each $k\in I$ the procedure used in the previous item to get the transition sequence (\ref{eq:goodseqa}), we get a sequence
%\begin{equation}
%\label{eq:goodseqab}
%s_0\lts{}\ldots  \lts{}s_n=s
%\end{equation}
%such that, for each $k\in I$, any transition of the form $\ttr\HH_k?a$ is immediately followed by a transition $\HH_k\tts!a$.\\
%We proceed now by complete induction on the length $n$ of the transition sequence 
%($\ref{eq:goodseqab}$).\\
%{\em Case $n=0$}. In such a case $s_0 = (({p_0}_{\ttp})_{\ttp\in\roles},\vec{\varepsilon})$
%and we get that $\restrict{s_0}{\cs}$ is the initial configuration of $\cs$ by
%definitions of connection policy and projection (\cref{def:projectedconf} and  \cref{def:connpol}).\\
%{\em Case $n > 0$}. We consider separately the possible cases, where
%we assume, for $0\leq z\leq n$,  $s_{z} = (\vec{{q}_{z}},\vec{{w}_{z}})$:
%\begin{description}
%%
%\item
%$s_{n-1}\lts{\tts\ttr?\msg[a]}s_n$, with $\ttr\not\in\Set{\HH_i}_{i\in I}$.\\
%The thesis follows immediately by induction since, by definition of projection 
%(Def. \ref{def:projectedconf}), we have that $\restrict{s_{n-1}}{\cs} = \restrict{s_{n}}{\cs}$.
%%
%\item
%$s_{n-1}\lts{\tts\ttr!\msg[a]}s_n$, with $\tts\not\in\Set{\HH_i}_{i\in I}$.\\
%This case can be treated as the previous one. 
%%
%\item
%$s_{n-1}\lts{\ttr\HH_v?\msg[a]}s_n$, with $\ttr\not\in\Set{\HH_i}_{i\in I}$.\\
%This case cannot actually apply, since, by definition of gateway, we would get 
%${q_n}_{\HH_v}\in \widehat Q_{\HH_v}$,  so making our thesis vacuously true.
%%
%\item
%$s_{n-1}\lts{\HH_v\ttr!\msg[a]}s_n$, with $\ttr\not\in\Set{\HH_i}_{i\in I}$.\\
%Then necessarily $s_{n-2}\lts{\HH_j\HH_v?\msg[a]}s_{n-1}\lts{\HH_v\ttr!\msg[a]}s_n$ for a certain $j\in I$,
%where
%$$({q_{n-2}}_{\HH_v},\HH_j\HH_v?\msg[a],\hat q),
%    (\hat q, \HH_v\ttr!\msg[a], {q_{n}}_{\HH_v})\in\delta_{\HH_v}
%\quad\text{ and }\quad {w_{n-2}}_{\HH_j\HH_v} = a\cdot {w_{n}}_{\HH_j\HH_v}$$
%and where, by definition of gateway and connection policy,
%$$(\dot{q_{n-2}}_{\HH_v},\KK_j\KK_v?\msg[a],\dot{q_{n}}_{\HH_v})\in \delta^\cs_{\KK_v}$$
%Since, by definition of gateway, we have ${q_{n-2}}_{\HH_v}\not\in\widehat{Q_{\HH_v}}$,
%we have also that ${q_{n-2}}_{\HH_k}\not\in\widehat{Q_{\HH_k}}$ for each $k\in I$.  
%Hence we can recur to the induction hypothesis,  getting
%$\restrict{s_{n-2}}{\cs}\in RS(S)$. 
%If $\restrict{s_{n-2}}{\cs} = (\vec{q'},\vec{w'})$
%and $\restrict{s_{n}}{\cs} = (\vec{q''},\vec{w''})$ then,
% by definition of projection and the above, we have 
%$$q'_{\KK_v} = \dot{{{q}}_{n-2}}_{\HH_v} \quad
%q''_{\KK_v} = \dot{{{q}}_{n}}_{\HH_v} \quad
%%\quad\text{and}\quad 
%w'_{\KK_j\KK_v} = {w_{n-2}}_{\HH_j\HH_v} = \msg[a]\cdot {w_{n}}_{\HH_j\HH_v} = \msg[a]\cdot {w''}_{\KK_j\KK_v}.$$
%Hence, by definition of transition, we get $\restrict{s_{n-2}}{\cs}\lts{\KK_j\KK_v?\msg[a]}\restrict{s_{\bmr n \emr}}{\cs}$, namely $\restrict{s_{n}}{\cs}\in RS(\cs)$.
%%
%\item
%$s_{n-1}\lts{\HH_v\HH_z!\msg[a]}s_n$.\\
%Then necessarily $s_{n-2}\lts{\ttr\HH_v?\msg[a]}s_{n-1}\lts{\HH_v\HH_z!\msg[a]}s_n$ for a certain 
%$\ttr\not\in\Set{\HH_i}_{i\in I}$,
%where
%$$({q_{n-2}}_{\HH_v},\ttr\HH_v?\msg[a],\hat q),
%    (\hat q, \HH_v\HH_z!\msg[a], {q_{n}}_{\HH_v})\in\delta_{\HH_v}
%\quad\text{ and }\quad {w_{n}}_{\HH_v\HH_z} = {w_{n-2}}_{\HH_v\HH_z}\cdot \msg[a]$$
%and where, by definition of gateway and connection policy,
%$$(\dot{q_{n-2}}_{\bmr \HH_v\emr} %{\HH_j}
%,\KK_v\KK_z!\msg[a],\dot{q_{n}}_{\HH_v})\in \delta^\cs_{\KK_v}$$
%Since, by definition of gateway, we have ${q_{n-2}}_{\HH_v}\not\in\widehat{Q_{\HH_v}}$,
%we have also that ${q_{n-2}}_{\HH_k}\not\in\widehat{Q_{\HH_k}}$ for each $k\in I$.  
%Hence we can recur to the induction hypothesis,  getting
%$\restrict{s_{n-2}}{\cs}\in RS(S)$. 
%If $\restrict{s_{n-2}}{\cs} = (\vec{q'},\vec{w'})$
%and $\restrict{s_{n}}{\cs} = (\vec{q''},\vec{w''})$ then,
% by definition of projection and the above, we have 
%$$q'_{\KK_v} = \dot{{{q}}_{n-2}}_{\HH_v} \quad
%q''_{\KK_v} = \dot{{{q}}_{n}}_{\HH_v} \quad
%%\quad\text{and}\quad 
%w''_{\KK_v\KK_z} = {w_{n}}_{\HH_v\HH_z} = {w_{n-2}}_{\HH_v\HH_z}\cdot \msg[a] = {w'}_{\bmr \KK_v\KK_z\emr}\cdot \msg[a].$$
%%\bmc erase the following sentence?: Hence, by definition of transition, we get $\restrict{s_{n-2}}{\cs}\lts{\KK_j\KK_v?\msg[a]}\restrict{s_{n-2}}{\cs}$, namely $\restrict{s_{n-2}}{\cs}\in RS(\cs)$. \emc
%Hence, by definition of transition, we get $\restrict{s_{n-2}}{\cs}\lts{\KK_v\KK_z?\msg[a]}\restrict{s_{n}}{\cs}$, namely $\restrict{s_{n}}{\cs}\in RS(\cs)$.
%%
%\item
%\bfr
%$s_{n-1}\lts{\HH_z\HH_v?\msg[a]}s_n$.\\
%This case cannot occur, since $s_n = s$ and we assumed ${q}_{\HH_k}\not\in\widehat{Q_{\HH_k}}$ for each $k\in I$.
%\efr
%\end{description}
%%\brc
%%Isn't the case $s_{n-1}\lts{\HH_z\HH_v?\msg[a]}s_n$ missing?
%%\erc
%
%\brc
%\\
%Below I have shortened the proof of (ii) by omitting \bfr your prelude \efr
%and using~\cref{lem:swap-rolf}(ii) instead.\\

\noindent
{ (\ref{lem:nohatrestrict-b})}.
Let $s = (\vec{q},\vec{w}) \in \RS(S)$ such that ${q}_{\HH_i}\not\in\widehat{Q_{\HH_i}}$ for all $i \in I$.
We prove $\restrict{s}{\cs}\in \RS(\cs)$  by (well-founded) induction on the length $n$ of the transition sequence to reach $s$ from the initial state $s_0$.

{\em Case $n=0$}. 
In this case $s_0 = (({q_0}_{\ttp})_{\ttp\in\roles},\vec{\varepsilon})$.
Then, by definitions of orchestrated connection policy  %policy 
and projection (\cref{def:cs} and \cref{def:projectedconf} ),
$\restrict{s_0}{\cs} = (({p}_{\ttu})_{\ttu \in \roles_{\cs}},(w'_{\ttu\ttv})_{\ttu,\ttv\in \roles_{\cs},\ttu\neq\ttv})$
%$\restrict{s_0}{\cs} = (({p}_{\KK_i})_{i \in I},(w'_{\ttp\ttq})_{\ttp,\ttq\in \Set{\KK_i}_{i\in I},\ttp\neq\ttq})$
where, recalling that $\roles_{\cs}=\Set{\kk_i}_{i\in I}\cup\rolescsint$, we have that ${p}_{\KK_i} = \dot{q_0}_{\HH_i}$ for each $i\in I$ (since ${q_0}_{\HH_i}\not\in \widehat Q_{\HH_i}$ for all $i \in I$) and  ${p}_{\ttu} = {q_0}_{\ttu}$ for each $\ttu\in \rolescsint$.
Moreover, $w'_{\ttu\ttv} = %{w_0}_{\HH_j\HH_v} = 
\varepsilon$ for each $\ttu,\ttv \in \roles_{\cs},\ttu\neq\ttv$.
Obviously, $\restrict{s_0}{\cs}$ is the initial configuration of $\cs$
and thus $\restrict{s_0}{\cs}\in \RS(\cs)$.\\
{\em Case $n>0$}.
Then there exists $s_x = (\vec{q_x},\vec{w_x}) \in \RS(S)$ and an action $\elle_x$ over $\roles$
such that $s_x\lts{\elle_x}s$ and $s_x$ is reachable from $s_0$ in $n-1$ steps. We consider the following cases:

\begin{description}
%
\item

\underline{$\diamond$}
$\elle_x$ is neither of the form $\_\,\HH_i?\_$ nor of the form $\HH_i\_!\_$ for any $i\in I$.\\
Then, for each $i\in I$, ${q_x}_{\HH_i}={q}_{\HH_i}\not\in\widehat{Q_{\HH_i}}$.
We can hence infer, by definition of projection (\cref{def:projectedconf}), 
that $\restrict{s_x}{\cs}$ is defined.
Moreover, by~\cref{lem:indrestrict}(\ref{lem:indrestrict2cs}),
either $\restrict{s_x}{\cs}\lts{\elle_x}\restrict{s}{\cs}$ or  $\restrict{s_x}{\cs}=\restrict{s}{\cs}$.\\
Since ${q_x}_{\HH_i}\not\in\widehat{Q_{\HH_i}}$ we can apply the induction hypothesis for $s_x$ and obtain $\restrict{s_x}{\cs}\in \RS(\cs)$.
Hence $\restrict{s}{\cs}\in \RS(\cs)$.

%%
%\item
%$s_x\lts{\ttr\tts?\msg[a]}s$, with $\tts\not\in\Set{\HH_i}_{i\in I}$.
%Then there is no move in a gateway and also the channels between gateways
%are untouched. Hence, by definition of projection 
%(\cref{def:projectedconf}), $\restrict{s_x}{\cs} = \restrict{s}{\cs}$.
%In particular, ${q_x}_{\HH_i} = {q}_{\HH_i}\not\in\widehat{Q_{\HH_i}}$ for all $i \in I$.
%Then, by the induction hypothesis for $s_x$, we have $\restrict{s_x}{\cs} \in \RS(\cs)$ and thus $\restrict{s}{\cs} \in \RS(\cs)$. 
%%
%\item
%$s_x\lts{\ttr\tts!\msg[a]}s$, with $\ttr\not\in\Set{\HH_i}_{i\in I}$.\\
%This case is treated as the previous one. 
%
\item

\underline{$\diamond$}
$\elle_x$ is of the form $\_\,\HH_i?\_$  for some $i\in I$.  \\
% $s_x\lts{\ttr\HH_i?\msg[a]}s$, with 
%\ttr\not\in\Set{\HH_i}_{i\in I} and$ $i\in I$.\\
This case cannot apply, since, by definition of gateway, we would get 
$q_{\HH_i}\in \widehat Q_{\HH_i}$ which contradicts our assumption.

\item

\underline{$\diamond$}
$\elle_x$ is of the form $\HH_i\tts!\_$ for some $i\in I$ and $\tts\not\in\Set{\HH_i}_{i\in I}\cup\rolescsint$ (i.e. $\tts\in\roles_i$).\\
More explicitly, let $s_x\lts{\HH_i\tts!\msg[a]}s$. 
%$s_x\lts{\HH_i\tts!\msg[a]}s$, with $i \in I$ and $\tts\not\in\Set{\HH_i}_{i\in I}$.\\
Then, by~\cref{lem:swap-rolf}(\ref{lem:swap-rolf-item3}), there exist
transitions
$s_y \lts{\ttr\HH_i?\msg[a]}s_x\lts{\HH_i\tts!\msg[a]}s$
such that $s_y = (\vec{q_y},\vec{w_y})\in \RS(S)$ is reachable from $s_0$ in $n-2$ steps.
Moreover, by definition of gateways and since $\tts\not\in\Set{\HH_i}_{i\in I}\cup\rolescsint$, we have $\ttr \in\Set{\HH_i}_{i\in I}\cup\rolescsint$ and %\HH_j$ for some $j \in I \setminus \{i\}$. 
%Hence, $s_y \lts{\HH_j\HH_i?\msg[a]}s_x\lts{\HH_i\tts!\msg[a]}s$ where 
$$({q_y}_{\HH_i},\ttr\HH_i?\msg[a],\hat q),
    (\hat q, \HH_i\tts!\msg[a], {q}_{\HH_i})\in\delta_{\HH_i}
\quad\text{ and }\quad {w_y}_{\ttr\HH_i} = \msg[a]\cdot {w}_{\ttr\HH_i}.$$
Then,  by  \cref{fact:uniquesending}(\ref{fact:uniquesending-iiia}),  %definition of gateway and connection policy,
$$(\dot{q_y}_{\HH_i},\ttr\KK_i?\msg[a],\dot{q_{\HH_i}})\in \delta^\cs_{\KK_i}.$$

Since, by definition of gateway, we have
${q_y}_{\HH_i}\not\in\widehat{Q_{\HH_i}}$ and since
${q_y}_{\HH_k} = q_{\HH_k}\not\in\widehat{Q_{\HH_k}}$
for all $k\in I\setminus\{i\}$ (by assumption),
we have also that ${q_y}_{\HH_k}\not\in\widehat{Q_{\HH_k}}$ for each $k\in I$. 
 So $\restrict{s_y}{\cs}$ is defined.
Then we can apply the induction hypothesis for $s_y$ and get 
$\restrict{s_y}{\cs}\in \RS(\cs)$. 

If $\restrict{s_y}{\cs} = (\vec{q'},\vec{w'})$
and $\restrict{s}{\cs} = (\vec{q''},\vec{w''})$ then,
 by definition of projection and the above, we have 
$$q'_{\KK_i} = \dot{{{q}}_y}_{\HH_i} \quad
q''_{\KK_i} = \dot{{{q}}}_{\HH_i} \quad
%\quad\text{and}\quad 
w'_{\ttr\KK_i} = {w_y}_{\ttr\HH_i} = \msg[a]\cdot {w}_{\ttr\HH_i} = \msg[a]\cdot {w''}_{\ttr\KK_i}.$$
Since $(\dot{q_y}_{\HH_i},\ttr\KK_i?\msg[a],\dot{q}_{\HH_i})\in \delta^\cs_{\KK_i}$, we get $\restrict{s_y}{\cs}\lts{\ttr\KK_i?\msg[a]}\restrict{s}{\cs}$ and thus $\restrict{s}{\cs}\in \RS(\cs)$.
%
\item

\underline{$\diamond$}
$\elle_x$ is of the form $\HH_i\tts!\_$ for some $i\in I$ and $\tts\in\Set{\HH_i}_{i\in I}\cup\rolescsint$ (i.e. $\tts\not\in\roles_i$).\\
More explicitly, let $s_x\lts{\HH_i\tts!\msg[a]}s$. 
%$s_y\lts{\HH_i\HH_j!\msg[a]}s$ with $i,j \in I, i\neq j$.\\
Then, by~\cref{lem:swap-rolf}(\ref{lem:swap-rolf-item3}), there exist transitions
$s_y \lts{\ttr\HH_i?\msg[a]}s_x\lts{\HH_i\tts!\msg[a]}s$
such that $s_y = (\vec{q_y},\vec{w_y})\in \RS(S)$ is reachable from $s_0$ in $n-2$ steps.
Moreover, by definition of gateways and since 
$\tts\in\Set{\HH_i}_{i\in I}\cup\rolescsint$, we have $\ttr\in\roles_i$ and
$$({q_y}_{\HH_i},\ttr\HH_i?\msg[a],\hat q),
    (\hat q, \HH_i\tts!\msg[a], q_{\HH_i})\in\delta_{\HH_i}
\quad\text{ and }\quad w_{\HH_i\tts} = {w_y}_{\HH_i\tts}\cdot \msg[a]$$
Then,  by \cref{fact:uniquesending}(\ref{fact:uniquesending-iiib}) %by definition of gateway and connection policy,
$$(\dot{q_y}_{\HH_i} %{\HH_j}
,\KK_i\tts!\msg[a],\dot{q}_{\HH_i})\in \delta^\cs_{\KK_i}.$$

Since, by definition of gateway, we have
${q_y}_{\HH_i}\not\in\widehat{Q_{\HH_i}}$ and since
${q_y}_{\HH_k} = q_{\HH_k}\not\in\widehat{Q_{\HH_k}}$
for all $k\in I\setminus\{i\}$ (by assumption),
we have also that ${q_y}_{\HH_k}\not\in\widehat{Q_{\HH_k}}$ for each $k\in I$. 
So  $\restrict{s_y}{\cs}$ is defined.
Then we can apply the induction hypothesis for $s_y$ and get 
$\restrict{s_y}{\cs}\in \RS(\cs)$. 

If $\restrict{s_y}{\cs} = (\vec{q'},\vec{w'})$
and $\restrict{s}{\cs} = (\vec{q''},\vec{w''})$ then,
 by definition of projection and the above, we have 
$$q'_{\KK_i} = \dot{{{q}}_y}_{\HH_i} \quad
q''_{\KK_i} = \dot{{{q}}}_{\HH_i} \quad
%\quad\text{and}\quad 
w''_{\KK_i\tts} = w_{\HH_i\tts} = {w_y}_{\HH_i\tts}\cdot \msg[a] = {w'}_{\KK_i\tts}\cdot \msg[a].$$

Since $(\dot{q_y}_{\HH_i} %{\HH_j}
,\KK_i\tts!\msg[a],\dot{q}_{\HH_i})\in \delta^\cs_{\KK_i}$,
we get $\restrict{s_y}{\cs}\lts{\KK_i\tts!\msg[a]}\restrict{s}{\cs}$ and thus $\restrict{s}{\cs}\in \RS(\cs)$.
\end{description}
\end{proof}



%Item (\ref{lem:nohatrestrict-b}) of the previous lemma could actually be strengthened \bmc how? \emc to
%$\restrict{s}{\cs}\in RS(\cs)$, but we actually do not need such a stronger statement in
%the following results. The following lemma would in fact suffice when we shall deal with
%a $\restrict{s}{\cs}\in RS(\cs)$ projection in case $({q}_{\HH_k}\not\in\widehat{Q_{\HH_k}}$ for each $k\in I)$ does not hold.

The following lemma will be handy to prove the preservation of reception-error freedom (\cref{prop:nurPreservation}). Roughly, it states that from any reachable configuration we can reach
configurations not containing any of the gateway intermediate states. 
In fact, the single outgoing transitions out of them are, by definition of gateway, output transitions. They can hence be always fired, so increasing the lengths of the corresponding buffers.
\begin{lemma}
\label{lem:addendum}
Let $s = (\vec{q},\vec{w}) \in \RS(S)$. Then there exists $s'= (\vec{q'},\vec{w'}) \in \RS(S)$ such that $s \to^* s'$,
$|w_{\ttp\ttq}| \leq |w'_{\ttp\ttq}|$ for all $\ttp\ttq \in C$
and,  for each $i \in I$, $q'_{\HH_i}\not\in\widehat{Q_{\HH_i}}$ and
 $(q_{\HH_i}\not\in\widehat{Q_{\HH_i}} \implies q'_{\HH_i}= q_{\HH_i})$.
%Let $s = (\vec{q},\vec{w}) \in RS(S)$. % where $S = \MC(\Set{S_i}_{i\in I}, \cs)$.
%Then $s \lts{}^* s'= (\vec{q}',\vec{w}') $ such that 
%${q'}_{\HH_k}\not\in\widehat{Q_{\HH_k}}$ for each $k\in I$.
%Moreover, for each $k\in I$, $q_{\HH_k}\not\in\widehat{Q_{\HH_k}} \implies q'_{\HH_k}= q_{\HH_k}$.
\end{lemma}
\begin{proof}
If ${q}_{\HH_i}\in\widehat{Q_{\HH_i}}$  for each $i\in I$, we are done by setting $s'=s$.
Otherwise, given an $i\in I$ such that  ${q}_{\HH_i}\in\widehat{Q_{\HH_i}}$, 
%\brc : ${q}_{\HH_i}\not\in\widehat{Q_{\HH_i}}$ replaced by
%${q}_{\HH_i}\in\widehat{Q_{\HH_i}}$\erc 
by~\cref{fact:uniquesending}(\ref{fact:uniquesending-i}) we can infer that there exists a configuration transition of the form
$$
s \lts{\HH_k\tts!\msg[a]} s''
$$
such that, for $s'' = (\vec{q''}, \vec{w''})$, it holds that 
$|w_{\ttp\ttq}| \leq |w''_{\ttp\ttq}|$ for all $\ttp\ttq \in C$,
$q''_{\HH_i}\not\in \widehat{Q_{\HH_i}}$ 
and, for each $j\in I$ with $j\neq i$, $q_{\HH_j}\not\in\widehat{Q_{\HH_j}} \implies q''_{\HH_j}= q_{\HH_j}$.
By iterating this sort of transitions, we get the thesis.
\end{proof}







%\bigskip
%Notice that our previous results do not guarantee preservation of  deadlock freedom.
%Assume we had two systems $S_1$ and $S_2$ which are both  deadlock-free in the sense of 
%Definition \ref{def:safeness}(\ref{def:safeness-i})
%but at least one of them is not strongly deadlock free (= no orphan + progress).
%Then our current results would not help to say anything about the composed system.
%In the following subsection we hence provide such a preservation proof












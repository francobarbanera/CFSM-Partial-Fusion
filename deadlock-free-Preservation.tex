%!TEX root = Main-asynchCFSM-multicomp.tex

\section{Preservation results}
\label{sect:presres}
In the following subsections we proceed to prove the preservation by orchestrated multicomposition of the
various communication properties separately.

%\brc
%\\
%I have not seen a point in the following where we need input- or output-deterministic.
%\bfc So it seems to me too. \efc
%\erc

\subsection{Preservation of  deadlock-freedom}

\begin{lemma}[Projections of deadlocks are deadlocks]
\label{lem:weakdfpreservation}
%Let $s= (\vec{q},\vec{w}) \in RS(S)$ be a deadlock configuration of $S$ \bfr and let
%the interfaces of the composed CSs be without any mixed state.\efr 
%\brc The reader could perhaps misunderstand ``the interfaces of the composed CSs'' because the composed CSs yield $S$.
%Perhaps we should be a bit more precise and say: \erc
%\\
Let $S = \MC(\Set{S_i}_{i\in I}, \cs)$ such that the interfaces of each $S_i$,
i.e.\ the CFSMs $M_{h_i}$ in $S_i$, have no mixed state.
Let $s= (\vec{q},\vec{w}) \in \RS(S)$ be a deadlock configuration of $S$.
Then either 
\begin{itemize}
\item
there exists $i\in I$ such that $\restrict{s}{i}\in \RS(S_i)$ and $\restrict{s}{i}$
is a deadlock configuration of $S_i$; or
\item
$\restrict{s}{\cs}\in \RS(\cs)$ and $\restrict{s}{\cs}$
is a deadlock configuration of $\cs$.
\end{itemize}
\end{lemma}

\begin{proof}
By definition of deadlock configuration we have that, for each $i\in I$,
$q_{\HH_i}\not\in\widehat{Q_{\HH_i}}$.
Otherwise there would exist a $j \in I$ such that $q_{\HH_i}\in\widehat{Q_{\HH_i}}$.
But then, by  \cref{fact:uniquesending}(\ref{fact:uniquesending-i}), 
there would be an output transition from $q_{\HH_j}$, contradicting $s$ to be a deadlock configuration of $S$.
So, by~\cref{lem:nohatrestrict}, we get $\restrict{s}{i}\in \RS(S_i)$ for each $i\in I$, as well as $\restrict{s}{\cs}\in \RS(\cs)$.

Now, since $s= (\vec{q},\vec{w})$ is a deadlock configuration of $S$, we have $\vec{w}=\vec{\varepsilon}$ and, for each $\ttp\in\roles$, $q_\ttp$  is a receiving state.
Hence, by definitions of  gateway and multicomposition 
(Definitions \ref{def:gatewaymc} and \ref{def:multicomposition}),
we need  to 
take into account the following cases concerning the shapes of the transitions from  the various $q_{\HH_i}$  in their respective $\delta_{\HH_i}$, i.e.\ the gateway transitions of $\HH_i$. 

\begin{description}  
  \item
\underline{$\diamond$} 
{\em  There exists $v\in I$ and a transition $(q_{\HH_v},\tts{\HH_v}?\_,\_)$ with  some $\tts \in \roles_v$.}\\
 By the no-mixed-state assumption,
the CFSM $M_{\hh_v}$ in $S_v$ has no mixed state.
Therefore, by definition of gateway, all the transitions from $q_{\HH_v}$ in $\delta_{\HH_v}$ 
are of the form  $(q_{\HH_v},\tts{\HH_v}?\_,\_)$ with $\tts \in \roles_v$.\\ %  (and hence $\tts\neq{\HH_v}$).}\\
Then we can infer, again from the definition of gateway, that all transitions from $q_{\HH_v}$
 in $\delta^v_{\HH_v}$, i.e.\ transitions in  $M_{h_v}$, are of the form $(q_{\HH_v},\tts{\HH_v}?\_,\_)$.
  Hence we obtain that $\restrict{s}{v}$ is a deadlock configuration of $S_v$,
 since, for each $\ttu\in\rolescsint$, we have that $q_{\ttu}$ is a receiving state. 
  

\item
\underline{$\diamond$}
{\em  For each $v\in I$  all transitions from $q_{\HH_v}$ in $\delta_{\HH_v}$
are of the form  $(q_{\HH_v},\ttu_v\HH_v?\_,\_)$
where the $\ttu$'s depend on the particular transition and are such that
 $\ttu_v \in \{\HH_i\}_{i \in I\setminus{\{v\}}}\cup\rolescsint$.}\\
Without loss of generality, we consider a 
generic index $v$ and a generic single transition $(q_{\HH},\ttu{\HH_v}?\msg[a],\widehat{q})$.
Now, by definition of gateway, we have that
$$(q_{\HH_v},\ttu{\HH_v}?\msg[a],\widehat{q}),(\widehat{q},\HH_v{\tts}!\msg[a],q'_{\HH_v})\in \delta_{\HH_v} \text{ for some } \tts\in\roles_v.$$
This implies that
$$(q_{\HH_v},\HH_v{\tts}!\msg[a],q'_{\HH_v})\in \delta^v_{\HH_v}.$$
So, by definition of orchestrated connection policy, we can infer that 

$$({\dot {q_{\HH_v}}},\tilde{\ttu}{\KK_v}?\msg[a],\dot {q'_{\HH_v}})\in \delta^{\cs}_{\KK_v},$$
where we recall that $\tilde{\ttu} = \left\{\begin{array}{l@{\quad\text{if }}l}
                                                                \kk & \ttu=\hh\in\Set{\HH_i}_{i \in I}\\
                                                                \ttu & \ttu\in\rolescsint
                                          \end{array}\right.$
       
Since, for each $i\in I$, $q_{\HH_i}\not\in\widehat{Q_{\HH_i}}$, we get that
$\restrict{s}{\cs}$ is defined.                              
Then, by definition of projection, $\restrict{s}{\cs} = (\vec{p},\vec{w'})$
%where $\restrict{\vec{q}}{\cp} = \bfr(p_{\ttu})_{i\in  \roles_{\cp} }\efr$ is the configuration of $\cp$ such that, for each $i \in   I$,
where, for each $i \in   I$, $p_{\kk_i} = \dot{q_{\HH_i}}$  and, for each $\ttw \in   \rolescsint$, $p_{\pw} = q_{\pw}$.
Moreover, 
%$\restrict{\vec{w}}{\cp} =  (w'_{\ttp\ttq})_{\ttp,\ttq\in \roles_{\cp},\ttp\neq\ttq}$
%is such that, 
for each pair $\ttp,\ttq\in \roles_{\cs}$ with $\ttp\neq\ttq$, 
$w'_{\ttp\ttq} = w_{\tilde{\tilde{\ttp}}\tilde{\tilde{\ttq}}}$,
where we recall that
$$\tilde{\tilde{\ttp}} = \left\{\begin{array}{l@{\quad\text{if }}l}
                                                                \hh & \ttp=\kk\in\Set{\kk_i}_{i \in I}\\
                                                                \ttp & \ttp\in\rolescsint
                                          \end{array}\right.$$ 
 
%$\restrict{s}{\cp} = (\vec{p},\vec{w'})$
%such that  $p_{\KK_v}={\dot {q_{\HH}}}$ and $w'_{\KK_j\KK_i} = w_{\HH_j\HH_i}$ for each ${j,i\in I}$ with $j\neq i$.

Since $\vec{w} = \vec{\varepsilon}$, we obtain
 $\vec{w'}=\vec{\varepsilon}$. Moreover, $p_{\KK_v}$ is a receiving state since $p_{\KK_v}={\dot {q_{\HH}}}$ (recall that we considered a generic index $v$, so this holds for each $v\in I$).
 Hence,  $\restrict{s}{\cs}\in \RS(\cs)$ is a deadlock configuration of $\cs$
 since we have also that, for each $\ttu\in\rolescsint$, $q_{\ttu}$ is a receiving state.
 \end{description}
 \end{proof}
%{\em  For each $v\in I$  all transitions from $q_{\HH_v}$ in $\delta_{\HH_v}$
%are of the form $(q_{\HH_v},\HH'\HH_v?\_,\_)$
%with some $\HH' \in \{\HH_i\}_{i \in I\setminus{\{v\}}}$.}\\
%W.l.o.g.\ we consider a 
%generic single transition $(q_{\HH},\HH'{\HH_v}?a,\widehat{q})$.
%Now, by definition of gateway, we have that
%$$(q_{\HH_v},\HH'{\HH_v}?a,\widehat{q}),(\widehat{q},\HH_v{\tts}!a,q'_{\HH_v})\in \delta_{\HH_v} \text{ for some } \tts\in\roles_v.$$
%This implies that
%$$(q_{\HH_v},\HH_v{\tts}!a,q'_{\HH_v})\in \delta^v_{\HH_v}.$$
%So, by definition of connection policy, we can infer that 
%$$({\dot {q_{\HH_v}}},\KK'{\KK_v}?a,\dot {q'_{\HH_v}})\in \delta^{\cp}_{\KK_v}.$$
%By definition of projection, $\restrict{s}{\cp} = (\vec{p},\vec{w'})$
%such that  $p_{\KK_v}={\dot {q_{\HH}}}$ and $w'_{\KK_j\KK_i} = w_{\HH_j\HH_i}$ for each ${j,i\in I}$ with $j\neq i$.
%Since $\vec{w} = \vec{\varepsilon}$ we obtain
% $\vec{w'}=\vec{\varepsilon}$ and $p_{\KK_v}$ is a receiving state since $p_{\KK_v}={\dot {q_{\HH}}}$.
% Hence,  $\restrict{s}{\cp}\in \RS(\cp)$ is a deadlock configuration of $\cp$.
%\end{description}
%
%
% OLD version of the proof
%
%By definition of deadlock configuration and by Fact \ref{fact:uniquesending}(\ref{fact:uniquesending-i}), we have that, for each $i\in I$,
%$q_{\HH_i}\not\in\widehat{Q_{\HH_i}}$.
%Otherwise there would be an output transition from $q_{\HH_i}$, contradicting $s$ to be a deadlock configuration.
%So, by Lemma \ref{lem:nohatrestrict} we get $\restrict{s}{i}\in RS(S_i)$ for each $i\in I$,
%as well as $\restrict{s}{\cp}\in RS(\cp)$.
%
%Now, since $s$ is a deadlock configuration, we have \\
%\centerline{$\vec{w}=\vec{\varepsilon}$ and 
%$\forall \ttp\in\roles.~q_\ttp$  is a receiving state,}
%where $s=(\vec{q},\vec{w})$.\\
%By definitions of  gateway and multicomposition 
%(Defs \ref{def:gatewaymc} and \ref{def:multicomposition}) and 
%\bfr since all states of the configuration $s$ are receiving states, \efr
%\bfc I erased ``by the no mixed state condition on each $M_{\HH_i}$
%imposed by composability''\efc
% we need  to 
%take into account only the following cases concerning the shapes of the transitions from  the various $q_{\HH_i}$  in their respective $\delta_{\HH_i}$. 
%
%\begin{description}
%%\item
%%\underline{$\diamond$} 
%%{\em  All the transitions from $q_\HH$ in $\delta_\HH$ are of the form $(q_\HH,\KK\HH?\_,\_)$ and
%%all the transitions from $q_\KK$ in $\delta_\KK$ are of the form $(q_\KK,\tts\KK?\_,\_)$ with $\tts \in \roles_2$ (and hence $\tts\neq\HH$).}\\
%%Since all the transitions from $q_\KK$ in $\delta_\KK$ are of the form $(q_\KK,\tts\KK?\_,\_)$ with $\tts \in \roles_2$ (and hence $\tts\neq\HH$),
%%  we can infer, from the definition of  $\gateway{\cdot}$, that also  all the transitions from $q_\KK$ in $\delta^2_\KK$ are of the form $(q_\KK,\tts\KK?\_,\_)$. Hence we obtain that $\restrict{s}{2}$ is a deadlock configuration of $S_2$.
%  
%  \item
%\underline{$\diamond$} 
%{\em  There exists a $v\in I$ such that all the transitions from $q_{\HH_v}$ in $\delta_{\HH_v}$ 
%are of the form  $(q_{\HH_v},\tts{\HH_v}?\_,\_)$ with  $\tts \in \roles_v$  (and hence $\tts\neq{\HH_v}$).}\\
%In this case
%  we can infer, from the definition of gateway, that also  all the transitions from $q_{\HH_v}$
% in $\delta^v_{\HH_v}$ are of the form $(q_{\HH_v},\tts{\HH_v}?\_,\_)$. Hence we obtain that $\restrict{s}{v}$ is a deadlock configuration of $S_v$.
%
%\item
%\underline{$\diamond$}
%{\em  For each $v\in I$  the transitions from $q_{\HH_v}$ in $\delta_{\HH_v}$ 
%are of the form $(q_{\HH_v},\HH_{\bmr z_{v\hspace{0.3pt}a}\emr}{\HH_v}?\_,\_)$}\\
%In order not to cope with too many indexes and indexed indexes, we consider a 
%generic single transition $(q_{\HH},\HH'{\HH}?a,\widehat{q})$
%where we have set $\HH =\HH_v$, $\HH'=\HH_{z_{v\hspace{0.3pt}a}}$ (i.e $\HH'$ depends on $v$ and $a$),
%$a=a_v$ and $\widehat{q}=\widehat{q}_{j_{v\hspace{0.3pt}a}}$.
%Now, by definition of gateway, we have that
%$$(q_{\HH},\HH'{\HH}?a,\widehat{q}),(\widehat{q},\HH{\tts}!a,q'_{\HH})\in \delta_{\HH}$$
%for some $\tts\in\roles_v$ (where $\tts$ depends on $v$ and $a$).\\
%This implies that
%$$(q_{\HH},\HH{\tts}!a,q'_{\HH})\in \delta^v_{\HH}.$$
%So, by definition of connection policy, we can infer that 
%$$({\dot {q_{\HH}}},\KK'{\KK_v}?a,\dot {q'_{\HH}})\in \delta^{\cp}_{\KK_v}.$$
%We note that by definition of projection, if $\restrict{s}{\cp} = (\vec{p},\vec{w'})$,
% we have that $w'_{\KK_j\KK_i} = w_{\HH_j\HH_i}$, for each ${j,i\in I}$ such that $j\neq i$, namely
% $\vec{w'}=\vec{\varepsilon}$. The thesis hence follows because we have shown above that
% $\restrict{s}{\cp}\in RS(\cp)$ and, by definition of projection, $p_{\KK_v}={\dot {q_{\HH}}}$.
%\end{description}
%\end{proof}

\begin{corollary}[Preservation of deadlock-freedom]%\hfill\\
\label{prop:weakdfPreservation}
Let $S = \MC(\Set{S_i}_{i\in I}, \cs)$ such that, for each $i\in I$, $S_i$  is deadlock-free and $\cs$ is deadlock-free.
Moreover, let the interfaces of each $S_i$,
i.e.\ the CFSMs $M_{h_i}$ in $S_i$, have no mixed state.
Then $S$ is deadlock-free.
\end{corollary}
\begin{proof}
By contradiction, let us assume there is an $s\in \RS(S)$ which is a deadlock configuration of $S$. Then we get a contradiction by Lemma \ref{lem:weakdfpreservation}.
\end{proof}











